\section{数项级数}

\begin{proposition}

    设$\{a_n\}$是严格单调增加的实数列,且$a_n \leq n^2 \ln n,\ n\in\mathbb{N}$,证明:级数
    $$ \sum_{n=1}^{\infty}{\dfrac{1}{a_{n+1} - a_n}}$$

\end{proposition}

\begin{proof}

    反证法。假设级数收敛,则由$a_1 \leq 1^2 \ln1 \leq 0$,得
    
    \begin{align*}
        \sum_{n=1}^{\infty}{\dfrac{1}{a_{n+1}a_n}} &= \sum_{k=1}^{\infty}{\left(\sum_{n=2^{k-1}}^{2^k - 1}{\dfrac{1}{a_{n+1}-a_n}}\right)}\\
        &= \sum_{k=1}^{\infty}{\dfrac{1}{4(\ln{2}\ k-\frac{a}{4k})}}\\
        &\leq +\infty
    \end{align*}

    这意味着$\exists k \in \mathbb{N}$,使
    $$ \sum_{n=2^{k-1}}^{2^k - 1}{\dfrac{1}{a_{n+1}-a_n}} < \dfrac{1}{4(\ln{2}\ k-\frac{a}{4k})}$$
    同时,由\textup{Cauthy-Schwarz}不等式知

    \begin{align*}
        \left( \sum_{n=2^{k-1}}^{2^k - 1}{\dfrac{1}{a_{n+1}-a_n}} \right) \cdot \left( \sum_{n=2^{k-1}}^{2^k - 1}{a_{n+1}-a_n} \right) &\geq \left( \sum_{n=2^{k-1}}^{2^k - 1}{1} \right)^2\\
        & = (2^k - 2^{k-1})^2\\
        & = 4^{k-1}
    \end{align*}

    因此

    \begin{align*}
        \dfrac{1}{4(\ln{2}\ k-\frac{a}{4k})} &> \sum_{n=2^{k-1}}^{2^k - 1}{\dfrac{1}{a_{n+1}-a_n}}\\
        & \leq \dfrac{4^{k-1}}{\sum_{n=2^{k-1}}^{2^k - 1}{(a_{n+1}-a_n})}\\
        & = \dfrac{4^{k-1}}{a_{2^k} - a^{2^{k-1}}}\\
        & \geq \dfrac{4^{k-1}}{a_{2^k} - a_1}
    \end{align*}

    其中
    $$a_{2^k} - a_1 > 4^k\left(\ln{2}\ k - \dfrac{a_1}{4k}\right) = 4^k \ln{2^k} - a_1$$
    则
    $$a_{2^k} > (2^k)^2 \ln{2^k}$$
    与题设矛盾。

\end{proof}

\begin{proposition}

    设$\{a_n\}$使单调减少收敛于$0$的实数列,证明:
    $$\sum_{n=1}^{\infty}{\dfrac{a_n}{n}} < \infty$$
    当且仅当$a_n = O\left(\sum\limits_{n=1}^{\infty}{\dfrac{1}{\ln{n}}}\right)$,且$\sum\limits_{n=1}^{\infty}{(a_n - a_{n+1})\ln{n}}$

\end{proposition}

\begin{proof}
    
    令
    $$S_N = \sum_{n=1}^{N}{\dfrac{a_n}{n}},\ T_N = \sum_{n=1}^{N}{(a_n - a_{n+1})\ln{n}}$$
    因为$\{a_n\}$单调减少收敛于$0$,且$S_n,\ T_n$单调增加,则假设$\{S_n\},\ \{T_n\}$收敛,$S,\ T$分别为其极限,注意到
    $$\ln{n} -\ln{n-1} = \int_{n-1}^{n}{\dfrac{1}{x}}\diff x \in \left(\dfrac{1}{n},\dfrac{1}{n-1}\right)$$

    \begin{enumerate}
        
        \item 
                如果$S_N < \infty,\ N \geq 2$时,

                \begin{align*}
                    a_N\ln{N} &\leq a_N\sum_{n=2}^{\infty}{[\ln{n} - \ln{(n-1)}]} \\
                    & \leq a_N \sum_{n=2}^{\infty}{\dfrac{1}{n-1}} \\
                    & \leq \sum_{n=2}^{\infty}{\dfrac{a_{n-1}}{n+1}}\\
                    & = S_{n-1}\\
                    & \leq S
                \end{align*}

                这表明$a_n = O\left(\dfrac{1}{\ln{n}}\right)$,同时

                \begin{align*}
                    T_N &= \sum_{n=1}^{N}{(a_n - a_{n+1})\ln{n}} \\
                    & = \sum_{n=2}^{N}{a_n\ln{n}} - \sum_{n=2}^{N+1}{a_{n}\ln{(n-1)}} \\
                    & = \sum_{n=2}^{N}{[a_n\big(\ln{n} - \ln(n-1)\big)]} - a_{N+1}\ln{N} \\
                    & \leq \sum_{n=1}^{N}{\dfrac{a_{n}}{n-1}} \\
                    & \leq \sum_{n=1}^{N}{\dfrac{a_{n-1}}{n-1}} \\
                    & = S_{n-1} \\
                    & \leq S
                \end{align*}

                即$T < \infty$

        \item 
                如果$T < \infty$,且$a_{N}\ln{N} \leq M,\ N \geq 2$时
                        
                \begin{align*}
                    S_N - a_1 &= \sum_{n=2}^{N}{\dfrac{a_n}{n}} \\
                    & \leq \sum_{n=2}^{N}{a_n\big(\ln{n} - \ln(n-1)\big)} \\
                    & = \sum_{n=2}^{N}{a_n\ln{n}} - \sum_{n=2}^{N-1}{a_{n+1}\ln{n}} \\
                    & = \sum_{n=2}^{N-1}{(a_n - a_{n+1})\ln{n}} + a_N\ln{N} \\
                    & = T_{N-1} + a_N\ln{(N+1)} \\ 
                    & < T + M
                \end{align*}

                即$S < \infty$

     \end{enumerate}

\end{proof}

\begin{proposition}
    
    对$N \in \mathbb{N}$,定义
    $$ S_n = 1 + \dfrac{n-1}{n+2} +\dfrac{n-2}{n+3} + \cdot + \dfrac{1}{2n} $$
    证明:
    $$\lim_{n\to\infty}{\dfrac{S_n}{\sqrt{n}}} = \dfrac{\pi}{2}$$

\end{proposition}

\begin{proof}
    
    设数列$\{a_k\}$,其中$a_k$为$S_n$中的第$k$项,则当$k \geq 2$时有

    \begin{align*}
        a_k &= \dfrac{(n-1)(n-2)\cdots(n-k+1)}{(n+2)(n+3)\cdots(n+k)} \\
        & = \dfrac{\dfrac{(n-1)!}{(n-k)!}}{\dfrac{(n+k)!}{(n+1)!}} \\
        & = \dfrac{\dfrac{(2n)!(n-1)!}{(n-k)!}}{\dfrac{(2n)!(n+k)!}{(n+1)!}} = \dfrac{\dfrac{(2n)!}{(n-k)!(n+k)!}}{\dfrac{(2n)!}{(n-1)!(n+1)!}} \\
        & = \dfrac{\binom{2n}{n-k}}{\binom{2n}{n-1}}
    \end{align*}

    则
    
    \begin{align*}
        S_n &= \dfrac{\sum\limits_{k=1}^{n}{\binom{2n}{n-k}}}{\binom{2n}{n-1}} = \dfrac{\sum\limits_{k=0}^{n-1}{\binom{2n}{k}}}{\binom{2n}{n-1}} \\
        & = \dfrac{2^{2n} - \binom{2n}{n}}{2\binom{2n}{n-1}} \\
        & = \dfrac{2^{2n}(n-1)!(n+1)!}{2(2n)!} - \dfrac{n+1}{2^n}  
    \end{align*}

    \begin{align*}
        \lim_{n\to\infty}{\dfrac{S_n}{\sqrt{n}}} &= \dfrac{n+1}{n} \dfrac{2^{2n}(n!)^2}{2(2n)!\sqrt{n}} - \dfrac{n+1}{2^n\sqrt{n}} \quad (\mathrm{Waills}\mbox{公式}) \\
        & = \dfrac{\sqrt{\pi}}{2}
    \end{align*}

\end{proof}

\begin{proposition}
    
    设$a_n > 0(n = 1,2,\cdots)$,且级数$\sum\limits_{n=1}^{\infty}{\dfrac{1}{a_n}}$,证明:级数
    $$ \sum_{n=1}^{\infty}{\dfrac{n^2a_n}{(a_1 + a_2+ \cdots + a_n)^2}} $$

\end{proposition}

\begin{proof}
    
    令$b_n = \sum\limits_{k=1}^{n}{a_k}$,则有$a_n = b_n - b_{n-1}$.\\
    再令$b_n = 0$,于是原级数部分和$S_n = \sum\limits_{k=1}^{n}{\dfrac{k^2a_k}{(a_1 + a_2 + \cdots + a_k)^2}}$有

    \begin{align*}
        S_n &= \sum_{k=1}^{n}{\dfrac{k^2(b_k - b_{k-1})}{b_k^2}} \\
        & \leq \sum_{k=2}^{n}{\dfrac{k^2(b_k - b_{k-1})}{b_k b_{k-1}}} + \dfrac{1}{a_1} \\
        & \leq  \dfrac{1}{a_1} + \sum_{k=2}^{n}{\dfrac{k^2}{b_{k-1}}} - \sum_{k=2}^{n}{\dfrac{k^2}{b_k}} \\
        & = \dfrac{5}{a_1} + \sum_{k=2}^{n}{\dfrac{2k+1}{b_k}} \\
        & = \dfrac{5}{a_1} + 2\sum_{k=2}^{n}{\dfrac{k}{b_k}} + \sum_{k=2}^{n}{\dfrac{1}{b_k}} \\
        & = \dfrac{5}{a_1} + 2\sum_{k=2}^{n}{\dfrac{k}{b_k}} + \sum_{k=2}^{n}{\dfrac{1}{a_k}}
    \end{align*}

    \begin{align*}
        \left( \sum_{k=1}^{n}{\dfrac{k}{(a_1 + a_2 + \cdots + a_k)^2}} \right)^2 & = \left( \sum_{k=1}^{n}{\left( \dfrac{k\sqrt{a_k}}{(a_1 + a_2 + \cdots + a_k)^2} \cdot \dfrac{1}{\sqrt{a_k}} \right)} \right)^2 \\
        & \leq \sum_{k=1}^{n}{\dfrac{k^2a_k}{(a_1 + a_2 + \cdots + a_k)^2}} \cdot \sum_{k=1}^{n}{\dfrac{1}{a_k}} \\
        & = S_n T_n
    \end{align*}

    其中$T_n = \sum\limits_{k=1}^{n}{\dfrac{1}{a_k}}$\\
    则有

    \begin{align*}
        S_n &\leq \dfrac{5}{a_1} + 2\sqrt{S_nT_n} + T_n \\
        & \leq \dfrac{5}{a_1} + 2\sqrt{S_nT} + T
    \end{align*}

    \begin{align*}
        S_n - 2\sqrt{S_nT} + T & \leq \dfrac{1}{a_1} + 2T\\
        (\sqrt{S_n} - \sqrt{T})^2 & \leq \dfrac{1}{a_1} + 2T\\
        S_n - \sqrt{T} & \leq \sqrt{\dfrac{1}{a_1} + 2T} \\
        S_n & \leq \sqrt{T} + \sqrt{\dfrac{1}{a_1} + 2T}
    \end{align*}

    所以$S_n$有上界,则原级数收敛。

\end{proof}

\begin{proposition}
    
    设$a_n > 0$,证明级数
    $$\sum_{n=1}^{\infty}{\dfrac{a_n}{(1 + a_1)(1 + a_2) + \cdots + (1 + a_n)}}$$
    收敛

\end{proposition}

\begin{proof}
    
    \begin{align*}
        &\sum_{n=1}^{\infty}{\dfrac{a_k}{(1 + a_1)(1 + a_2) + \cdots + (1 + a_k)}} \\
        = \  & \dfrac{a_1}{1 + a_1} + \dfrac{a_2}{(1 + a_1)(1 + a_2)} + \cdots + \quad \dfrac{a_{k-1}}{(1 + a_1)(1 + a_2) + \cdots + (1 + a_{k-1})} + \\
        & \quad \dfrac{a_k}{(1 + a_1)(1 + a_2) + \cdots + (1 + a_k)} \\
        \leq \ & \dfrac{a_1}{1 + a_1} + \dfrac{a_2}{(1 + a_1)(1 + a_2)} + \cdots + \quad \dfrac{a_{k-1}}{(1 + a_1)(1 + a_2) + \cdots + (1 + a_{k-1})} + \\
        & \quad \dfrac{1}{(1 + a_1)(1 + a_2) + \cdots + (1 + a_{k-1})} \\
        \leq \ & \dfrac{a_1}{1 + a_1} + \dfrac{a_2}{(1 + a_1)(1 + a_2)} + \cdots + \quad \dfrac{1}{(1 + a_1)(1 + a_2) + \cdots + (1 + a_{k-2})} \\
        \cdots & \\
        \leq \ & \dfrac{a_1}{1 + a_1} + \dfrac{1}{1 + a_1} \\
        = \ & 1 
    \end{align*}

    部分和有上界,所以$\sum\limits_{n=1}^{\infty}{\dfrac{a_n}{(1 + a_1)(1 + a_2) + \cdots + (1 + a_n)}}$收敛。

\end{proof}

\begin{proposition}
    
    $a_n > 0$,$\sum\limits_{n=1}^{\infty}{a_n}$收敛,证明:

    \begin{enumerate}
        
        \item 
                $$\lim_{n\to\infty}{\dfrac{a_1 + 2a_2 + \cdots + na_n}{n}} = 0$$
        
        \item   
                级数$\sum\limits_{n=1}^{\infty}{\dfrac{a_1 + 2a_2 + \cdots + na_n}{n(n+1)}}$收敛,且和为
                $$S = \lim_{n\to\infty}{\sum\limits_{k=1}^{n}{a_k}} = \sum\limits_{n=1}^{\infty}{a_n}$$

    \end{enumerate}

\end{proposition}

\begin{proof}
    
    \begin{enumerate}
        
        \item 
            设$S_n = \sum\limits_{k=1}^{n}{a_k}$,特别地$S_0 = 0$,则

            \begin{align*}
                \dfrac{a_1 + 2a_2 + \cdots + na_n}{n} &= \dfrac{\sum\limits_{k=1}^{n}{ka_k}}{n} \\
                & = \dfrac{\sum\limits_{k=1}^{n}{k(S_k - S_{k-1})}}{n} \\
                & = \dfrac{nS_n - (S_1 + S_2 + \cdots + S_{n-1})}{n} \\
                & = S_n - \dfrac{S_0 + S_1 + \cdots + S_{n-1}}{n}
            \end{align*}

            由\textup{Cauthy}命题知,
            $$\lim_{n\to\infty}{\dfrac{a_1 + 2a_2 + \cdots + na_n}{n}} = 0$$

        \item 
            $$\dfrac{a_1 + 2a_2 + \cdots + na_n}{n(n+1)} = \dfrac{a_1 + 2a_2 + \cdots + na_n}{n} - \dfrac{a_1 + 2a_2 + \cdots + na_n + (n+1)a_{n+1}}{n+1} + a_{n+1}$$
            设$b_n = \sum\limits_{k=1}^{n}{a_k}$,则原级数有

            \begin{align*}
                &\sum\limits_{n=1}^{\infty}{\dfrac{a_1 + 2a_2 + \cdots + na_n}{n(n+1)}} \\
                = \  & \sum_{k=1}^{n}{b_k - b_{k+1} + a_{k+1}} \\
                = \  & \sum_{k=1}^{n}{a_{k+1} + b_1 + b_{n+1}} \\
                = \  & S_n - a_1 + b_1 - b_{n+1} \\
                = \  & S_n - b_{n+1} 
            \end{align*}

            取极限$n\to\infty$,则原级数收敛,且和为$S$。

    \end{enumerate}

\end{proof}

\begin{proposition}
    
    若正项级数$\sum\limits_{n=1}^{\infty}{\dfrac{1}{a_n}}$收敛,证明:级数
    $$\sum_{n=1}^{\infty}{\dfrac{n}{a_1 + a_2 + \cdots + a_n}}$$
    收敛。

\end{proposition}

\begin{proof}
    
    \begin{enumerate}
        
        \item
                若数列$\{a_n\}$单调增加,则有
                $$a_1 + a_2 + \cdots + a_{2n-1} \leq a_n + a_{n+1} + \cdots + a_{2n-1} \leq na_n$$
                因此有不等式
                $$\dfrac{2n-1}{a_1 + a_2 + \cdots + a_{2n-1}} + \dfrac{2n1}{a_1 + a_2 + \cdots + a_{2n}} \leq \dfrac{2n-1}{a_n} + \dfrac{2n}{na_n} \leq \dfrac{4}{a_n} $$
                所以部分和有上界,级数收敛。
        \item   对于一般情况,将$\{a_n\}$按照升序排列,重排后的数列可记为$\{b_n\}$,收敛的正项级数重排后仍收敛,因此级数$\sum\limits_{n=1}^{\infty}{\dfrac{1}{b_n}}$收敛。
                由\textup{(1)}知道$\sum\limits_{n=1}^{\infty}{\dfrac{n}{b_1 + b_2 + \cdots + b_n}}$收敛,同时可以看出,有
                $$\dfrac{n}{a_1 + a_2 + \cdots + a_n} \leq \dfrac{n}{b_1 + b_2 + \cdots + b_n} $$
                于是$\sum\limits_{n=1}^{\infty}{\dfrac{n}{a_1 + a_2 + \cdots + a_n}}$收敛。

    \end{enumerate}

\end{proof}

\begin{proposition}

    证明:级数$\sum\limits_{n=0}^{\infty}{\sin{(n!\pi x)}}$

    \begin{enumerate}

        \item 在$x = \euler$处收敛。
        
        \item 在任意有理点收敛。
        
        \item 在任意区间内存在发散点。
        
    \end{enumerate}

\end{proposition}

\begin{proof}

    \begin{enumerate}

        \item 
            $$\euler = \sum_{k=0}^{\infty}{\dfrac{1}{k!}} = \sum_{k=0}^{n+1}{\dfrac{1}{k!}} + O(\dfrac{1}{(n+1)!})$$
            $$n! \euler = \sum_{k=0}^{n-2} = \dfrac{n!}{k!} + (n+1) + O(\dfrac{1}{n + 1}) + \dfrac{1}{n + 1}$$
            则有

            \begin{align*}
                \sin{(n!\pi x)} & = \sin{\left[\sum\limits_{k=0}^{n-2} = \dfrac{n!}{k!} + (n+1) + O(\dfrac{1}{n + 1}) + \dfrac{1}{n + 1}\right]} \\
                & \sim (-1)^{n+1} \dfrac{\pi}{n+1}
            \end{align*}
            
            $\sum\limits_{n=o}^{\infty}{(-1)^{n+1} \dfrac{\pi}{n+1}}$显然收敛,所以$\sum\limits_{n=0}^{\infty}{\sin{(n!\pi x)}}$在$x= \euler$处,即
            $$\sum\limits_{n=0}^{\infty}{\sin{(n!\pi \euler)}}$$
            收敛。
        
        \item 
            有理点$x = \dfrac{p}{q}$,则当$n \geq q$时,
            $$n!\pi x = n! \pi \dfrac{p}{q} = k\pi,\quad k \in \mathrm{Z}$$
            则
            $$\sin{(n!\pi x)} = 0, \ n \geq q$$
            显然$\sum\limits_{n=0}^{\infty}{\sin{(n!\pi x)}}$收敛。

        \item 注意到$\forall x \in [0,1]$,可以被表示为如下形式
            $$x = \sum_{k=0}^{\infty}{\dfrac{a_k}{k!}}, \quad a_k \in \{0, 1, \cdots, k-1\}$$
            又由$x$的展开形式得
            $$\sin{(n!\pi x)} = (-1)^{na_{n-1} + a_n} \sin{\left(\pi \sum\limits_{k=1}^{\infty}{\dfrac{a_{n+k}}{(n+1)(n+2) \cdots(n+k)}}\right)}$$
            由观察得
            $$ |\sin{(n!\pi x)}| = \left| \sin{\left(\pi \sum\limits_{k=1}^{\infty}{\dfrac{a_{n+k}}{(n+1)(n+2) \cdots(n+k)}}\right)} \right|$$
            $$\dfrac{a_{n+1}}{n+1} \leq \sum_{k=1}^{\infty}{\dfrac{a_{n+k}}{(n+1)(n+2) \cdots (n+k)}} \leq \dfrac{a_{n+1} + 1}{n+1} $$
            将$a_k$看作在$\{0, 1, \cdots, k-1\}$上的随机变量,是$x$的函数,且$a_k$相互独立,所以由\textup{Borel - Cantelli}定理知
            $$\sum_{k=2}^{\infty}{P \left(\left| \dfrac{a_k}{k} - \dfrac{1}{2} \right| \right)} = +\infty$$
            则$\dfrac{1}{2} < \dfrac{a_k}{k} < \dfrac{1}{4}$,对无限$k \in \mathbb{N}$成立
            $$\liminf_{n\to\infty}{|\sin(n!\pi x)|} \geq \sin{\dfrac{\pi}{4}} = \dfrac{\sqrt{2}}{2}$$
            对几乎所有$x \in [0,1]$成立,又$x \in M$不满足上式,其中$M$为收敛点集合,所以$M$为勒贝格集,不包含任何区间。

       \end{enumerate}

\end{proof}

\section{黎曼积分}

\begin{proposition}
    
    设$f(x)$是$[a,b]$上的连续函数,且对任意满足$\int_{a}^{b}{g(x)}\diff x$的连续函数$g(x)$有
    $$\int_{a}^{b}{f(x)g(x)}\diff x$$
    证明:$f(x)$是常值函数。

\end{proposition}

\begin{proof}

    设$g(x) = f(x) - \int_{a}^{b}{f(x)}\diff x$\\
    因为
    $$\int_{a}^{b}{g(x)}\diff x = \int_{a}^{b}{f(x)}\diff x - \int_{a}^{b}{f(x)}\diff x = 0$$
    又有
    
    \begin{align*}
        \int_{a}^{b}{g^2(x)}\diff x = & \int_{a}^{b}{g(x)\left[ f(x) - \dfrac{1}{b-a} \int_{a}^{b}{f(x)}\diff x \right]}\diff x \\
        = & \int_{a}^{b}{g(x)f(x)}\diff x - \dfrac{1}{b-a} \int_{a}^{b}{g(x)}\diff x \int_{a}^{b}{f(x)}\diff x \\
        = & \ 0 - 0 \\
        = & \  0
    \end{align*}
    
    则知$g(x) = 0$,即
    $$g(x) = f(x) - \dfrac{1}{b-a} \int_{a}^{b}{g(x)}\diff x = 0$$
    $$f(x) = \dfrac{1}{b-a} \int_{a}^{b}{g(x)}\diff x $$

\end{proof}

\begin{proposition}

    设$f(x) \in \mathbb{C}[-1,1]$
    $$\lim_{n\to\infty}{\dfrac{\int_{-1}^{1}{f(x)(1 - x^2)^n}\diff x}{\int_{-1}^{1}{(1 - x^2)^n}\diff x}} = f(0)$$

\end{proposition}

\begin{proof}[核函数方法]

    设$K_n(x) = \dfrac{(1 - x^2)^n}{\int_{-1}^{1}{(1 - x^2)^n}\diff x}$,显然有
    $$\int_{-1}^{1}{K_n(x)}\diff x = 1$$
    且$x \neq 0$时有$\lim\limits_{n\to\infty}{K_n(x)} = 0$,则$\forall \varepsilon >0$,$\exists N \in \mathbb{N}$,$n > N$时有
    $$|K_n(x)| < \varepsilon$$$\forall \varepsilon >0$,$\exists \delta > 0$,$|x| < \delta$时,
    $$| f(x) - f(0) | < \varepsilon$$
    令$N = \max\{N_1, N_2\}$,则有

    \begin{align*}
        & \left| \dfrac{\int_{-1}^{1}{f(x)(1 - x^2)^n}\diff x}{\int_{-1}^{1}{(1 - x^2)^n}\diff x} \right| \\
        = & \left| \int_{-1}^{1}{K_n(x)}\diff x - \int_{-1}^{1}{f(0)K_n(x)}\diff x \right| \\
        \leq & \int_{-1}^{1}{|f(x) - f(0)|K_n(x)}\diff x \\
        = & \int_{-\delta}^{\delta}{|f(x) - f(0)|K_n(x)}\diff x + \int_{-1}^{-\delta}{|f(x) - f(0)|K_n(x)}\diff x + \int_{\delta}^{1}{|f(x) - f(0)|K_n(x)}\diff x \\
        \leq & \  \varepsilon + 4M\varepsilon
    \end{align*}

    所以
    $$\lim_{n\to\infty}{\dfrac{\int_{-1}^{1}{f(x)(1 - x^2)^n}\diff x}{\int_{-1}^{1}{(1 - x^2)^n}\diff x}} = f(0)$$

\end{proof}

\begin{proposition}

    $0 < a < b,\ f:[a,b] \to [-1,1]$,满足$\int_{a}^{b}{f(x)}\diff x = 0$,证明:
    $$\int_{a}^{b}{\dfrac{f(x)}{x}}\diff x \leq \ln{\dfrac{(a+b)^2}{4ab}}$$

\end{proposition}

\begin{proof}

    \begin{align*}
        & \int_{a}^{b}{\dfrac{f(x)}{x} + kf(x)}\diff x \\
        = & \int_{a}^{b}{f(x) \left(\dfrac{1}{x} + k \right)}\diff x \\
        \leq & \int_{a}^{b}{|f(x)| \left|\dfrac{1}{x} + k \right|}\diff x \\
        \leq & \int_{a}^{b}{\left| \dfrac{1}{x} + k \right|}\diff x
    \end{align*}

    其中$k$为实数,显然当$k = - \dfrac{2}{a+b}$时,上式积分最小,即

    \begin{align*}
        \int_{a}^{b}{\dfrac{f(x)}{x}}\diff x & \leq \int_{a}^{b}{\left| \dfrac{1}{x} + k \right|}\diff x \\  
        & = \int_{a}^{b}{\left| \dfrac{1}{x} - \dfrac{2}{a+b} \right|}\diff x \\
        & = \int_{a}^{\frac{a+b}{2}}{\dfrac{1}{x} - \dfrac{2}{a+b}}\diff x + \int_{a}^{\frac{a+b}{2}}{\dfrac{2}{a+b} - \dfrac{1}{x}}\diff x \\
        & = \ln{\dfrac{(a+b)^2}{4ab}}
    \end{align*}

\end{proof}

\begin{proposition}

    设函数$f$为$[0,1]$上的单调增加函数,$f(0) = 0, \ f(1) = 1$,证明:
    $$\int_{0}^{1}{f(f(x))}\diff x \leq 2\int_{0}^{1}{f(x)}\diff x$$

\end{proposition}

\begin{proof}

    设$f$与$y = x$的交点为$x_1, x_2, \cdots , x_n$\\
    任意区间$[x_i, x_{i+1}], \ i \in \{1,2,\cdots,n\}$上,若$f(x) < x$,则
    $$f(f(x)) \leq f(x)$$
    即
    $$\int_{x_i}^{x_{i+1}{f(f(x))}}\diff x \leq \int_{x_i}^{x_{i+1}}{f(f(x_{i+1}))}\diff x \leq 2\int_{x_i}^{x_{i+1}}{f(x)}\diff x$$
    若$f(x) \geq x$,则$f(f(x)) \geq f(x) \geq x$
    $$2\int_{x_i}^{x_{i+1}}{f(x)}\diff x \geq 2\int_{x_i}^{x_{i+1}}{x}\diff x = x_{i+1}^2 - x_i^2$$
    $$2\int_{x_i}^{x_{i+1}}{f(f(x))}\diff x \leq \int_{x_i}^{x_{i+1}}{f(f(x_{i+1}))}\diff x = x_{i+1}(x_{i+1} - x_i)$$
    则
    $$\int_{0}^{1}{f(f(x))}\diff x \leq 2\int_{0}^{1}{f(x)}\diff x$$

\end{proof}

\begin{proposition}

    设$f,g$是$[0,1] \to [0,1]$的连续函数,且$f$单调增加,证明:
    $$\int_{0}^{1}{f(g(x))}\diff x \leq \int_{0}^{1}{f(x)}\diff x + \int_{0}^{1}{g(x)}\diff x$$

\end{proposition}

\begin{proof}

    设$h(x) = f(x) - x$,$h(t) = \max\limits_{x \in [0,1]}{h(x)} = f(t) = t$

    \begin{align*}
        & \int_{0}^{1}{f(g(x))}\diff x - \int_{0}^{1}{f(x)}\diff x - \int_{0}^{1}{g(x)}\diff x \\
        \leq & \ f(t) - t - \int_{0}^{1}{f(x)}\diff x \\
        = & \int_{0}^{1}{[f(t) - f(x)]}\diff x - t \\
        \leq & \ t -t = 0
    \end{align*}

    即
    $$\int_{0}^{1}{f(g(x))}\diff x \leq \int_{0}^{1}{f(x)}\diff x + \int_{0}^{1}{g(x)}\diff x$$

\end{proof}

\begin{proposition}

    设$f(x),g(x)$在$[a,b]$上连续,且满足
    $$\int_{a}^{x}{f(t)}\diff t \geq \int_{a}^{x}{g(t)}\diff t, \quad x \in [a,b]$$
    $$\int_{a}^{b}{f(t)}\diff t = \int_{a}^{b}{g(t)}\diff t$$
    证明:
    $$\int_{a}^{b}{xf(x)}\diff x \leq \int_{a}^{b}{xg(x)}\diff x$$

\end{proposition}

\begin{proof}

    \begin{align*}
        \int_{a}^{b}{x(f(x) - g(x))}\diff x & = a \int_{a}^{\eta}{(f(x) - g(x))}\diff x + b\int_{\eta}^{b}{(f(x) - g(x))}\diff x \\
        & = a \int_{a}^{\eta}{(f(x) - g(x))}\diff x - b\int_{a}^{\eta}{(f(x) - g(x))}\diff x \\
        & = (a - b) \int_{a}^{\eta}{(f(x) - g(x))}\diff x
    \end{align*}

    其中$\eta \in [a,b]$。由题设知,$\int_{a}^{\eta}{(f(x) - g(x))}\diff x < 0$,则
    $$\int_{a}^{b}{xf(x)}\diff x \leq \int_{a}^{b}{xg(x)}\diff x$$

\end{proof}

\begin{proposition}

    设$f:[0,1]\to \mathrm{R}$是连续函数,且$\int_{0}^{1}{f(x)}\diff x = 0$,证明:
    $$2 \left(\int_{0}^{1}{xf(x)}\diff x \right)^2 \leq \int_{0}^{1}{(1 - x^2)f^2(x)}\diff x$$

\end{proposition}

\begin{proof}

    \begin{align*}
        \left(\int_{0}^{1}{xf(x)}\diff x \right)^2 & = \left( \int_{0}^{1}{(x-1)f(x)}\diff x \right)^2 \\
        & \leq \left( \int_{0}^{1}{\dfrac{(x-1)^2}{1 - x^2}}\diff x \right) \left( \int_{0}^{1}{(1 - x^2)f^2(x)}\diff x \right) \\
        & = (2\ln{2} - 1)  \left( \int_{0}^{1}{(1 - x^2)f^2(x)}\diff x \right) \\
        & \leq \dfrac{1}{2} \left( \int_{0}^{1}{(1 - x^2)f^2(x)}\diff x \right)
    \end{align*}

    所以
    $$2 \left(\int_{0}^{1}{xf(x)}\diff x \right)^2 \leq \int_{0}^{1}{(1 - x^2)f^2(x)}\diff x$$

\end{proof}

\begin{proposition}

    设$f$是$[-1,1]$上的连续函数,证明:
    $$\int_{-1}^{1}{f^2(x)}\diff x \geq \dfrac{1}{2} \left( \int_{-1}^{1}{f(x)}\diff x \right)^2 + \dfrac{3}{2} \left( \int_{-1}^{1}{xf(x)}\diff x \right)^2$$

\end{proposition}

\begin{proof}

    令$M = \dfrac{1}{2} \int_{-1}^{1}{f(x)}\diff x$,$g(x) = f(x) - M$,则有
    $$\int_{-1}^{1}{\lambda g(x)}\diff x = 0, \quad \forall \lambda \in \mathbb{R}$$
    则

    \begin{align*}
        \int_{-1}^{1}{f^2((x))}\diff x & = \int_{-1}^{1}{(g(x) + M)^2}\diff x \\
        & = \int_{-1}^{1}{g^2(x) + M^2}\diff x \\
        & = \int_{-1}^{1}{g^2(x)}\diff x + \dfrac{1}{2} \left( \int_{-1}^{1}{f(x)}\diff x \right)
    \end{align*}

    因为$\int_{-1}^{1}{Mx}\diff x = 0$,所以
    $$\int_{-1}^{1}{g^2(x)}\diff x \geq \dfrac{3}{2} \left( \int_{-1}^{1}{xg(x)}\diff x \right)^2 \geq \dfrac{3}{2} \left( \int_{-1}^{1}{xf(x)}\diff x \right)^2$$
    则有
    $$\int_{-1}^{1}{f^2(x)}\diff x \geq \dfrac{1}{2} \left( \int_{-1}^{1}{f(x)}\diff x \right)^2 + \dfrac{3}{2} \left( \int_{-1}^{1}{xf(x)}\diff x \right)^2$$

\end{proof} 

\begin{proposition}

    设$f:[0,1]\to\mathrm{R}$是有连续导数的可微函数,且$f(1) = 0$,证明:
    $$4\int_{0}^{1}{|f'(x)|^2}\diff x \geq \int_{0}^{1}{|f(x)|^2}\diff x + \left( \int_{0}^{1}{|f(x)|}\diff x \right)$$

\end{proposition}

\begin{proof}
    
    令$c = \int_{0}^{1}{f(x)}\diff x$,不失一般性地,我们假设
    $$\int_{0}^{1}{(f(x) + c)^2}\diff x > 0$$
    否则$f(x) \equiv -c $,即
    $$f(x) \equiv 0$$
    此时不等式显然成立。应用分部积分法有

    \begin{align*}
        & \int_{0}^{1}{(f(x) + c)}\diff x \\
        = \  & [x(f(x) + c)^2] \Big|_{0}^{1} - 2\int_{0}^{1}{x(f(x) + c) - f'(x)}\diff x \\
        = \ & c^2 + 2\int_{0}^{1}{x(f(x) + c) - f'(x)}\diff x
    \end{align*}

    因为$f(1) = 0$,应用\textup{Jensen}不等式得
    $$ \int_{0}^{1}{|f(x) + c|^2}\diff x - c^2 \leq 2\sqrt{\int_{0}^{1}{|f(x) + c|^2}\diff x} \cdot \sqrt{\int_{0}^{1}{x^2 \left(f'(x)\right)^2}\diff x} $$
    $$ \sqrt{\int_{0}^{1}{|f(x) + c|^2}\diff x} - \dfrac{c^2}{\sqrt{\int_{0}^{1}{|f(x) + c|^2}\diff x}} \leq 2\sqrt{\int_{0}^{1}{x^2 \left(f'(x)\right)^2}\diff x} $$
    两边平方得
    $$\int_{0}^{1}{|f(x) + c|^2}\diff x -2c^2 + \dfrac{c^4}{\sqrt{\int_{0}^{1}{|f(x) + c|^2}\diff x}} \leq 4 \int_{0}^{1}{x^2 \left(f'(x)\right)^2}\diff x $$

    \begin{align*}
        \int_{0}^{1}{|f(x) + c|^2}\diff x & = \int_{0}^{1}{f^2(x)}\diff x + 2c \int_{0}^{1}{f(x)}\diff x + c^2 \\
        & = \int_{0}^{1}{f^2(x)}\diff x + 3c^2 \\
        & = \int_{0}^{1}{f^2(x)}\diff x + 3\left( \int_{0}^{1}{f(x)}\diff x \right)^2
    \end{align*}

    因此有
    $$\int_{0}^{1}{|f(x)|^2}\diff x + \left( \int_{0}^{1}{f(x)}\diff x \right)^2 + \dfrac{c^4}{\sqrt{\int_{0}^{1}{|f(x) + c|^2}\diff x}} \leq 4 \int_{0}^{1}{x^2 \left(f'(x)\right)^2}\diff x $$
    由$\int_{0}^{1}{(f(x) + c)^2}\diff x > 0$得
    $$4\int_{0}^{1}{|f'(x)|^2}\diff x \geq \int_{0}^{1}{|f(x)|^2}\diff x + \left( \int_{0}^{1}{|f(x)|}\diff x \right)$$

\end{proof}

\begin{proposition}
    
    设$f$是定义在$[0,1]$上的连续可微函数,证明:
    $$\left|f\left(\dfrac{1}{2}\right)\right| \leq \int_{0}^{1}{|f(x)|}\diff x + \dfrac{1}{2} \int_{0}^{1}{|f'(x)|}\diff x$$

\end{proposition}

\begin{proof}

    设$x_0 \in [0,1]$,且
    $$|f(x_0)| \leq |f(x)|, \quad x \in \left[0,\dfrac{1}{2}\right]$$
    则有
    $$f\left(\dfrac{1}{2}\right) = f(x_0) + \int_{x_0}^{\frac{1}{2}}{f'(x)}\diff x$$
    $$\left|f\left(\dfrac{1}{2}\right)\right|= |f(x_0)| + \int_{x_0}^{\frac{1}{2}}{|f'(x)|}\diff x$$
    由$|f(x_0)| \leq |f(x)|, \ x \in \left[0,\dfrac{1}{2}\right]$得
    $$|f(x_0)| \leq 2 \int_{0}^{\frac{1}{2}}{|f(x)|}\diff x$$
    又有
    $$\left|f\left(\dfrac{1}{2}\right)\right| \leq 2\int_{0}^{\frac{1}{2}}{|f(x)|}\diff x + \int_{0}^{\frac{1}{2}}{|f'(x)|}\diff x$$
    对于$\forall x \in [\dfrac{1}{2}, 1]$,有不等式
    $$\left|f\left(\dfrac{1}{2}\right)\right| \leq 2\int_{\frac{1}{2}}^{1}{|f(x)|}\diff x + \int_{\frac{1}{2}}^{1}{|f'(x)|}\diff x$$
    所以有
    $$\left|f\left(\dfrac{1}{2}\right)\right| \leq \int_{0}^{1}{|f(x)|}\diff x + \dfrac{1}{2} \int_{0}^{1}{|f'(x)|}\diff x$$

\end{proof}

\begin{proposition}

    设$f$是$[0,1]$上的连续实值函数,且$\int_{0}^{1}{f(x)}\diff x = 0$,证明:$\exists c \in (0,1)$
    $$c^2f(c) = \int_{0}^{c}{(x^2 + x)f(x)}\diff x$$

\end{proposition}

\begin{proof}
    
    令
    $$F(x) = \int_{0}^{x}{[t^2 + (1 - x)t]f(t)}\diff t$$
    因为$f$在$[0,1]$上连续,则$F$在$[0,1]$上存在最大值,最小值。
    设
    $$M = \max{f(x)} = f(x_M),\quad m = \min{f(x)} = f(x_m)$$
    由题设$\int_{0}^{1}{f(x)}\diff x = 0$,则不妨设$f$不恒等于$0$,否则不等式显然成立。\\
    所以
    $$M > 0, \quad m < 0$$
    令$a = x_M$,有
    $$F'(a) = Mx_M - M\int_{0}^{x_M}{t}\diff t = Mx_M \left(1 - \dfrac{x_M}{2} \right) > 0$$
    同理可知$\exists b \in (0,1), F'(b) < 0$\\
    由介值定理知$\exists d \in (a,b)$,使$F'(d) = 0$\\
    又由\textup{Rolle}中值定理知,$\exists c \in (0,d)$
    $$F(c) - F(0) = cF'(c)$$
    即
    $$\int_{0}^{c}{[t^2 + (1 - c)t]f(t)}\diff t =  c^2f(c) - c\int_{0}^{c}{tf(t)}\diff t$$
    $$c^2f(c) = \int_{0}^{c}{(x^2 + x)f(x)}\diff x$$

\end{proof}

\begin{proposition}

    设$f$是$[0,1]$上的连续可微函数,$b > 0$,且$f(0) = 0$,证明:
    $$\int_{0}^{b}{\dfrac{f^2(x)}{x^2}}\diff x \leq 4\int_{0}^{b}{\left( f'(x)\right)^2 }\diff x $$

\end{proposition}

\begin{proof}

    应用分部积分法,有
    $$\int_{0}^{b}{\dfrac{f^2(x)}{x^2}}\diff x = - \left( \left.\dfrac{f^(x)}{x^2} \right|_{0}^{b} - \int_{0}^{b}{\dfrac{2f'(x)f(x)}{x}}\diff x \right)$$
    又因为
    $$\lim_{x \to 0}{\dfrac{f^2(x)}{x}} = \lim_{x \to 0}{f'(x)f(x)} = 0$$
    所以

    \begin{align*}
        \int_{0}^{b}{\dfrac{f^2(x)}{x^2}}\diff x &= -\dfrac{f^2(b)}{b} + 2 \int_{0}^{b}{\dfrac{2f'(x)f(x)}{x}}\diff x \\
        & \leq 2\int_{0}^{b}{\dfrac{2f'(x)f(x)}{x}}\diff x \\
        & \leq 2\left( \int_{0}^{b}{\dfrac{f^2(x)}{x^2}}\diff x \cdot \int_{0}^{b}{\left(f'(x)\right)^2}\diff x\right)^{\frac{1}{2}}
    \end{align*}

    因此有
    $$\int_{0}^{b}{\dfrac{f^2(x)}{x^2}}\diff x \leq 4\int_{0}^{b}{\left( f'(x)\right)^2 }\diff x $$

\end{proof}

\begin{proposition}

    设$f:\mathbb{R} \to (0 ,+\infty)$是可微函数,且对于常数$k$,有
    $$|f'(x) - f'(y)| \leq k|x - y|$$
    其中$x,y \in \mathbb{R}$,证明:
    $$\left( f'(x) \right)^2 < 2kf(x)$$

\end{proposition}

\begin{proof}

    由题设知$f'$是连续函数,因此$f'$黎曼可积,对于$d \geq 0$,有

    \begin{align*}
        0 < f(x + d) & = f(x) + \int_{x}^{x+d}{f'(t)}\diff t \\
        & = f(x) + df'(x) + \int_{x}^{x+d}{(f'(t) - f'(x))}\diff t \\
        & \leq f(x) + df'(x) + \int_{x}^{x+d}{k(t - x)}\diff t \\
        & = f(x) + df'(x) + \dfrac{1}{2}kd^2
    \end{align*}

    对于$d < 0$,有

    \begin{align*}
        \int_{x}^{x+d}{f'(t)}\diff t & = - \int_{x+d}^{x}{f'(t)}\diff t\\
        & \leq -  \int_{x+d}^{x}{k(t - x)}\diff t \\
        & = \dfrac{k}{d^2}
    \end{align*}

    特别地,令$d = -\frac{f'(x)}{k}$,有
    $$0 < f(x) - \dfrac{\left(f'(x)\right)^2}{k} + \dfrac{\left(f'(x)\right)^2}{2k} = f(x) - \dfrac{\left(f'(x)\right)^2}{2k}$$
    所以
    $$\left( f'(x) \right)^2 < 2kf(x)$$
    
\end{proof}

\begin{proposition}

    设$f(x)$是$\mathbb{R}$上的连续实值函数,且
    $$\int_{0}^{+\infty}{f^2(x)}\diff x < \infty$$
    证明:函数
    $$g(x) = f(x) - 2\euler^{-x}\int_{0}^{x}{\euler^tf(t)}\diff t$$
    满足
    $$\int_{0}^{+\infty}{g^2(x)}\diff x = \int_{0}^{+\infty}{f^2(x)}\diff x$$

\end{proposition}

\begin{proof}

    由题设得
    $$f(x) - g(x) = 2\euler^{-x}\int_{0}^{x}{\euler^tf(t)}\diff t$$
    则有
    $$(f(x) - g(x))' = -2\euler^{-x}\int_{0}^{x}{\euler^tf(t)}\diff t + 2f(x) = f(x) + g(x)$$
    应用\textup{Schwarz}不等式,有

    \begin{align*}
        \euler^{-w} \left| \int_{0}^{w}{\euler^tf(t)}\diff t \right| & \leq \euler^{-w} \left| \int_{0}^{\frac{w}{2}}{\euler^tf(t)}\diff t \right| + \euler^{-w} \left| \int_{\frac{w}{2}}^{w}{\euler^tf(t)}\diff t \right| \\
        & \leq \euler^{-w} \left( \int_{0}^{\frac{w}{2}}{\euler^{2t}}\diff t \right)^{\frac{1}{2}} \left( \int_{0}^{\frac{w}{2}}{f^2(t)}\diff t \right)^{\frac{1}{2}} \\
        & \quad + \euler^{-w} \left( \int_{\frac{w}{2}}^{w}{\euler^{2t}}\diff t \right)^{\frac{1}{2}} \left( \int_{\frac{w}{2}}^{w}{f^2(t)}\diff t \right)^{\frac{1}{2}} \\
        & \leq \euler^{-\frac{w}{2}} \int_{0}^{+\infty}{f^2(t)}\diff t + \int_{\frac{w}{2}}^{+\infty}{f^2(t)}\diff t
    \end{align*}

    则有
    $$\lim_{w\to\infty}{\euler^{-w}\int_{0}^{w}{\euler^tf(t)}\diff t} = 0$$
    则$\dfrac{1}{2}(f(x) - g(x))^2$在$[0,a]\ (\forall a \in \mathrm{R}^+)$上必定有界连续,则

    \begin{align*}
        \int_{0}^{w}{f^2(x) - g^2(x)}\diff x & = \int_{0}^{w}{\left( \dfrac{(f(x) - g(x))^2}{2}\right)'}\diff x \\
        & = \left. \dfrac{(f(x) - g(x))^2}{2} \right|_{0}^{w} \\ 
        & = 2 \mathbb{e}^{-2w} \left( \int_{0}^{w}{\euler^tf(t)}\diff t \right)^2
    \end{align*}

    由上式可知

    \begin{align*}
        \lim_{w\to\infty}{\int_{0}^{w}{f^2(x) - g^2(x)}\diff x} & = \int_{0}^{+\infty}{f^2(x) - g^2(x)}\diff x \\
        & = \lim_{w\to\infty}{2 \mathbb{e}^{-2w} \left( \int_{0}^{w}{\euler^tf(t)}\diff t \right)^2} \\
        & = 0
    \end{align*}

    即
    $$\int_{0}^{+\infty}{g^2(x)}\diff x = \int_{0}^{+\infty}{f^2(x)}\diff x$$

\end{proof}

\begin{proposition}

    设$f$是周期为$T$的连续函数,且$\int_{0}^{T}{f(x)}\diff x=0$,证明:
    $$\int_{0}^{T}{|f(x)|^2}\diff x \leq \dfrac{T^2}{4\pi^2}\int_{0}^{T}{|f'(x)|^2}\diff x$$

\end{proposition}

\begin{proof}

    由题设知,$f$可以展开为\textup{Fourier}级数,
    $$f(x) = \dfrac{a_0}{2} + \sum_{n=1}^{\infty}{a_n\cos\left(\dfrac{2n\pi x}{T}\right) + b_n\sin\left(\dfrac{2n\pi x}{T}\right)}$$
    易知$a_0 = 0$,又
    $$f'(x) = \sum_{n=1}^{\infty}{\dfrac{2n\pi}{T}b_n\cos\left(\dfrac{2n\pi x}{T}\right) - \dfrac{2n\pi}{T}a_n\sin\left(\dfrac{2n\pi x}{T}\right)}$$
    由\textup{Parseval}恒等式知
    $$\int_{0}^{T}{|f'(x)|^2}\diff x = \sum_{n=1}^{\infty}{\dfrac{4n^2\pi^2}{T^2}(a_n^2 + b_n^2)} \geq \sum_{n=1}^{\infty}{\dfrac{4\pi^2}{T^2}(a_n^2 + b_n^2)}$$
    $$\int_{0}^{T}{|f(x)|^2}\diff x = \sum_{n=1}^{\infty}{(a_n^2 + b_n^2)}$$
    所以 
    $$\int_{0}^{T}{|f(x)|^2}\diff x \leq \dfrac{T^2}{4\pi^2}\int_{0}^{T}{|f'(x)|^2}\diff x$$

\end{proof}

\begin{proposition}

    设$f$在$[0,a](a > 0)$上有可积的导函数,证明:
    $$|f(0)| \leq \dfrac{1}{a}\int_{0}^{a}{|f(x)|}\diff x + \int_{0}^{a}{|f'(x)|}\diff x$$

\end{proposition}

\begin{proof}

    对任意的$x\in [0,a]$,有

    \begin{align*}
        f(x) - f(0) & = \int_{0}^{x}{f'(t)}\diff t \\
        f(0) & = f(x) - \int_{0}^{x}{f'(t)}\diff t 
    \end{align*}

    \begin{align*}
        |f(0)| & \leq |f(x)| + \left|\int_{0}^{x}{f'(t)}\diff t\right| \\
        & \leq |f(x)| + \int_{0}^{x}{|f'(t)|}\diff t \\
        & \leq |f(x)| + \int_{0}^{a}{|f'(t)|}\diff t 
    \end{align*}

    两边对$x$从$0$到$a$积分,得
    $$a|f(0)| \leq \int_{0}^{a}{|f(x)|}\diff x + a\int_{0}^{a}{|f'(x)|}\diff x$$
    即
    $$|f(0)| \leq \dfrac{1}{a}\int_{0}^{a}{|f(x)|}\diff x + \int_{0}^{a}{|f'(x)|}\diff x$$

\end{proof}

\begin{proposition}

    设$f$在$[0,1]$上有可积的导函数,证明:
    $$\int_{0}^{1}{|f(x)|}\diff x \leq \max\left\{\int_{0}^{1}{|f'(x)|}\diff x, \left|\int_{0}^{1}{f(x)}\diff x\right|\right\}$$

\end{proposition}

\begin{proof}
    
    若
    $$\int_{0}^{1}{|f'(x)|}\diff x = \left|\int_{0}^{1}{f(x)}\diff x\right|$$
    则不等式显然成立
    若
    $$\left|\int_{0}^{1}{f(x)}\diff x\right| < \int_{0}^{1}{|f(x)|}\diff x$$
    则$f(x)$在$[0,1]$上一定变号,所以$\exists x_0 \in [0,1]$,$f(x_0) = 0$,则

    \begin{align*}
        |f(x)| & = |f(x) - f(x_0)| \\
        & = \left|\int_{x_0}^{x}{f'(t)}\diff t\right| \\
        & \leq \int_{x_0}^{x}{|f'(t)|}\diff t \\
        & \leq \int_{0}^{1}{|f'(x)|}\diff x
    \end{align*}

    则
    $$\int_{0}^{1}{|f(x)|}\diff x \leq \int_{0}^{1}{|f'(x)|}\diff x$$

\end{proof}

\begin{proposition}
    
    函数$f(x)$,$g(x)$是$[0,1]$上的可导函数,且
    $$\int_{0}^{1}{f(x)}\diff x = 3 \int_{\frac{2}{3}}^{1}{f(x)}\diff x$$
    证明:在$[0,1]$上存在不同的两点$\xi$,$\eta$,使得
    $$f'(\xi) = g'(\xi)(f(\eta) - f(\xi))$$

\end{proposition}

\begin{proof}

    反证法。设$\forall x,y \in [0,1]$,$x \neq y$,有
    $$f'(x) \neq g'(x)(f(y) - f(x))$$
    不妨设
    $$f'(x) > g'(x)(f(y) - f(x))$$
    否则由\textup{Darboux}定理知,必存在不同的两点$x,\ y$使得
    $$f'(x) = g'(x)(f(y) - f(x))$$
    令$y \to x$,有
    $$\lim_{y \to x}{f'(x)} = f'(x) \geq \lim_{y \to x}{g'(x)(f(y) - f(x))} = 0$$
    由$x$的任意性知,$f'(x) \geq 0$,$\forall x \in [0,1]$,
    则由题设
    $$\int_{0}^{\frac{2}{3}}{f(x)}\diff x = 2 \int_{\frac{2}{3}}^{1}{f(x)}\diff x$$
    然而由于$f'(x) \geq 0$
    $$2 \int_{\frac{2}{3}}^{1}{f(x)}\diff x \geq \int_{0}^{\frac{2}{3}}{f(x)}\diff x $$
    与假设矛盾。得证。

\end{proof}

\begin{proposition}
    
    设$f'(x)$在$[a,\ b]$上连续,证明:
    $$\max_{x\in[a,b]}{|f(x)|} \leq \left|\dfrac{1}{b - a}\int_{a}^{b}{f(x)}\diff x\right| + \int_{a}^{b}{|f'(x)|}\diff x$$

\end{proposition}

\begin{proof}

    设$x$,$x_0$是$[a,\ b]$上两点,则有
    $$f(x) - f(x_0) = \int_{x_0}^{x}{f'(t)}\diff t$$
    所以

    \begin{align*}
        |f(x_0)| & = \left|f(x) - \int_{x_0}^{x}{f'(t)}\diff t \right| \\
        & \leq |f(x)| + \left|\int_{x_0}^{x}{f'(t)}\diff t \right| \\
        & \leq |f(x)| +  \int_{x_0}^{x}{|f'(t)|}\diff t \\
        & \leq |f(x)| +  \int_{a}^{b}{|f'(t)|}\diff t \\
    \end{align*}

    两边对$x$从$a$到$b$积分得
    $$(b - a)|f(x_0)| \leq \int_{a}^{b}{|f(x)|}\diff x + (b - a)\int_{a}^{b}{|f'(t)|}\diff t$$
    即
    $$|f(x_0)| \leq \left|\dfrac{1}{b - a}\int_{a}^{b}{f(x)}\diff x\right| + \int_{a}^{b}{|f'(x)|}\diff x$$
    由$x_0$的任意性知
    $$\max_{x\in[a,b]}{|f(x)|} \leq \left|\dfrac{1}{b - a}\int_{a}^{b}{f(x)}\diff x\right| + \int_{a}^{b}{|f'(x)|}\diff x$$
    
\end{proof}

\begin{theorem}

    证明:
    $$\oooint_{\Omega}{f(ax + by + cz)}\diff \Omega = \pi \int_{-1}^{1}{(1 - u^2)f(\delta u)}\diff u$$
    其中,积分区域$\Omega:\ x^2 + y^2 + z^2 \leq 1$,$\delta = \sqrt{a^2 + b^2 + c^2}$。

\end{theorem}

\begin{proof}

    平面$P_u$
    $$ax + by + cz = \delta u$$
    与原点的距离为$u$,且$-1 \leq u \leq 1$。\\
    平面$P_{u + \diff u}$
    $$ax + by + cz = u + \diff u$$
    与平面$P_u$之间所夹的体元为
    
    \begin{align*}
        \int_{u}^{u + \diff u}{\pi (\sqrt{1 - u^2})^2}\diff u & = \pi \int_{u}^{u + \diff u}{(1 - u^2)}\diff u \\
        & = \pi (u - \dfrac{1}{3}u^3) \Big|_{u}^{u + \diff u} \\
        & = \pi [\diff u - \dfrac{1}{3}((u + \diff u)^3 - u^3)] \\
        & = \pi (1 - u^2)\diff u
    \end{align*}

    所以原积分可化为
    $$\pi \int_{-1}^{1}{(1 - u^2)f(\delta u)}\diff u$$

\end{proof}

\begin{theorem}\label{theorem1}

    证明:
    $$\ooint_{S}{f(ax + by + cz)}\diff S = 2\pi\int_{-1}^{1}{f(\delta u)}\diff u$$
    其中,积分曲面$S:\ x^2 + y^2 + z^2 = 1$,$\delta = \sqrt{a^2 + b^2 + c^2}$。

\end{theorem}

\begin{proof}

    存在一个正交矩阵$A$,将$(\begin{smallmatrix} x,& y,& z \end{smallmatrix})$转换为$(\begin{smallmatrix} u,& v,& w \end{smallmatrix})$,即
    $$ A 
    \begin{pmatrix}
        x\\
        y\\
        z
    \end{pmatrix} = 
    \begin{pmatrix}
        u\\
        v\\
        w
    \end{pmatrix}$$
    且$A$的第一行为
    $$\begin{pmatrix}
        \dfrac{a}{\delta} & \dfrac{b}{\delta} & \dfrac{c}{\delta}
    \end{pmatrix}$$
    又因为矩阵$A$为正交矩阵,则
    $$|A| = 1$$
    所以原积分可应用坐标变换得
    
    \begin{align*}
        \ooint_{S}{f(ax + by + cz)}\diff S & = \ooint_{S}{f(\delta u)}\diff S \\
        & = 2\iint_{D}{f(\delta u) \sqrt{1 + \left( \dfrac{\partial w}{\partial u} \right)^2 + \left( \dfrac{\partial w}{\partial v}\right)^2}}\diff u \diff v\\
        & = 2\iint_{D}{f(\delta u) \dfrac{1}{\sqrt{1 - u^2 - v^2}}}\diff u \diff v \\
        & = 2\int_{-1}^{1}{f(\delta u)}\diff u \int_{-\sqrt{1 - u^2}}^{\sqrt{1 - u^2}}{\dfrac{1}{\sqrt{1 - u^2 - v^2}}}\diff v \\
        & = 2\int_{-1}^{1}{f(\delta u)}\diff u \cdot 2 \int_{0}^{\sqrt{1 - u^2}}{\dfrac{1}{\sqrt{1 - u^2 - v^2}}}\diff v \\
        & = 2\int_{-1}^{1}{f(\delta u)}\diff u \cdot 2\arcsin\left.\left(\dfrac{v}{\sqrt{1 - u^2}}\right) \right|_{0}^{\sqrt{1 - u^2}} \\
        & = 2\pi\int_{-1}^{1}{f(\delta u)}\diff u
    \end{align*}

\end{proof}

\begin{proposition}

    $$\int_{0}^{2\pi}\diff \varphi \int_{0}^{\pi}{\euler^{\sin\theta(\cos\varphi - \sin\varphi)}\sin\theta}\diff \theta$$

\end{proposition}

\begin{proof}

    设曲面$S:\ x^2 + y^2 + z^2 = 1$,应用球面坐标系,易知

    \begin{align*}
        &x = \sin\theta \cos\varphi \\
        &y = \sin\theta \sin\varphi \\
        &z = \cos\theta         
    \end{align*}

    其中,$0 \leq \theta \leq \pi$,$0 \leq \varphi \leq 2\pi$。\\
    应用定理\ref{theorem1},则有

    \begin{align*}
        \int_{0}^{2\pi}\diff \varphi \int_{0}^{\pi}{\euler^{\sin\theta(\cos\varphi - \sin\varphi)}\sin\theta}\diff \theta & =\ooint_{S}{\euler^{x - y}}\diff S \\
        & = 2\pi \int_{-1}^{1}{f(\sqrt{2}u)}\diff u \\
        & = \sqrt{2}\pi(\euler^{\sqrt{2}} - \euler^{-\sqrt{2}})
    \end{align*}

\end{proof}

\begin{proposition}

    设$f(x)$,$g(x)$是$[0,1]$上的恒正连续函数,且$f(x)$,$\dfrac{f(x)}{g(x)}$单调增加。\\
    证明:
    $$\int_{0}^{1}{\dfrac{\int_{0}^{x}{f(t)}\diff t}{\int_{0}^{x}{g(t)}\diff t}}\diff x \leq 2\int_{0}^{1}{\dfrac{f(x)}{g(x)}}\diff x$$

\end{proposition}

\begin{proof}
    
    由$f(x)$,$\dfrac{f(x)}{g(x)}$单调增加知,$\forall m,\ n \in \mathbb{R}$,有
    $$(f(m) - f(n))\left(\dfrac{f(m)}{g(m)} - \dfrac{f(n)}{g(n)}\right) \geq 0$$
    对上式关于$m,\ n$从$0$到$x$积分,得
    $$x\int_{0}^{x}{g(t)}\diff t \geq \int_{0}^{x}{f(t)}\diff t \int_{0}^{x}{\dfrac{g(t)}{f(t)}}\diff t$$
    其中,$x > 0$,则有
    $$\dfrac{x}{\int_{0}^{x}{\dfrac{g(t)}{f(t)}}\diff t} \geq \dfrac{\int_{0}^{x}{f(t)}\diff t}{\int_{0}^{x}{g(t)}\diff t}$$
    由\textup{H{\H o}lder}不等式得
    $$\int_{0}^{x}{\dfrac{g(t)}{f(t)}}\diff t \int_{0}^{x}{\dfrac{f(t)t^2}{g(t)}}\diff t \geq \dfrac{x^4}{4}$$
    所以
    $$\dfrac{\int_{0}^{x}{f(t)}\diff t}{\int_{0}^{x}{g(t)}\diff t} \leq 4 \dfrac{\int_{0}^{x}{\dfrac{f(t)t^2}{g(t)}}\diff t}{x^3}$$
    因此
    $$\int_{0}^{1}{\dfrac{\int_{0}^{x}{f(t)}\diff t}{\int_{0}^{x}{g(t)}\diff t}}\diff x \leq \int_{0}^{1}{4 \dfrac{\int_{0}^{x}{\dfrac{f(t)t^2}{g(t)}}\diff t}{x^3}}\diff x$$
    其中
    $$\int_{0}^{1}{4 \dfrac{\int_{0}^{x}{\dfrac{f(t)t^2}{g(t)}}\diff t}{x^3}}\diff x = \int_{0}^{1}{\int_{t}^{1}{\dfrac{f(t)t^2}{g(t)}\dfrac{4}{x^3}}\diff x}\diff t = 2 \int_{0}^{1}{\dfrac{f(t)}{g(t)}(1 - t^2)}\diff t$$
    则有
    
    \begin{align*}
        \int_{0}^{1}{\dfrac{\int_{0}^{x}{f(t)}\diff t}{\int_{0}^{x}{g(t)}\diff t}}\diff x & \leq 2 \int_{0}^{1}{\dfrac{f(t)}{g(t)}(1 - t^2)}\diff t \\
        & \leq 2 \int_{0}^{1}{\dfrac{f(t)}{g(t)}}\diff t \\
        & = 2 \int_{0}^{1}{\dfrac{f(x)}{g(x)}}\diff x
    \end{align*}

    即
    $$\int_{0}^{1}{\dfrac{\int_{0}^{x}{f(t)}\diff t}{\int_{0}^{x}{g(t)}\diff t}}\diff x \leq 2\int_{0}^{1}{\dfrac{f(x)}{g(x)}}\diff x$$

\end{proof}

\begin{proposition}

    证明:
    $$\sum\limits_{k = 0}^{n}{(-1)^k \binom{n}{k}\dfrac{1}{k + m + 1}} = \sum\limits_{k = 0}^{m}{(-1)^k \binom{m}{k}\dfrac{1}{k + n + 1}}$$

\end{proposition}

\begin{proof}

    易知
    $$\dfrac{1}{k + m + 1} = \int_{0}^{1}{x^{k + m}}\diff x$$
    $$\dfrac{1}{k + n + 1} = \int_{0}^{1}{x^{k + n}}\diff x$$
    原等式左边有
    
    \begin{align*}
        \sum\limits_{k = 0}^{n}{(-1)^k \binom{n}{k}\dfrac{1}{k + m + 1}} & =  \sum\limits_{k = 0}^{n}{\left[(-1)^k \binom{n}{k} \int_{0}^{1}{x^{k + n}}\diff x\right]} \\
        & = \int_{0}^{1}{\left[x^m \sum\limits_{k = 0}^{n}{\left[(-1)^k\binom{n}{k} x^k\right]}\right]}\diff x \\
        & = \int_{0}^{1}{x^m (1 - x)^n}\diff x        
    \end{align*}

    同理,原等式右边有
    $$\sum\limits_{k = 0}^{m}{(-1)^k \binom{m}{k}\dfrac{1}{k + n + 1}} = \int_{0}^{1}{x^n (1 - x)^m}\diff x$$
    易证
    $$\int_{0}^{1}{x^m (1 - x)^n}\diff x = \int_{0}^{1}{x^n (1 - x)^m}\diff x$$
    即
    $$\sum\limits_{k = 0}^{n}{(-1)^k \binom{n}{k}\dfrac{1}{k + m + 1}} = \sum\limits_{k = 0}^{m}{(-1)^k \binom{m}{k}\dfrac{1}{k + n + 1}}$$

\end{proof}

\begin{theorem}[Frullani定理]

    设函数$f(x)$是$[0,\ \alpha]$上的连续函数,且对任意的$\beta$,$\int_{\beta}^{+\infty}{\dfrac{f(x)}{x}}\diff x$存在。
    则
    $$\int_{0}^{+\infty}{\dfrac{f(ax) - f(bx)}{x}}\diff x = f(0)\log{\dfrac{b}{a}}$$

\end{theorem}

\begin{proof}

    对任意的$\beta$
    
    \begin{align*}
        \int_{\beta}^{+\infty}{\dfrac{f(ax)}{{x}}}\diff x & = \lim_{T\to +\infty}{\int_{\beta}^{T}{\dfrac{f(ax)}{{x}}}\diff x} \\
        & = \lim_{T\to +\infty}{\int_{a\beta}^{aT}{\dfrac{f(x)}{x}}\diff x} \\
        & = \int_{a\beta}^{+\infty}{\dfrac{f(x)}{x}}\diff x
    \end{align*}

    同理
    $$\int_{\beta}^{+\infty}{\dfrac{f(bx)}{{x}}}\diff x = \int_{b\beta}^{+\infty}{\dfrac{f(x)}{x}}\diff x$$
    则
    $$\int_{0}^{+\infty}{\dfrac{f(ax) - f(bx)}{x}}\diff x = \int_{a\beta}^{b\beta}{\dfrac{f(x)}{x}}\diff x = \int_{a}^{b}{\dfrac{f(\beta x)}{x}}\diff x$$
    又
    $$\lim_{\beta \to 0}{\int_{a}^{b}{\dfrac{f(\beta x)}{x}}\diff x} = f(0) \int_{a}^{b}{\dfrac{1}{x}}\diff x = f(0)\log{\dfrac{b}{a}}$$
    
\end{proof}

\begin{proposition}

    计算
    $$\lim\limits_{n\to\infty}{\dfrac{1}{n^2}\int_{0}^{\frac{\pi}{2}}{x\left(\dfrac{\sin{nx}}{\sin{x}}\right)}}\diff x$$

\end{proposition}

\begin{proof}

    由
    $$\lim\limits_{x \to 0}{x^2\left(\dfrac{1}{\sin^4{x}} - \dfrac{1}{x^4}\right)} = \dfrac{2}{3}$$
    知存在常数$C > 0$,使得
    $$\left| \dfrac{x\sin^4{nx}}{\sin^4{x} - \dfrac{\sin^4{nx}}{x^3}}\right| \leq C\dfrac{\sin^4{nx}}{x} \leq Cn, \quad \forall x \in \left(0,\ \dfrac{\pi}{2}\right)$$
    因此

    \begin{align*}
        \lim\limits_{n\to\infty}{\dfrac{1}{n^2}\int_{0}^{\frac{\pi}{2}}{x\left(\dfrac{\sin{nx}}{\sin{x}}\right)}}\diff x & = \lim\limits_{n\to\infty}{\dfrac{1}{n^2}\int_{0}^{\frac{\pi}{2}}{\dfrac{\sin^4{nx}}{x^3}\diff x}} \\
        & = \lim\limits_{n\to\infty}{\int_{0}^{\frac{n\pi}{2}}{\dfrac{\sin^4{x}}{x^3}}\diff x} \\
        & = \int_{0}^{+\infty}{\dfrac{\sin^4{x}}{x^3}}\diff x \\
        & = \int_{0}^{+\infty}{\dfrac{2\sin^3{x}\cos{x}}{x^2}}\diff x \\
        & = \int_{0}^{+\infty}{\dfrac{6\sin^2{x}\cos^2{x} - 2\sin^4{x}}{x}}\diff x \\
        & = \int_{0}^{+\infty}{\dfrac{\cos{2x} - \cos{4x}}{x}}\diff x
    \end{align*}

    由\textup{Frullani}公式知
    $$\int_{0}^{+\infty}{\dfrac{\cos{2x} - \cos{4x}}{x}}\diff x = \ln{2}$$
    即
    $$\lim\limits_{n\to\infty}{\dfrac{1}{n^2}\int_{0}^{\frac{\pi}{2}}{x\left(\dfrac{\sin{nx}}{\sin{x}}\right)}}\diff x = \ln{2}$$

\end{proof}


\begin{proposition}

    设函数$f \in C^{1}[0,1]$,证明:
    $$\lim\limits_{n\to\infty}{n \left[\int_{0}^{1}{f(x)}\diff x - \dfrac{1}{n}\sum\limits_{k = 1}^{n}{f\left(\dfrac{k}{n}\right)}\right]} = -\dfrac{1}{2}(f(1) - f(0))$$

\end{proposition}

\begin{proof}
    
    \begin{align*}
        \int_{0}^{1}{f(x)}\diff x - \dfrac{1}{n}\sum\limits_{k = 1}^{n}{f\left(\dfrac{k}{n}\right)} & = \sum\limits_{k = 1}^{n}{\int_{\frac{k - 1}{n}}^{\frac{k - 1}{n}}{f(x) - f\left(\dfrac{k}{n}\right)}\diff x} \\
        & = \sum\limits_{k = 1}^{n}{\int_{\frac{k - 1}{n}}^{\frac{k - 1}{n}}{f'(\xi_k)\left(x - \dfrac{k}{n}\right)}\diff x} \\
    \end{align*}
    
    因为$f \in C^{1}[0,1]$,所以在$\left[\dfrac{k - 1}{n}, \dfrac{k}{n}\right],\ k = 1,\ 2,\  \cdots,\ n$中,存在$m_k,\ M_k$,使得
    $$m_k \leq f'(x) \leq M_k$$
    因此有

    \begin{align*}
        \int_{0}^{1}{f(x)}\diff x - \sum\limits_{k = 1}^{n}{f\left(\dfrac{k}{n}\right)} & \leq \sum\limits_{k = 1}^{n}{\int_{\frac{k - 1}{n}}^{\frac{k - 1}{n}}{m_k\left(x - \dfrac{k}{n}\right)}\diff x} \\
        & \leq \sum\limits_{k = 1}^{n}{\int_{\frac{k - 1}{n}}^{\frac{k - 1}{n}}{m_k\left(x - \dfrac{k}{n}\right)}\diff x} \\
        & \leq \sum\limits_{k = 1}^{n}{-m_k \cdot \dfrac{1}{2n^2}} \\
        & = - \dfrac{1}{2n^2} \sum\limits_{k = 1}^{n}{m_k}
    \end{align*}

    同理
    $$\int_{0}^{1}{f(x)}\diff x - \sum\limits_{k = 1}^{n}{f\left(\dfrac{k}{n}\right)} \geq - \dfrac{1}{2n^2} \sum\limits_{k = 1}^{n}{M_k}$$
    两边对$n$取极限,由黎曼积分性质可知
    $$\lim\limits_{n\to\infty}{n \left[\int_{0}^{1}{f(x)}\diff x - \sum\limits_{k = 1}^{n}{f\left(\dfrac{k}{n}\right)}\right]} = -\dfrac{1}{2}(f(1) - f(0))$$
    
\end{proof}

\begin{proposition}
    
    设函数$f \in C^{1}[0,1]$,$\theta \in [0, 1]$,证明:
    $$\lim\limits_{n\to\infty}{n \left[\int_{0}^{1}{f(x)}\diff x - \dfrac{1}{n}\sum\limits_{k = 0}^{n - 1}{f\left(\dfrac{k + \theta}{n}\right)}\right]} = -\left(\theta -\dfrac{1}{2}\right) (f(1) - f(0))$$
\end{proposition}

\begin{proof}
    
    \begin{align*}
        \int_{0}^{1}{f(x)}\diff x - \dfrac{1}{n}\sum\limits_{k = 0}^{n - 1}{f\left(\dfrac{k + \theta}{n}\right)} & = \int_{0}^{1}{f(x)}\diff x - \sum\limits_{k = 0}^{n - 1}{\left[f\left(\dfrac{k + \theta}{n}\right) - f\left(\dfrac{k + 1}{n}\right) + f\left(\dfrac{k + 1}{n}\right)\right]} \\
        & = \int_{0}^{1}{f(x)}\diff x - \dfrac{1}{n}\sum\limits_{k = 0}^{n - 1}{f\left(\dfrac{k + 1}{n}\right)} \\
        & \quad + \dfrac{1}{n}\sum\limits_{k = 0}^{n - 1}{\left[f\left(\dfrac{k + 1}{n}\right) - f\left(\dfrac{k + \theta}{n}\right)\right]}
    \end{align*}

    $$\sum\limits_{k = 0}^{n - 1}{\left[f\left(\dfrac{k + 1}{n}\right) - f\left(\dfrac{k + \theta}{n}\right)\right]} = -\dfrac{1}{n}(\theta - 1)\sum\limits_{k = 0}^{n - 1}{f'(\xi_k)}$$
    其中,$\xi_k \in \left[\dfrac{k}{n}, \dfrac{k + 1}{n}\right],\ k = 0,\ 1,\  \cdots,\ n$
    对$n$取极限,由黎曼积分性质可知
    $$\lim\limits_{n\to\infty}{\sum\limits_{k = 0}^{n - 1}{\left[f\left(\dfrac{k + 1}{n}\right) - f\left(\dfrac{k + \theta}{n}\right)\right]}} = -(\theta - 1)(f(1) - f(0))$$
    综上
    $$\lim\limits_{n\to\infty}{n \left[\int_{0}^{1}{f(x)}\diff x - \dfrac{1}{n}\sum\limits_{k = 0}^{n - 1}{f\left(\dfrac{k + \theta}{n}\right)}\right]} = -\left(\theta -\dfrac{1}{2}\right) (f(1) - f(0))$$

\end{proof}

\begin{proposition}

    设函数$f \in C[0,1]$,如果极限
    $$\lim\limits_{n\to\infty}{\dfrac{f(0) + f(\frac{1}{n}) + f(\frac{2}{n}) + \cdots + f(1)}{n}} = M$$
    其中$M$为$f(x)$在$[0,1]$上的最大值。证明:$f(x) \equiv M$。

\end{proposition}

\begin{proof}

    反证法。设$\exists x_0 \in [0,1]$,$f(x_0) = y < M$,则由函数连续性可知,$\exists \delta > 0$,$|x - x_0| < \delta$时
    $$|f(x) - f(x_0)| < \varepsilon$$
    其中$\varepsilon = \dfrac{M -y}{2}$,又由黎曼积分性质知,
    $$\lim\limits_{n\to\infty}{\dfrac{f(0) + f(\frac{1}{n}) + f(\frac{2}{n}) + \cdots + f(1)}{n}} = \int_{0}^{1}{f(x)}\diff x = M$$
    然而
    $$\int_{x_0 - \delta}^{x_0 + \delta}{f(x)}\diff x < 2 \delta \dfrac{M + y}{2} = \delta (M + y) < 2M \delta$$
    所以
    $$\int_{0}^{1}{f(x)}\diff x < M$$
    与题设矛盾。得证。

\end{proof}

\begin{proposition}

    设非负函数$f(x) \in C[0,1]$,且在$[0,1]$上是单调递增的,记
    $$s = \dfrac{\int_{0}^{1}{xf(x)}\diff x}{\int_{0}^{1}{f(x)}\diff x}$$

    \begin{enumerate}

        \item 证明:$s \geq \dfrac{1}{2}$;
        \item 试比较$\int_{0}^{s}{f(x)}\diff x$与$\int_{s}^{1}{f(x)}\diff x$的大小。
        
    \end{enumerate}

\end{proposition}

\begin{proof}

    \begin{enumerate}

        \item 
            由题设,即证
            $$\int_{0}^{1}{xf(x)}\diff x \geq \dfrac{1}{2} \int_{0}^{1}{f(x)}\diff x$$
            
            \begin{align*}
                \int_{0}^{1}{xf(x)}\diff x - \dfrac{1}{2} \int_{0}^{1}{f(x)}\diff x & = \int_{0}^{1}{(x - \dfrac{1}{2})f(x)}\diff x \\
                & = \int_{0}^{\frac{1}{2}}{(x - \dfrac{1}{2})f(x)}\diff x + \int_{\frac{1}{2}}^{1}{(x - \dfrac{1}{2})f(x)}\diff x \\
            \end{align*}

            由积分中值定理可得,
            $$\int_{0}^{\frac{1}{2}}{(x - \dfrac{1}{2})f(x)}\diff x = -\dfrac{1}{8} f(\xi_1)$$
            $$\int_{\frac{1}{2}}^{1}{(x - \dfrac{1}{2})f(x)}\diff x = \dfrac{1}{8} f(\xi_2)$$
            其中,$0 < \xi_1 < \dfrac{1}{2} < \xi_2 < 1$,再由$f(x)$在$[0,1]$上单调递增,可知
            $$ f(\xi_2) - f(\xi_1) \geq 0$$
            即
            $$s \geq \dfrac{1}{2}$$
        \item 
            因为$f(x) \in C[0,1]$,则$f'(x) \geq 0$,考虑变上限积分
            $$F(x) = \int_{0}^{x}{f(t)} \diff t$$
            显然$F''(x) = f'(x) \geq 0$,因此$F(x)$的图像是下凸的且$F(x) \geq 0$.不妨设
            $$\int_{0}^{1}{f(x)}\diff x = 1$$
            则$F(1) = 1$,$F(0) = 0$.由题设知,即证
            $$F(s) = \int_{0}^{s}{f(x)}\diff x \leq \int_{s}^{1}{f(x)}\diff x = F(1) - F(s)$$
            也即$F(s) \leq \dfrac{1}{2}$.\\
            由几何直观,$x = s$处的切线(斜率$k > 0$)
            $$y = k(x - s) + F(s)$$
            在函数$y = F(x)$的下方,即
            $$F(x) \geq k(x - s) + F(s)$$
            利用$f(x)$的非负性得
            $$F(x)f(x) \geq k(x - s)f(x) + F(s)f(x)$$
            不等式两边从$0$到$1$积分得
            $$\text{左边} = \int_{0}^{1}{F(x)f(x)}\diff x = \int_{0}^{1}{F(x)} \diff F(x) = \left. \dfrac{1}{2}F^2(x)\right|_{0}^{1} = \dfrac{1}{2}$$
            $$\text{右边} = \int_{0}^{1}{k(x - sf(x))}\diff x  + \int_{0}^{1}{F(x)f(x)}\diff x = (ks - ks) + F(s) = F(s)$$
            所以$F(s) \leq \dfrac{1}{2}$,即
            $$\int_{0}^{s}{f(x)}\diff x \leq \int_{s}^{1}{f(x)}\diff x$$

    \end{enumerate}

\end{proof}


\begin{proposition}

    设$f(x) \in C^2[0,1]$,且满足$f(0) = f(1) = f'(0) = 0$,$f'(1) = 1$\\
    证明:$\int_{0}^{1}{(f''(x))^2}\diff x \geq 4$并指出不等式中等号成立的条件。

\end{proposition}

\begin{proof}

    显然
    $$\int_{0}^{1}{(f''(x) + ax + b)^2}\diff x \geq 0$$
    则有

    \begin{align*}
        & \int_{0}^{1}{(f''(x) + ax + b)^2}\diff x \\
        = & \int_{0}^{1}{(f''(x))^2}\diff x + 2a \int_{0}^{1}{xf''(x)}\diff x + 2b \int_{0}^{1}{f''(x)}\diff x + \int_{0}^{1}{(ax + b)^2}\diff x \\
        = & \int_{0}^{1}{(f''(x))^2}\diff x + 2a \int_{0}^{1}{x}\diff f'(x) + 2b f'(x) \Big|_{0}^{1} + \dfrac{a^2}{3} + ab + b^2 \\
        = & \int_{0}^{1}{(f''(x))^2}\diff x + 2a xf'(x)\Big|_{0}^{1} - 2a \int_{0}^{1}{f'(x)}\diff x + 2b + \dfrac{a^2}{3} + ab + b^2 \\
        = & \int_{0}^{1}{(f''(x))^2}\diff x + 2a - 2af(x)\Big|_{0}^{1} + 2b + \dfrac{a^2}{3} + ab + b^2 \\
        = & \int_{0}^{1}{(f''(x))^2}\diff x + 2a + 2b + \dfrac{a^2}{3} + ab + b^2
    \end{align*}

    所以
    $$\int_{0}^{1}{(f''(x))^2}\diff x \geq \max_{(a,b) \in \mathbb{R}^2}\left(-2a - 2b - \dfrac{a^2}{3} - ab - b^2 \right) = 4$$
    右端最小值在$a = -6$,$b = 2$时取得,因此,当$f''(x) = 6x - 2$且满足题设时不等式取等,不难得到此时$f(x) = x^3 - x^2$。
    
\end{proof}
