\section{连续函数}

\begin{proposition}

    设$f(x) = \sum\limits_{k=1}^{n}{c_k\euler^{\lambda_k x}}$
    其中$\lambda_1,\lambda_2,\cdots,\lambda_n$是互异实数,$c_1, c_2, \cdots , c_n$不同时为$0$.\\
    证明:$f(x)$的零点个数小于$n$。

\end{proposition}

\begin{proof}

    应用数学归纳法。$n=1$时,$f(x) = c_1 \euler^{\lambda_1 x}$无零点,结论成立。\\
    假设$n = m$时成立,则$n = m + 1$时,应用反证法,假设此时零点个数大于等于$m+1$
    $$f(x) = \sum_{k=1}^{m+1}{c_k\euler^{\lambda_k x}} = \left[c_{m+1} + \sum_{k=1}^{m}{c_k\euler^{(\lambda_k - \lambda_{m+1}) x}}\right]$$
    令
    $$g(x) = \sum_{k=1}^{m}{c_k\euler^{(\lambda_k - \lambda_{m+1}) x}} + c_{m+1}$$
    则$g(x)$有至少$m+1$个零点,由\textup{Rolle}中值定理得,$g'(x) = \sum_{k=1}^{m}{c_k (\lambda_k - \lambda_{m+1}) \euler^{(\lambda_k - \lambda_{m+1}) x}}$至少有$m$个零点\\
    因为$a_k = c_k(\lambda_k - \lambda_{m+1})$互异,且$\lambda_k - \lambda_{m+1}$不同时为$0$\\
    所以$g'(x)$零点个数小于$m$个,与假设矛盾。则$n = m+1$时结论成立。\\
    由数学归纳法,结论得证。

\end{proof}

\begin{proposition}

    设$\lim\limits_{n\to\infty}{f(x)} = 0$,且$f(x) - f\left(\dfrac{x}{2}\right) = O(x) \ (x\to 0)$\\
    证明:
    $$f(x) = O(x) \quad (x \to 0)$$

\end{proposition}

\begin{proof}

    因为$f(x) - f\left( \dfrac{x}{2} \right) = O(x) \ (x\to 0)$,所以
    $$\lim_{x\to\infty}{\left| \dfrac{f(x) - f\left(\dfrac{x}{2}\right)}{x} \right|}$$
    $\forall \varepsilon > 0$,$\exists \delta > 0$,$|x| < \delta$
    $$\left| \dfrac{f(x) - f\left(\dfrac{x}{2}\right)}{x} \right| < \dfrac{\varepsilon}{2}$$
    即
    $$f(x) - f\left(\dfrac{x}{2}\right) = x\alpha(x),\quad |x| < \delta$$
    其中$|\alpha(x)| < \dfrac{\varepsilon}{2},\ |x| < \delta$,则
    $$f(x) - f\left(\dfrac{x}{2^n}\right) = \sum_{k=1}^{n}{\left( f\left(\dfrac{x}{2^{k-1}}\right) - f\left(\dfrac{x}{2^k}\right) \right)} = \sum_{k=1}^{n}{\left[ \dfrac{x}{2^{k-1}}\alpha\left(\dfrac{x}{2^k}\right) \right]}$$
    因为$k \in \mathbb{N}$时,
    $$\left| \dfrac{x}{2^{k-1}} \right| \leq |x|$$
    所以

    \begin{align*}
        \left|f(x) - f\left(\dfrac{x}{2^k}\right)\right| & \leq \sum_{k=1}^{n}{\left[\dfrac{x}{2^{k-1}}\left|\alpha\left(\dfrac{x}{2^{k-1}}\right)\right| |x| \right]} \\
        & \leq |x| \sum_{k=1}^{n}{\left(\dfrac{1}{2^k}\right)} \varepsilon \\
        & = |x| \left[1 - \left(\dfrac{1}{2}\right)^n \right] \varepsilon 
    \end{align*}

    所以
    $$\left| \dfrac{f(x) - f\left(\dfrac{x}{2^n}\right)}{x} \right| \leq \varepsilon \left[1 - \left(\dfrac{1}{2}\right)^n \right] < \varepsilon$$
    又
    $$\lim_{n\to\infty}{f\left(\dfrac{x}{2^n}\right)} = \lim_{n\to\infty}{f(x)} = 0$$
    则
    $$\lim_{n\to\infty}{\left| \dfrac{f(x) - f\left(\dfrac{x}{2^n}\right)}{x} \right|} = \left| \dfrac{f(x)}{x} \right| < \varepsilon $$
    所以
    $$f(x) = O(x) \quad (x \to 0)$$

\end{proof}

\begin{proposition}

    设$f \in C(I)$,$I$为区间。证明:若$x_0 \in I$是$f$的唯一极值点,则$x_0$一定是最值点;若$x_0$是极小值(极大值)点,则$x_0$是$f$的最小值(最大值)点。

\end{proposition}

\begin{proof}

    不妨设$x_0$是$f$的唯一极小值点,$x_1$是$f$的最小值点,应用反证法,假设$x_0 \neq x_1$,则不妨设$x_0 < x_1$\\
    因为$x_0$是极小值点,所以存在$ x_2 > x_0$,$x_2 < x_1$
    $$f(x_2) > f(x_0) > f(x_1)$$
    由介值定理得,$\exists x_3 \in (x_2, x_1)$
    $$f(x_3) = f(x_0)$$
    则易知$\exists \eta \in (x_3, x_0)$
    $$f'(\eta) = 0$$
    其中$\eta$为极值点,与题设矛盾。\\
    所以$x_0 = x_1$,又若最值点不唯一,即$\exists \alpha \in I,\ \alpha \neq x_0$
    $$f(\alpha) = f(x_0)$$
    同上可知与题设矛盾。

\end{proof}

\begin{proposition}

    若$f:[0,1] \to [0,1]$为

    \begin{enumerate}

            \item 单调增加
            
            \item 单调减少

    \end{enumerate}

    是否存在$x \in [0,1]$,使得$f(x) = x$

\end{proposition}

\begin{proof}

    \begin{enumerate}

        \item 
            存在\\
            令$A = \{x | x \in [0,1] \land f(x) > x \} $,若$f(0) = 0$,则结论显然成立。\\
            若不然,则$A$非空,因此$A$有上确界,不妨设为$a$,令$b = f(a)$

            \begin{enumerate}

                \item $a<b$时\\
                    因为$f$是单调的,$a$为上确界,可得
                    $$b = f(a) \leq f\left(\dfrac{a+b}{2}\right) \leq \dfrac{a+b}{2}$$
                    与$a<b$矛盾。

                \item 
                    $a>b$时\\
                    因为$a$为上确界,所以存在$\eta \in A, \ \eta > \dfrac{a+b}{2}$,则
                    $$b = f(a) \geq f(\eta) > \eta > \dfrac{a+b}{2}$$
                    与$a>b$矛盾。

            \end{enumerate}

            所以$a=b$,即$f(a) = 0$.
        
        \item 
            不一定存在\\
            例如,令
            $$f(x) = \left\{
                \begin{aligned}
                    & 1 - \dfrac{x}{2}, &x \leq \dfrac{1}{2} \\
                    & \dfrac{1}{2} - \dfrac{x}{2} , &x > \dfrac{1}{2} \\
                \end{aligned}
                \right.
            $$
            
    \end{enumerate}

\end{proof}

\begin{proposition}

    设$f,g :[a,b] \to [a,b]$是单调增加函数,且$f \circ g = g \circ f$,证明:$f$和$g$有一个公共不动点。

\end{proposition}

\begin{proof}

    设
    $$A = \{x| x \in [a,b] \land x \leq f(x) \land x \leq g(x)\}$$
    显然$A$非空,所以存在上确界$ u = \sup{A}$\\
    易知$u \leq f(u),\ u \leq g(u)$,所以            
    $$f(u) \leq f(g(u)) = g(f(u))$$ 
    又
    $$f(f(u)) \geq f(u)$$
    即$f(u) \in A$,则
    $$f(u) \leq u$$
    所以
    $$u =f(u) = g(u)$$
    则$u$为$f$和$g$的公共不动点。

\end{proof}

\begin{proposition}
    
    设函数$f$在区间$I$上只有可去间断点,定义
    $$g(x) = \lim_{t \to x}{f(t)}$$
    证明:$g \in C(I)$

\end{proposition}

\begin{proof}

    $\forall x_0 \in I$,因为$g(x_0) = \lim\limits_{t \to x}{f(t)}$,所以存在$\delta > 0, \ t \in \mathring{U}(x_0, \delta)$时
    $$|f(t) - g(x_0)| < \dfrac{\varepsilon}{2}$$
    于是,当$|x - x_0| < \dfrac{\delta}{2}$时,有

    \begin{align*}
        |g(x) - g(x_0)| & = \left|\lim_{t \to x}{f(t) - g(x_0)} \right| \\
        & = \lim_{t \to x}{|f(t) - g(x_0)|} \\
        & \leq \dfrac{\varepsilon}{2} \\
        & < \varepsilon 
    \end{align*}

    所以
    $$\lim_{x\to x_0}{g(x)} = g(x_0)$$
    即$g$在$x_0$处连续,由$x_0$的任意性可知,$g$在区间$I$上连续。

\end{proof}

\begin{proposition}

    设函数$f$和$g$在$[a,b]$上连续,且有$x_n \in [a,b]$,使得
    $$g(x_n) = f(x_{n+1}), \quad n \in \mathbb{N}$$
    证明:必有一点$x_0 \in [a,b]$,使得$f(x_0) = g(x_0)$

\end{proposition}

\begin{proof}

    反证法。若$F(x) = f(x) - g(x) \neq 0,\ x \in [a,b]$,不妨设$F(x) > 0$\\
    则由$F$连续知,$F$在$[a,b]$上有最值,设$\min{F(x)} = m$\\
    则
    $$F(x_{n+1}) = f(x_{x_{n+1}}) - g(x_{n+1}) = f(x_{n+1}) - f(x_{n+2}) \geq m$$
    所以
    
    \begin{align*}
        &f(x_n) - f(x_{n+1}) \geq m \\
        &\cdots \\
        &f(x_1) - f(x_2) \geq m 
    \end{align*}

    则
    $$f(x_{n+1}) \leq f(x_1) - mn$$
    即$f$无界,与题设矛盾。得证。

\end{proof}

\begin{proposition}

    设函数$f$在区间$[0,+\infty)$上连续且有界,证明:对任意给定的$\lambda$,存在一个数列$\{x_n\}$满足

    \begin{enumerate}

        \item $$\lim_{n\to\infty}{x_n} = +\infty$$
        
        \item $$\lim_{n\to\infty}{[f(x_n + \lambda) - f(x_n)]} = 0$$
        
    \end{enumerate}

\end{proposition}

\begin{proof}

    反证法。不失一般性地,我们设$\lambda > 0$,$\lambda = 0$时结论显然成立;若$\lambda < 0$,则有
    $$ - [f(x_n + \lambda  +| \lambda|) - f(x_n + \lambda) ] = f(x_n + \lambda) - f(x_n) $$
    化为$\lambda > 0$的形式。\\
    若命题不成立,则必存在$\lambda > 0,\ \varepsilon > 0, \ \forall M > 0, \ x > M$时
    $$|f(x+\lambda) - f(x)| > \varepsilon$$
    由$f$在$(0,+\infty)$上连续,所以$f(x+\lambda) - f(x)$不变号,不妨设
    $$f(x+\lambda) - f(x) > \delta, \quad x > M$$
    则

    \begin{align*}
        f(x_n + m\lambda) & \geq f[x_n + (m+1)\lambda] + \varepsilon \\
        & \cdots \\
        & \geq f(x_n) + m\lambda
    \end{align*}

    令$m \to \infty$,则
    $$f(x_n + m\lambda)  \to \infty$$
    与$f$有界矛盾。
    
\end{proof}

\begin{proposition}

    设函数$f$在$[0,+\infty)$上一致连续,且任意$x \in [0,1]$有
    $$\lim_{n\to\infty}{f'(x+n)} = 0$$
    证明:
    $$\lim_{n\to\infty}{f(x)} = 0$$

\end{proposition}

\begin{proof}

    因为$f$在$[0,+\infty)$上一致连续,所以$\forall \varepsilon > 0$,$\exists \delta > 0$,当$x', x'' \in [0,+\infty)$时
    $$|f(x') - f(x'')| < \dfrac{\varepsilon}{2}$$
    将$[0,1]$等分成$0 = x_0 < x_1 < \cdots < x_m = 1$,使每个区间的长度为$\dfrac{1}{m} < \delta$\\
    因为$x - \lfloor x \rfloor,\ \forall x \geq 1$,且
    $$x\to+\infty \leftrightarrow \lfloor x \rfloor \to +\infty$$
    所以存在$k \in \{0,1,2,\cdots , m-1\}$
    $$x - \lfloor x \rfloor \in [x_k, x_{k+1}]$$
    即
    $$0 \leq x - \lfloor x \rfloor - x_k < \delta$$
    由于$\lim\limits_{n\to\infty}{f(x+n)} = 0$,$\forall x \in [0,1]$成立,所以对任意$x_k,\ k \in \{0,1,2,\cdots , m-1\}$
    $$\lim_{n\to\infty}{f(x+n)} = 0$$
    因此$\exists N \in \mathbb{N}$,$n >N$时
    $$|f(x_k + n)| < \dfrac{\varepsilon}{2}, \quad k \{0,1,2,\cdots , m\}$$
    当$x > N > 1$时,$\lfloor x \rfloor > N$,且

    \begin{align*}
        |f(x)| & = |f(x) - f(\lfloor x \rfloor + x_k) + f(x_k + \lfloor x \rfloor)| \\
        & = |f(x) - f(\lfloor x \rfloor + x_k)| + |f(x_k + \lfloor x \rfloor)| \\
        & < \dfrac{\varepsilon}{2} + \dfrac{\varepsilon}{2} = \varepsilon
    \end{align*}

    所以
    $$\lim_{n\to\infty}{f(x)} = 0$$

\end{proof}


\begin{proposition}
    
    设函数$f$在区间$[a,b]$上定义,且处处有极限,证明:

    \begin{enumerate}

        \item 对任意$\varepsilon > 0$,在$[a,b]$上使$\left| \lim\limits_{t \to x}{f(t) - f(x)} \right| > \varepsilon$的点至多只有有限个。
        
        \item $f$在$[a,b]$中至多只有可列个间断点。
        
    \end{enumerate}

\end{proposition}

\begin{proof}

    \begin{enumerate}

        \item
            反证法。若对$\varepsilon_0 > 0$,在$[a,b]$上使$\left| \lim\limits_{t \to x}{f(t) - f(x)} \right| > \varepsilon$的点有无限个,\\
            令间断点集合为点列$\{x_n\} \subset [a,b]$,由聚点定理得,存在子列$\{x_k\}$收敛,即
            $$\lim_{k \to \infty}{x_k} = \alpha $$
            又由题设,设$\lim\limits_{t\to \alpha}{f(t)} = \alpha$,由极限性质得知,$\exists \delta > 0$,$|t- a| < \delta$
            $$|f(t) - \alpha| < \dfrac{1}{2} \varepsilon_0$$
            又由\textup{Heine}归结定理得
            $$\lim_{k \to +\infty}{f(x_k)} = \lim\limits_{t\to \alpha}{f(t)} = \alpha$$
            则存在$k_0 \in \mathbb{N}$,$k \geq k_0$时
            $$|f(t) - f(x_k)| \leq |f(t) - \alpha| + |f(x_k) - \alpha| = \dfrac{1}{2} \varepsilon + \dfrac{1}{2} \varepsilon = \varepsilon$$
            其中$|t - x_k|$可任意小,与题设矛盾。
        
        \item 
            设$A = \left\{x \Big| x \in [a,b] \land \left| \lim\limits_{t \to x}{f(t) - f(x)} \right| > 0 \right\}$为间断点集合
            $$A = \bigcup_{n=1}^{\infty}{A_n}$$
            其中
            $$A_n = \left\{x \Big| x \in [a,b] \land \left| \lim\limits_{t \to x}{f(t) - f(x)} \right| > \dfrac{1}{n} \right\}$$
            由\textup{(1)}知$A_n$为有限集,所以$A$至多可列。

    \end{enumerate}

\end{proof}

\begin{proposition}

    设函数$f$在$[0,+\infty)$上定义,且在其中的每个有界子区间上上有界,证明:
    $$\lim_{x \to +\infty}{\dfrac{f(x)}{x}} = \lim_{x \to +\infty}{[f(x+1) - f(x)]}$$

\end{proposition}

\begin{proof}

    因为$\lim\limits_{x \to +\infty}{[f(x+1) - f(x)]} = A$,所以$\forall \varepsilon > 0$,$\exists \delta_1 > \max\{0,a\}$,$x > \delta_1$时
    $$|f(x+1) - f(x) - A| < \dfrac{\varepsilon}{3}$$
    固定$\delta_1$,由题设知$f$在$(\delta_1, \delta_1+1)$上有界,则存在$M > 0$
    $$|f(x)| < M,\quad x \in(\delta_1, \delta_1+1)$$
    选取$\delta > \delta_1$,使
    $$\dfrac{M}{\delta} < \dfrac{\varepsilon}{3}, \quad \left| \dfrac{\delta_1 +1}{\delta} \right| < \dfrac{\varepsilon}{3}$$
    于是,当$x > \delta$时,$\exists n \in \mathbb{N}$,使$x - n \in (\delta_1, \delta_1+1)$\\
    又

    \begin{align*}
        \left| \dfrac{f(x)}{x} - A \right| & = \left| \dfrac{\sum\limits_{i=1}^{n}{[f(x - i) - f(x - i -1) + f(x -n)]}}{x} - A \right| \\
        & = \left| \dfrac{\sum\limits_{i=1}^{n}{[f(x - i) - f(x - i - 1) - A]} + nA + f(x - n)}{x} - A \right| \\
        & \leq \sum_{i = 1}^{n}{\left| \dfrac{f(x -i) - f(x -i -1)}{x} - A \right|} + \left| \dfrac{(x-n)A}{x} \right| + \left| \dfrac{f(x -n)}{x} \right| \\
        & < \dfrac{\varepsilon}{3} \cdot \dfrac{n}{x} + \dfrac{\delta_1 + 1}{\delta} A + \dfrac{M}{\delta} \\
        & < \dfrac{\varepsilon}{3} +\dfrac{\varepsilon}{3} + \dfrac{\varepsilon}{3} = \varepsilon
    \end{align*}

    因此
    $$\lim_{x \to +\infty}{\dfrac{f(x)}{x}} = A =  \lim_{x \to +\infty}{[f(x+1) - f(x)]}$$

\end{proof}

\begin{proposition}

    设$f$是$[a,b]$上的可微函数,且$f(a) = 0$,$\exists A > 0$,$\beta \geq 1$
    $$ |f'(x)| \leq A|f(x)|^{\beta}, \quad \forall x \in [a,b]$$
    证明:$f(x) \equiv 0$

\end{proposition}

\begin{proof}

    反证法。设$f(x) \neq 0,\ x \in [a,b]$\\
    令$S = \{x | x \in [a,b] \land f(x) = 0\}$,则显然$S$非空。设$ c = \inf{S}$\\
    如果$c= b$,则$f(x) = 0, \ \forall x \in [a,b]$,由连续性可知
    $$f(b) = \lim_{x\to b}{f(x)} = 0$$
    与题设矛盾。因此$c \neq b$,又$c =a $,则有$f(c) = 0$,否则$a<c<b$时
    $$f(x) = 0, \quad \forall x < c$$
    则由函数连续性知$f(c) = 0$. 综上
    $$f(c) = 0, \quad a \leq c < b$$
    因为$f$在$x = c$处连续且$f(c) = 0$,则存在$d > c$,使$d$充分接近$c$,且
    $|f(x)| \leq 1$,$A(x - c) \leq \dfrac{1}{2}, \ x \in [c,d]$
    因为$f$在$[c,d]$连续,所以存在最值,即令
    $$ |f(t)| = \max_{x\in[c,d]}{|f(x)|}, \quad t \in [c,d]$$
    又因为$c = \inf{S}$,则$\forall \varepsilon > 0$,$\exists m \in (m, m+\varepsilon)$,使得$f(m) = \neq 0$\\
    则$f(t) \neq 0$,且$t > c$,由\textup{Lagrange}中值定理知
    $$| f(t)| = |f(t) - f(c)| = |t-c||f'(u)| \leq |t-c||f(u)|^{\beta}, \quad u \in (c,t)$$
    又由$A(x-c) \leq \dfrac{1}{2}$得
    $$|t-c||f(u)|^{\beta} \dfrac{1}{2}|f(u)|^{\beta} \leq \dfrac{1}{2} |f(u)|$$
    即$|f(t)| \leq \dfrac{1}{2}|f(u)|$
    则
    $$f(t) = f(u) = 0$$
    也即$f(x) \equiv 0$,与假设矛盾。得证。

\end{proof}

\begin{proposition}
    
    设$y = (1 + \sqrt{x})^{2n+2},\ n \in \mathbb{N}$,求$y^{(n)}(1)$

\end{proposition}

\begin{proof}

    设$f(x) = (1 - \sqrt{x})^{2n+2}$,易知对$f(x)$求$n$次导后,每项依然会有$(1 - \sqrt{x})$,则
    $$f^{(n)}(1) = 0$$
    所以
    $$y^{(n)}(1) = \left[(1 + \sqrt{x})^{2n+2} + (1 - \sqrt{x})^{2n+2}\right]^{(n)}\Big|_{x=1}$$
    $$ (1 + \sqrt{x})^{2n+2} + (1 - \sqrt{x})^{2n+2} = 2 \sum_{k=0}^{n+1}{\binom{2n+2}{2k}x^k}$$
    对右侧函数求$n$次导,易知
    $$y^{(n)}(1) = 4(n+1)(n+1)!$$

\end{proof}

\begin{proposition}
    
    设$f \in C^2(a,b)$,$f + f' + f'' \geq 0$,证明:$f$有下界。

\end{proposition}

\begin{proof}

    设
    $$g(t) = \euler^{\frac{1}{\sqrt{3}}t}f\left( \frac{1}{\sqrt{3}}t \right)$$
    则
    $$g + g'' = \dfrac{4}{3}\euler^{\frac{1}{\sqrt{3}}t} \left[ f\left(\dfrac{1}{\sqrt{3}}t\right) + f'\left(\frac{1}{\sqrt{3}}t\right) + f''\left(\frac{1}{\sqrt{3}}t\right) \right] \geq 0, \quad t \in \left( \dfrac{\sqrt{3}}{2}a, \dfrac{\sqrt{3}}{2}b \right)$$
    选取$s,t$,使得
    $$\max\{b-a, \pi\} < t < s < b$$
    设
    $$h_s(t) = g(t) \cos(t-s) - g'(t)\sin(t-s)$$
    $$h'_s(t) = -[g(t) + g''(t)] \sin(t-s) \geq 0$$
    其中$\sin(t-s) < 0$则
    $$g(s) = h_s(s) \geq h_s(t) \geq -|g(t)| - |g'(t)|$$
    选取任意$t \in (\max\{b-a,\pi\},b)$,上式表明$g$在$(t,b)$上有上界。\\
    设$G(x) = g(a + b -x)$,则显然$G \in C^2(a,b)$
    $$G(x) + G''(x) = g(a+b-x) + g''(a+b-x) \geq 0$$
    所以$G$在$(t,b)$上有下界,即$g$在$(a, a+b-t)$上有下界。\\
    若$a + b -t > t$,即$t < \dfrac{a+b}{2}$时,$g$在$(a,b)$上有界。否则可在$(a+b -t ,t)$上重复以上操作。

\end{proof}

\begin{proposition}

    设$f$是可微实函数,且存在$M > 0$,使得
    $$|f(x+t) - 2f(x) + f(x-t)| \leq Mt^2, \quad \forall x,t$$
    证明:
    $$|f'(x+t) - f'(x)| \leq M|t|$$

\end{proposition}

\begin{proof}

    我们将证明即使没有$f$可微的条件,依然可以得出结论。\\
    由题设得
    $$-Mt^2 \leq f(x+t) - 2f(x) +f(x-t) \leq ^2,\quad \forall x,t$$
    设$h_1(x) = f(x) - \dfrac{M}{2}x^2$,则由上式得
    $$h_1(x+t) - 2h_1(x) + h_1(x-t) = f(x+t) - 2(x) + f(x-t) - Mt^2 \leq 0$$
    这表明$h_1$是$\mathbb{R}$上连续的下凸函数,同理设$h_2(x) = f(x) + \dfrac{M}{2}x^2$,$h_2$是$\mathbb{R}$上连续的上凸函数。则

    \begin{align}
        &h_1^{(-)}(x) \leq h_1^{(+)}(x) \tag{1}\\
        &h_2^{(-)}(x) \leq h_2^{(+)}(x) \tag{2}
    \end{align}

    所以$f^{(-)}$,$f^{(+)}$在任意点有定义,且满足

    \begin{align}
        &f^{(-)}(x) \equiv h_1^{(-)}(x) + Mx \tag{3}\\
        &f^{(+)}(x) \equiv h_1^{(+)}(x) + Mx \tag{4}\\
        &f^{(-)}(x) \equiv h_2^{(-)}(x) + Mx \tag{5}\\
        &f^{(+)}(x) \equiv h_2^{(+)}(x) + Mx \tag{6}
    \end{align}

    由\textup{(1)},\textup{(3)},\textup{(4)}得$f^{(-)} \geq f^{(+)}$,而由\textup{(2)},\textup{(5)},\textup{(6)}得$f^{(-)} \leq f^{(+)}$\\
    因此$f^{(-)} \equiv f^{(+)}$,即$f$在$\mathbb{R}$上可微,且$h_1$,$h_2$也可微,由$h_1$,$h_2$凸性得,$\forall x,y$,$x \leq y$有

    \begin{align*}
        &h_1'(y) - h_1'(x) = f'(y) - f'(x) - M(y-x) \leq 0 \\
        &h_2'(y) - h_2'(x) = f'(y) - f'(x) + M(y-x) \geq 0 
    \end{align*}

    即
    $$|f'(y) - f'(x) | \leq M|y - x|$$

\end{proof}

\begin{proposition}

    设$f:\mathbb{R}\to\mathbb{R}$是二阶可微,周期为$2\pi$的偶函数,证明:若
    $$f''(x) + f'(x) = \dfrac{1}{f(x + \frac{3\pi}{2})}, \quad \forall x \in \mathbb{R}$$
    则$f$的周期为$\dfrac{\pi}{2}$

\end{proposition}

\begin{proof}
    
    由$f$的偶性得
    $$f''(x) = f''(-x), \quad f(x) = f(-x), \quad \forall x \in \mathbb{R}$$
    则
    $$f''(-x) + f(-x) = f''(x) + f(x) = \dfrac{1}{f(-x + \frac{3\pi}{2})} = \dfrac{1}{f(x + \frac{3\pi}{2})},\quad \forall x \in \mathbb{R}$$
    即
    $$f\left(x + \dfrac{3\pi}{2}\right) = f\left(-x + \dfrac{3\pi}{2}\right) = f\left(x - \dfrac{3\pi}{2}\right)$$
    $f$以$3\pi$为周期,又由题设知,$f$周期为$2\pi$,则$f$必以$\pi$为周期。\\
    考察函数
    $$g(x) = f\left(x + \dfrac{\pi}{2}\right)$$
    因为
    $$f(x + \pi) = f(x), \quad \forall x \in \mathbb{R}$$
    则
    $$f''(x) + f(x) = \dfrac{1}{f(x + \frac{3\pi}{2})} = \dfrac{1}{f(x + \frac{\pi}{2})} = \dfrac{1}{g(x)}$$
    又
    $$g(-x) = f\left(-x + \dfrac{\pi}{2}\right) = f\left(x - \dfrac{\pi}{2}\right) = f\left(x + \dfrac{\pi}{2}\right) = g(x)$$
    所以$g(x)$也是偶函数,
    
    \begin{align*}
        &g'(x) = f'\left(x + \dfrac{\pi}{2}\right) \\
        &g''(x) = f''\left(x + \dfrac{\pi}{2}\right)
    \end{align*}

    有

    \begin{align}
        f''(x) + f(x) = \dfrac{1}{g(x)} \tag{1}\\
        g''(x) + f(x) = \dfrac{1}{f(x)} \tag{2}
    \end{align}

    即
    $$fg'' - gf'' = (fg' - g'f)'$$
    所以
    $$f'g - g'f = c, \quad c \in \mathbb{R}$$
    同时由于$g'(x)$,$f'(x)$是奇函数,则$c = 0$\\
    又$f$,$g$为$\mathbb{R}$上周期函数,所以$f$,$g$必有界,则由\textup{(1)}式得$g(x) \neq 0$\\
    故
    $$\left( \dfrac{f}{g} \right)' = \dfrac{c}{g^2} = 0$$
    即
    $$f = ag, \quad a \in \mathbb{R}$$
    又因为$f$在$\mathbb{R}$上连续且是周期函数,则存在最值点$x_0$,$x_1$,其中
    $$f(x_0) = \min{f(x)}, \quad f(x_1) = \max{f(x)}$$
    则有
    
    \begin{align*}
        &g(x_0) = f\left(x_0 + \dfrac{\pi}{2}\right) \geq f(x_0) \\
        &g(x_1) = f\left(x_1 + \dfrac{\pi}{2}\right) \leq f(x_1)
    \end{align*}

    则
    $$g(x) = f(x), \quad \forall x \in \mathbb{R}$$
    即
    $$f\left(x + \dfrac{\pi}{2}\right) = f(x), \quad \forall x \in \mathbb{R}$$

\end{proof}

\begin{proposition}
    
    设$f \in \mathbb{C}^{n}[0,1)$,满足
    $$f^{(k)} \leq 1 + |f| + |f'| + \cdots + |f^{(k-1)}|,\quad k \leq n$$
    证明:$f$有上界

\end{proposition}

\begin{proof}
    因为$f \in \mathbb{C}^{n}[0,1)$,由\textup{Taylor}公式知
    $$f(x) = f(0) + \dfrac{f'(0)x}{1!} + \dfrac{f''(0)x^2}{2!} + \cdots + \dfrac{f^{(n)(0)x^n}}{n!} + O(x^n)$$
    由题设
    $$f'(x) \leq 1 + |f(x)|$$
    $$f''(x) \leq 1 + |f(x)| + |f'(x)| \leq 2(1 + |f(x)|)$$
    应用数学归纳法可证
    $$f^{(k)}(x) \leq 2^{k-1}(1 + |f(x)|), \quad k \leq n$$
    所以

    \begin{align*}
        f(x) & \leq f(0) + (1 + |f(0)|)\left(\dfrac{x}{1!} + \dfrac{2x^2}{2!} + \cdots + \dfrac{2^{n-1}x^n}{n!}\right) + O(x^n) \\
        & = f(0) + \dfrac{1 + |f(0)|}{2}\left(\dfrac{2x}{1!} + \dfrac{(2x)^2}{2!} + \cdots + \dfrac{(2x)^{n}}{n!}\right) + O(x^n) \\
        & \leq f(0) + \dfrac{1}{2}(1 + |f(0)|)(\euler^{2x}-1) + O(x^n)
    \end{align*}

    因为$f$定义在$[0,1)$上,所以
    $$f(x) \leq f(0) + \dfrac{1}{2}(1 + |f(0)|)(\euler^2 - 1) + 1$$
    即$f$有上界
    
\end{proof}

\begin{proposition}
    设函数$f(x)$在区间$[0,+\infty]$上连续,且$\lim\limits_{x\to+\infty}{f(x)} = f(0)$,常数$h > 0$\\
    证明:存在$\xi \in [h,+\infty)$,使得
    $$f(\xi) = f(\xi - h)$$
\end{proposition}

\begin{proof}

    反证法。令
    $$g(x) = f(x) - f(x - h)$$
    则$g(x)$在区间$[0,+\infty)$上连续。则
    $$g(x) \neq 0, \quad x \in [0,+\infty)$$
    不妨设
    $$g(x) > 0, \quad \forall x \in [0,+\infty)$$
    
    \begin{align*}
        S_n & = g(h) + g(2h) + \cdots + g(nh) \\
        & = [f(h) - f(0)] + [f(2h) - f(h)] + [f(3h) - f(2h)] + \cdots + [f(nh) - f((n-1)h)] \\
        & = f(nh) - f(0)
    \end{align*}
  
    因为$g(x) > 0, \ \forall x \in [0,+\infty)$,所以$S_n > 0$,然而
    $$\lim\limits_{n\to\infty}{S_n} = \lim\limits_{n\to\infty}{f(nh)} - f(0) = 0$$
    矛盾,得证。

\end{proof}

\begin{proposition}

    设$f \in C(0,1)$,且
    $$\dfrac{f(x_2) - f(x_1)}{x_2 - x_1} = \alpha < \beta = \dfrac{f(x_4)- f(x_3)}{x_4 - x_3}$$    
    其中$x_1$,$x_2$,$x_3$,$x_4 \in (0,1)$.证明:对任意$\lambda \in (\alpha, \beta)$,存在$x_5$,$x_6 \in (0,1)$,使得
    $$\lambda = \dfrac{f(x_6) - f(x_5}{x_6 - x_5}$$

\end{proposition}

\begin{proof}

    设
    $$F(t) = \dfrac{f((1 - t)x_2 + tx_4) - f((1 - t)x_1 + tx_3)}{(1 - t)(x_2 - x_1) + t(x_4 - x_3)}$$
    则$F$在$[0,1]$上连续,且$\alpha = F(0) < \lambda < F(1) = \beta$. 根据连续函数的介值定理,存在$t_0 \in (0,1)$,使得$\lambda = F(t_0)$. 令
    $$x_5 = (1 - t_0)x_1 + t_0 x_3$$
    $$x_6 = (1 - t_0)x_2 + t_0 x_4$$
    则
    $$\lambda = F(t_0) = \dfrac{f(x_6) - f(x_5}{x_6 - x_5}$$

\end{proof}

\section{一元函数微分学}

\begin{proposition}
    
    设$f$在$[a,b]$上连续,在$(a,b)$上可导,且$\exists c\in (a,b)$,$f'(c)=0$.
    证明:$\exists \xi \in (a,b)$
    $$f'(\xi) = \dfrac{f(\xi)-f(a)}{b-a}$$

\end{proposition}

\begin{proof}
    
    设$F(x)= [f(x)-f(a)]\euler^{-\frac{x}{b-a}}$
    则$F'(x)=[f'(x)-\frac{f(x)-f(a)}{b-a}]\euler^{-\frac{x}{b-a}}$
    $$F'(c)= -\left(\dfrac{f(c)-f(a)}{b-a}\right)\euler^{-\frac{x}{b-a}}=-\dfrac{F(c)}{b-a}$$
    由\textup{Lagrange}中值定理得
    $\exists \eta \in (a,b)$,
    $$F'(\eta) = \dfrac{F(c)-F(a)}{c-a} = \dfrac{F(c)}{c-a}$$
    因为$F'(\eta)$与$F'(c)$异号,所以由达布定理知,$\exists \eta \in (a,b)$
    $$F'(\eta) = 0$$
    即
    $$f'(\xi) = \dfrac{f(\xi)-f(a)}{b-a}$$

\end{proof}

\begin{theorem}[Flett中值定理]
    
    设$f$在$[a,b]$上可微,$f'(a)=f'(b)$,证明:
    $\exists \xi \in (a,b)$,使得
    $$f'(\xi) = \dfrac{f(\xi)-f(a)}{\xi - a}$$

\end{theorem}

\begin{proof}
    
    设$$h(x)=\left\{
        \begin{aligned}
            &\dfrac{f(b)-f(x)}{b-x}, &&a \leq x < b\\
            &f'(b), &&x = b\\
        \end{aligned}
    \right.
    $$
    $$g(x)=\left\{
        \begin{aligned}
            &f'(a), &&x = a\\
            &\dfrac{f(x)-f(a)}{x-a}, &&a < x \leq b\\
        \end{aligned}
    \right.
    $$
    又令$F(X)=g(x)-h(x)$,则有
    $$F(a) = g(a) - h(a) = f'(a) - \dfrac{f(x)-f(a)}{b-a}$$
    $$F(b) = g(b) - h(b) = \dfrac{f(b) - f(a)}{b-a}  - f'(b) = \dfrac{f(b) - f(a)}{b-a} - f'(a)$$
    易见$F(a)F(b)$异号,所以存在$\eta \in (a,b)$,使得
    $$ f(\eta) = g(\eta) -h(\eta) = 0$$
    即
    $$g(\eta) = \dfrac{f(\eta) - f(a)}{\eta - a }= \dfrac{f(b) - f(a)}{b-a}$$
    则存在$\xi \in (\eta,b)$使得
    $$g'(\xi) = \dfrac{(\xi -a )f'(\xi)-[f(\xi) - f(a)]}{(\xi - a)^2}$$
    即
    $$f'(\xi) = \dfrac{f(\xi)-f(a)}{\xi - a}$$

\end{proof}

\begin{proposition}

    设$f:[a,b]\to \mathbb{R}$,$f(0)=f(1)$.
    证明:$\forall n\in \mathbb{N}$,$\exists \xi \in \left[0,1-\dfrac{1}{n}\right]$,使得
    $$f(\xi_n) = f\left(\xi_n+\dfrac{1}{n}\right)$$

\end{proposition}

\begin{proof}
    
    设$F(x) = f(x) - f\left(x + \dfrac{1}{n}\right),\quad \forall n \in \mathbb{N}$\\
    若不存在$\xi_n$,使得$f(\xi_n) = f\left(\xi_n+\dfrac{1}{n}\right)$\\
    不妨设$F(x)$恒大于$0$,则有
    $$f(1) = f(0) > f\left(\dfrac{1}{n}\right) > f\left(\dfrac{2}{n}\right) > \cdots > f(1)$$
    与题设矛盾,所以存在$\xi_n \in \left[0,1-\dfrac{1}{n}\right]$
    $$f(\xi_n) = f\left(\xi_n + \dfrac{1}{n}\right)$$

\end{proof}

\begin{proposition}
    
    设$f$与$g$是两个周期函数,且$\lim\limits_{n\to\infty}{(f(x)-g(x))}=0$.
    证明:
    $$f = g $$

\end{proposition}

\begin{proof}
    
    设$f$与$g$的周期分别为$T$和$S$,则对任意$x \in \mathbb{R}$,有
    
    \begin{align*}
        f(x)-g(x) &= \lim_{n\to\infty}{f(x)-g(x)}\\
        &= \lim_{n\to\infty}[(f(x+nT)-g(x+nT))+\\
        &\quad(g(x+nT+nS)-f(x+nT+nS)+f(x+nS)-g(n+nS))]\\
        &= 0 + 0 + 0\\
        &= 0
    \end{align*}

    所以
    $$f(x)=g(x)$$

\end{proof}

\begin{proposition}
    
    设$f$为$\mathbb{R}$上的连续函数,并且$\lim\limits_{n\to\infty}{f(x)}=+\infty$,又设$f$的最小值$f(a)<a$,
    证明:\\
    $f(f(x))$至少有两个最小值点。
\end{proposition}

\begin{proof}
    
    由题设知,$f(x)$的至于为$[f(a),+\infty]$,且$\lim\limits_{x\to\infty}{f(x)}=+\infty$\\
    因为$f(a)<a$,所以存在$x_1$,$x_2$,$x_1 \neq x_2$,$f(x_1) = a = f(x_2)$\\
    显然$f(f(x))\geq f(a) = f(f(x_1)) = f(f(x_2))$,\\
    所以$f(f(x))$至少有两个最小值点。

\end{proof}

\begin{proposition}[Cauthy方程]
    
    设$f$对任意$x$,$y \in \mathbb{R}$,满足方程$f(x+y) = f(x) + f(y)$,证明:

    \begin{enumerate}

        \item 若$f$在某一点$x_0$处连续,则$f(x) = f(1)x$
        
        \item 若$f$在$\mathbb{R}$单调,也有$f(x) = f(1)x$
        
    \end{enumerate}

\end{proposition}

\begin{proof}
    
    \begin{enumerate}
        
        \item 
            因为$f(x+y) = f(x) + f(y)$,令$x=y=0$,得$f(0)=0$\\
            因为$f(0) = f(x-x) = f(x) + f(-x), \quad \forall x$,所以$f(x) = -f(-x)$\\
            易知$f(2) = f(1) + f(1) = 2f(1)$,由数学归纳法易证
            $$f(n) = nf(1)$$
            对任意有理数$q=\dfrac{m}{n},\ m,\ n \in \mathbb{N}$,由于$f(1) = f\left(n \cdot \dfrac{1}{n}\right)$,
            故$f\left(\dfrac{1}{n}\right) = \dfrac{1}{n}f(1)$,从而
            $$f\left(\dfrac{m}{n}\right) = f\left(m \cdot \dfrac{1}{n}\right) = mf\left(\dfrac{1}{n}\right) = \dfrac{m}{n}f(1)$$
            则只需证$f(x) = f(1)x,\quad x \notin \mathbb{Q}$\\
            $f$在$x_0$处连续,故$\lim\limits_{x\to x_0}{f(x)} = f(x_0)$,也即
            $$\lim\limits_{\Delta x\to0}{f(x_0+\Delta x)} = f(x_0)$$
            因为$f(x+\Delta x) = f(x_0) + f(\Delta x)$,所以
            $$\lim\limits_{\Delta x\to 0}{f(\Delta x)} = 0$$
            对于任意$x \in \mathbb{R}$

            \begin{align*}
                \lim\limits_{\Delta x\to 0}{f(x+\Delta x)} &= \lim\limits_{\Delta x\to 0}{(f(x) + f(\Delta x))}\\
                & = f(x) + 0\\
                & = f(x)
            \end{align*}

            所以$f$在$\mathbb{R}$连续。\\
            设$p_0$为任意无理数,则存在有理数列$\{p_n\}\to p_0(n\to\infty)$
            $$f(p_0) = \lim_{n\to\infty}{f(p_n)} = \lim_{n\to\infty}{p_nf(1)} = p_0f(1)$$
            
        \item 
            不妨设$f$在$\mathbb{R}$上单调增加。$\forall x \in \mathbb{R}$,取有理数列$\{x_n'\},\ \{x_n''\}$,\\
            使$x_n'f(1) < x < x_n''$,且
            $$\lim_{n\to\infty}{x_n'} = \lim_{n\to\infty}{x_n''}=x$$
            由$f$单调增加的条件,
            $$x_n'f(1) < f(x_n') \leq f(x) \leq f(x_n'') = x_n''f(1)$$
            令$n\to+\infty$,
            $$xf(1) \leq f(x) \leq xf(1)$$
            即
            $$f(x) = xf(1)$$

    \end{enumerate}

\end{proof}

\begin{proposition}
    
    设函数$f(x)$在$[a,b]$连续,在$(a,b)$上可导,证明:$f(x)$在$[a,b]$内存在相异的两点$\xi,\ \eta$,使
    $$f'(\xi)f'(\eta) = \left( \dfrac{f(b) - f(a)}{b-a} \right)^2$$

\end{proposition}

\begin{proof}

    令$k = \dfrac{f(b) - f(a)}{b-a},\ h(x) = [f(x) = f(a)]^2 - k^2(x-a)^2$
    $$h'(x) = 2[f(x) - f(a)]f'(x) - 2k^2 (x-a)$$
    因为$f(a) = f(b) = 0$,由\textup{Rolle}中值定理知$\exists \xi \in (a,b)$,使得
    $$h'(\xi) = 0$$
    即
    $$f'(\xi) = k^2 \cdot \dfrac{\xi - a}{f(\xi) - f(a)}$$
    又由\textup{Lagrange}中值定理知
    $$\dfrac{f(\xi) - f(a)}{\xi - a} = f'(\eta), \quad \eta \in (a,\xi)$$
    所以
    $$f'(\xi)f'(\eta) = \left( \dfrac{f(b) - f(a)}{b-a} \right)^2$$

\end{proof}

\begin{proposition}
    
    设函数$f(x)$在闭区间$[0,1]$上连续,在$(0,1)$上可微,证明:
    $\exists \xi,\  \eta \in (0,1)$,使
    $$2\eta f(1) + (c^2 - 1)f'(\eta) = f(\xi)$$
    其中,$c \in (0,1)$
    
\end{proposition}

\begin{proof}

    令$F(x) = x^2f(1) + (c^2 - 1)f'(\eta) = f(\xi)$,则有
    $$F(1) - F(0) = c^2f(1) + (1 - c^2)f(0)$$
    $F(x)$在$[0,1]$上连续,由介值定理得,$\exists \xi \in (0,1)$
    $$f(\xi) = \dfrac{F(1) - F(0)}{1 - 0} = F(1) - F(0)$$
    又由\textup{Lagrange}中值定理知,$\exists \xi \in (0,1)$
    $$F'(\xi) = \dfrac{F(1) - F(0)}{1 - 0} = 2\eta f(1) + (c^2 - 1)f'(\eta) $$
    即
    $$2\eta f(1) + (c^2 - 1)f'(\eta) = f(\xi)$$

\end{proof}

\begin{proposition}

    设函数$f(x)$在闭区间$[0,1]$上连续,在$(0,1)$上可微,$f(0) = 0,\ f(1) = \dfrac{1}{2}$,证明:$\exists \xi,\eta \in (0,1),\ \xi \neq \eta$,使得
    $$f'(\xi) + f'(\eta) = \xi + \eta$$

\end{proposition}

\begin{proof}

    令$F(x) = f(x) - \dfrac{1}{2}x^2$,则$F(0) = F(1) = 0$\\
    若$F\left( \dfrac{1}{2} \right) = F(0) = F(1) = 0$,则存在$\xi,\eta \in (0,1),\ \xi \neq \eta$
    $$F'(\xi) + F'(\eta) = 0$$
    即
    $$f'(\xi) + f'(\eta) = \xi + \eta$$
    若$F\left( \dfrac{1}{2} \right) \neq 0$,不妨设$F\left( \dfrac{1}{2} \right) > 0$\\
    由\textup{Lagrange}中值定理,$\exists \xi,\eta \in (0,1),\ \xi \neq \eta$
    $$ \dfrac{F(1) - F(\frac{1}{2})}{1 - \frac{1}{2}} = F'(\xi) = -2F\left( \dfrac{1}{2} \right)$$
    $$ \dfrac{F(\frac{1}{2}) - F(0)}{\frac{1}{2} - 0} = F'(\eta) = 2F\left( \dfrac{1}{2} \right)$$
    则
    $$F'(\xi) + F'(\eta) = 0$$
    即
    $$f'(\xi) + f'(\eta) = \xi + \eta$$

\end{proof}

\begin{proposition}

    设函数$f(x),g(x)$在闭区间$[0,1]$上连续,在$(0,1)$上可微,且$f(0) = g(0) = f(1) = 0$,$g'(x) \neq 0$,证明: $\exists \xi,\eta \in (0,1),\ \xi < \eta$使得
    $$\dfrac{f'(\xi)}{g'(\xi)} = \dfrac{f'(\eta)}{g'(\eta)}$$

\end{proposition}

\begin{proof}

    $g(x)$在闭区间$[0,1]$上连续,则存在$\alpha \in (0,1)$,使得
    $$g(\alpha) = \dfrac{1}{2}[g(1) + g(0)]$$
    由\textup{Lagrange}中值定理得, $\exists \xi,\eta \in (0,1)$
    $$\dfrac{f(\alpha) - f(0)}{g(\alpha) - g(0)} = \dfrac{f'(\xi)}{g'(\xi)}$$
    $$\dfrac{f(1) - f(\alpha)}{g(1) - g(\alpha)} = \dfrac{f'(\eta)}{g'(\eta)} = -\dfrac{f'(\xi)}{g'(\xi)}$$
    则
    $$\dfrac{f'(\xi)}{g'(\xi)} = \dfrac{f'(\eta)}{g'(\eta)}$$

\end{proof}

\begin{proposition}

    设$f:\mathbb{R}\to\mathbb{R}$二阶可微,且满足$f(0) = 2,\ f'(0) = -2,\ f(1) = 1$,证明:$\exists \xi \in (0,1)$,使得
    $$f(\xi) + f'(\xi) + f''(\xi) = 0$$

\end{proposition}

\begin{proof}

    定义函数$g(x) = \dfrac{1}{2}f^2(x) + f'(x)$,则
    $$g'(x) = f(x) f'(x) + f''(x)$$
    因为$g(0) = 0$,所以只需证明$\exists \eta \in (0,1]$,使得$g(\eta) = 0$

    \begin{enumerate}

        \item 若$f$无零点,则令
            $$h(x) = \dfrac{x}{2} - \dfrac{1}{f(x)}$$
            因为$h(0) = h(1) = -\dfrac{1}{2}$,所以存在$\eta \in (0,1)$,使得
            $$h'(\eta) = 0$$
            又由$g(x) = f^2(x) h'(x)$,
            所以
            $$g(\eta) = 0$$

        \item 
            若$f$至少有一个零点,令$z_1$为第一个零点,$z_2$为最后一个零点,由题设知,$0 < z_1 \leq z_2 < 1$\\
            函数$f$在区间$[0,z_1],\ [z_2,1]$上是正的,这表明
            $$f'(z_1) \leq 0,\quad f'(z_2) \geq 0$$
            所以
            $$g(z_1) = f'(z_1) \leq 0, \quad g(z_2) = f'(z_2) \geq 0$$
            则存在$\eta \in [z_1,z_2]$,使得
            $$g'(\eta) = 0$$

    \end{enumerate}

\end{proof}

\begin{proposition}

    设函数$f(x)$可导,曲线$y = f(x)$上存在三点$(x_1,f(x_1)),\ (x_2,f(x_2)), \ (x_3,f(x_3))$共线,其中$0 < x_1 < x_2 < x_3$,证明:$\exists \xi,\  \eta \in (x_1,x_3)$,使得
    $$\xi f'(\eta) - f(\xi) = \eta f'(\eta) - f(\eta)$$

\end{proposition}

\begin{proof}

    若$\xi = \eta$则结论显然成立。\\
    $\xi \neq \eta$时,问题等价于
    $$ f'(\eta) = \dfrac{f(\eta) - f(\xi)}{\eta - \xi}$$
    因为三点共线,所以存在$\alpha$,$\beta \in (x_1, x_3)$,$\alpha \neq \beta$
    $$f'(\alpha) = f'(\beta)$$
    又由\textup{Flett}中值定理得,$\exists \xi,\eta \in (x_1,x_3)$
    $$f'(\eta) = \dfrac{f(\eta) - f(\alpha)}{\eta - \alpha} = \dfrac{f(\eta) - f(\xi)}{\eta - \xi}$$

\end{proof}


\begin{proposition}
    
    设$f$是$[0,1]$上的连续函数,且$f$在$(0,1)$上可微,$f(0) = 0,f(1) = 1$,证明:存在点列$\{\alpha_k\}$,使得
    $$\sum_{k=0}^{n}{\dfrac{1}{f'(\alpha_k)}} = n, \quad n \in \mathbb{N}$$
    
\end{proposition}

\begin{proof}

    由介值定理知,存在$n$个实数$b_0,\ b_1,\cdots, b_n$,其中
    $$0 = b_0 < b_1 < \cdots < b_n = 1$$
    且
    $$f(b_i) = \dfrac{i}{n}, \quad i = 0,1,\cdots,n$$
    由\textup{Lagrange}中值定理知,对于任意区间$[b_i, \ b_{i+1}],\ i = 1,2,\cdots,n$,$\exists \alpha_i \in (b_{i-1}, \ b_i)$
    $$f'(\alpha_i) = \dfrac{f(b_i) - f(b_{i-1})}{b_i - b_{i-1}} = \dfrac{1}{n(b_i - b_{i-1})}$$
    $$\sum_{i=1}^{n}{\dfrac{1}{f'(\alpha_i)}} = n \sum_{i=0}^{n}{(b_i - b_{i-1})} = n(b_n - b_0) = n$$

\end{proof}

\begin{proposition}

    设$f$在$[a,b]$上$n+1$阶可微,$f^{(n+1)}(x_0) \neq 0$,$n \in \mathbb{N}$,在$[a,b]$上有
    $$f(x) = f(x_0) + f'(x_0)(x - x_0) + \cdots + \dfrac{f^{(n)}(\xi)(x - x_0)^n}{n!}$$
    证明:
    $$\lim_{n\to\infty}{\dfrac{\xi - x_0}{x - x_0}} = \dfrac{1}{n + 1}$$

\end{proposition}

\begin{proof}

    分别将$f(x)$在$x = x_0$处展开为$n+1$阶\textup{Peano}余项的\textup{Taylor}公式和$n-1$阶\textup{Lagrange}余项的\textup{Taylor}公式
    $$f(x) = f(x_0) = f'(x_0)(x - x_0) + \cdots + \dfrac{f^{(n+1)}(x_0)(x - x_0)^{n+1}}{(n+1)!} + O(x - x_0)^{n+1}$$
    $$f(x) = f(x_0) + f'(x_0)(x- x_0) + \cdots + \dfrac{f^{(n-1)}(x_0)(x - x_0)^{n-1}}{(n-1)!} + \dfrac{f^{(n)}(\xi)(x - x_0)^n}{n!}$$
    两式相减有
    $$\dfrac{(f^{(n)}(\xi) - f^{(n)}(x_0))(x - x_0)^n}{n!} = \dfrac{f^{(n+1)}(x_0)(x - x_0)^{n+1}}{(n+1)!} + O(x - x_0)^{n+1}$$
    $$\dfrac{f^{(n)}(\xi) - f^{(n)}(x_0)}{\xi - x_0} \dfrac{\xi - x_0}{x- x_0} \dfrac{(x - x_0)^{n+1}}{n!} = \dfrac{f^{(n+1)}(x_0)(x - x_0)^{n+1}}{(n+1)!} + O(x - x_0)^{n+1}$$
    两边对$x$取极限,得
    $$\lim_{n\to\infty}{\dfrac{\xi - x_0}{x - x_0}} = \dfrac{1}{n + 1}$$

\end{proof}

\begin{proposition}

    设$\mathbb{R}$上的函数$f$有任意阶导数,并且对于任意$k \in \mathbb{N}$,存在$C_k > 0$使得
    $$\sup_{x \in \mathbb{R}}\left(|x|^k|f(x)| + |f^{(k)}(x)|\right) \leq C_k$$
    证明:对于任意$k,\ l \in \mathbb{N}$,存在$C_{k,l} > 0$使得
    $$\sup_{x \in \mathbb{R}}|x|^k|f^{(l)}(x)| \leq C_{k,l}$$

\end{proposition}

\begin{proof}

    观察到,问题可以约化为证明:对于任意$k \in \mathbb{N}^{+}$,存在$D_k > 0$,对于任意绝对值大于$2$的数$x \in \mathbb{R}$,成立
    $$|x|^k|f'(x)| \leq D_k$$
    任意给定$\varepsilon > 0$,由\textup{Taylor}定理知,存在$\theta \in (0,1)$,使得
    $$f(x + \varepsilon) - f(x) = f'(x)\varepsilon + \dfrac{1}{2}f''(x + \theta \varepsilon) \varepsilon^2$$
    于是我们有
    $$|f'(x)| \leq \dfrac{|f(x + \varepsilon)| + |f(x)|}{\varepsilon} + \dfrac{1}{2}|f''(x + \theta \varepsilon)| \varepsilon$$
    将$\varepsilon = \dfrac{1}{|x|^k}$代入上式,并利用已知条件,得到

    \begin{align*}
        |f'(x)| & \leq \dfrac{C_2}{2}|x|^{-k} + |x|^{-k} \cdot |x|^{2k}(|f(x)| + |f(x + |x|^{-k})|) \\
        & \leq \dfrac{C_2}{2}|x|^{-k} + C_{2k}|x|^{-k} + |x|^{-k} \cdot (2|x + |x|^{-k}|)^{2k}|f(x + |x|^{-k})| \\
        & \leq \left( \dfrac{C_2}{2} + (2^{2k} + 1)C_{2k} \right) |x|^{-k}
    \end{align*}

    取$D_k = \dfrac{C_2}{2} + (2^{2k} + 1)C_{2k}$,即证。
    
\end{proof}
