\chapter{离散数学}

\section{数理逻辑}

\section{集合论}

\subsection{函数}

\begin{proposition}

    设函数$f : \mathbb{N} \to \mathbb{R}$
    $$f(n) = \sum\limits_{k = 1}^{n}{\dfrac{1}{\sin{k}}}$$
    证明:函数$f$是单射。

\end{proposition}

\begin{proof}

    易知$\sin{n} = \dfrac{1}{2\imag}(\euler^{\imag n} - \euler^{-\imag n})$,则有
    $$f(n) = \sum\limits_{k = 1}^{n}{\dfrac{1}{\sin{k}}} = \dfrac{1}{2\imag}\sum\limits_{k = 1}^{n}{\dfrac{1}{\euler^{\imag k} - \euler^{-\imag k}}} = \dfrac{1}{2\imag}\sum\limits_{k = 1}^{n}{\dfrac{1}{z^n - z^{-n}}}$$
    其中$z = \euler^{\imag k}$. 因为$z$为超越数,所以$z$不能是任何有理系数代数方程的根。
    反证法。假设存在$n_1$,$n_2$,$n_1 < n_2$,使得$f(n_1) = f(n_2)$,则有
    $$f(n_2) - f(n_1) = \dfrac{1}{2\imag}\sum\limits_{k = n_1 + 1}^{n_2}{\dfrac{1}{z^n - z^{-n}}} = 0$$
    即
    $$\sum\limits_{k = n_1 + 1}^{n_2}{\dfrac{1}{z^n - z^{-n}}} = 0$$
    而上式为有理函数,与$z$为超越数的性质矛盾。得证。
    
\end{proof}

\section{代数结构}

\subsection{代数系统}

\begin{proposition}

    $$V_1 = <\mathbb{Z}, +, \cdot>, \quad V_2 = <\mathbb{Z}_n, \oplus, \otimes>$$
    其中,$\mathbb{Z}$为整数集,$+$,$\cdot$分别为普通加法和乘法,$\mathbb{Z}_n = \{1, 2, 3, \cdots, n - 1\}$,
    $\oplus$,$\otimes$分别为模$n$加法和模$n$乘法,令$f: \mathbb{Z} \to \mathbb{Z}_n$,$f(x) = x \bmod{n}$. 证明$f$为$V_1$到$V_2$的满同态映射。

\end{proposition}

\begin{proof}

    $\forall x, y \in \mathbb{Z}$
    $$f(x + y) = (x + y) \bmod{n}$$
    $$f(x) \oplus f(y) = [(x \bmod{n}) + (y \bmod{n})] \bmod{n}$$
    由取模运算的定义知,$\exists q_1, q_2 \in \mathbb{Z}$,$r_1, r_2 \in \mathbb{Z}_n$
    $$x = q_1 n + r_1$$
    $$y = q_2 n + r_2$$
    则
    
    \begin{align*}
        (x + y) \bmod{n} & =  q_1 n + r_1 + q_2 n + r_2 \\
        & = (q_1 + q_2)n + r_1 + r_2 \\
        & = (q_1 + q_2 + q_3)n + r_3 \\
        & = [(x \bmod{n}) + (y \bmod{n})] \bmod{n} \\
        & = (r_1 + r_2) \bmod{n}
    \end{align*}

    其中
    $$r_1 + r_2 = q_3 n + r_3, \quad q_3 \in \mathbb{Z}, r_3 \in \mathbb{Z}_n$$
    即
    $$f(x + y) = f(x) \oplus f(y)$$
    同理可证
    $$f(x \cdot y) = f(x) \otimes f(y)$$
    且$f$显然是满映射,得证。

\end{proof}

\subsection{群与环}

\begin{theorem}

    $G$为群,则$G$中满足消去律,即对任意$a, b, c \in G$有
    \begin{enumerate}

        \item 若$ab = ac$,则$b = c$
        
        \item 若$ba = ca$,则$b = c$
        
    \end{enumerate}

\end{theorem}

\begin{proof}

    \begin{enumerate}

        \item 
            因为$ab = ac$,且$G$是群,所以存在$a^{-1} \in G$, 两边左乘$a^{-1}$得
            $$a^{-1} ab = a^{-1}ac$$
            即$b = c$,得证。

        \item 同上。
        
    \end{enumerate}

\end{proof}

\begin{proposition}

    设$G$是群,$a, b \in G$是有限阶元。证明:

    \begin{enumerate}

        \item $|b^{-1}ab| = |a|$
        
        \item $|ab| = |ba|$
        
    \end{enumerate}

\end{proposition}

\begin{proof}

    \begin{enumerate}

        \item 
            设$|b^{-1}ab| = k$,$|a| = l$,则
            $$[b^{-1}ab]^l = \overbrace{[b^{-1}ab \  b^{-1}ab \ \cdots \  b^{-1}ab]}^{共l项} = b^{-1}a^{l}b = b^{-1}\circ e \circ b = e$$
            即$k \leq l$,同理设$|bb^{-1}abb^{-1}| = m$,显然$m = l$,同时
            $$[bb^{-1}abb^{-1}]^k = \overbrace{[b(b^{-1}ab)b^{-1} \  b(b^{-1}ab)b^{-1} \ \cdots \  b(b^{-1}ab)b^{-1}]}^{共k项} = b(b^{-1}ab)^kb^{-1} = b^{-1}\circ e \circ b = e$$
            因此$l = m \leq k \leq l$,即
            $$|b^{-1}ab| = |a|$$
            
        \item 
            $|ba| = |b^{-1}bab| = |ba|$,得证。
        
    \end{enumerate}

\end{proof}

\begin{proposition}

    偶数阶群必含$2$阶元。

\end{proposition}

\begin{proof}

    由群的性质和逆元的唯一性知,$\forall a \in G$,若$|a| > 2$,则存在$a^{-1} \in G$,且$a \neq a^{-1}$ \\
    即群中除单位元外的任何元素都和它的逆元成对存在。因此,偶数阶群必存在$2$阶元。

\end{proof}
