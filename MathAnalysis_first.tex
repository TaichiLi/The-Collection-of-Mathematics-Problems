\chapter{数学分析}

\section{数列与极限}

\begin{proposition}[Cauthy命题]

    设$\{x_n\}$收敛于$l$,则它的前$n$项的算术平均数也收敛于$l$,即有
    \[\lim\limits_{x \to \infty}{\dfrac{x_1 + x_2 + \cdots + x_n}{n}} = l\]

\end{proposition}

\begin{proof}

    $\{x_n\}$收敛于$l$,则$\forall \varepsilon > 0$,$\exists N \in \mathbb{N}$,$n > N$ 时
    \[|x_n - l| < \varepsilon\]
    则有

    \begin{align*}
        \left| \dfrac{x_1 + x_2 + \cdots + x_n}{n} - l \right| & = \left| \dfrac{\sum\limits_{i = 1}^{n}{x_i - l}}{n} \right| \\
        & \leq \dfrac{\sum\limits_{i = 1}^{n}{|x_i - l|}}{n} \\
        & = \dfrac{\sum\limits_{i = 1}^{N}{|x_i - l|}}{n} + \dfrac{\sum\limits_{i = N + 1}^{n}{|x_i - l|}}{n} \\
        & \leq \dfrac{M}{n} + \dfrac{(n - N)\varepsilon}{n}
    \end{align*}

    其中$M = \sum\limits_{i = 1}^{N}{|x_i - l|}$,$M$为有限值。\\
    $N_1 = \max\left\{ \left\lfloor \dfrac{M}{\varepsilon} \right\rfloor, N \right\}$,当$N > N_1$时成立不等式
    \[\left| \dfrac{x_1 + x_2 + \cdots + x_n}{n} - l \right| < \varepsilon\]
    也即
    \[\lim\limits_{x \to \infty}{\dfrac{x_1 + x_2 + \cdots + x_n}{n}} = l\]

\end{proof}

\begin{corollary}[Cauthy命题推论]

    若$\lim\limits_{n \to \infty}{(a_n - a_{n - 1})} = d$,
    则
    \[\lim\limits_{n \to \infty}{\dfrac{a_n}{n}} = d\]

\end{corollary}

\begin{proof}

    因为$\lim\limits_{n \to \infty}{(a_n - a_{n - 1})} = d$,所以数列$\{a_n - a_{n - 1}\}$收敛于$d$. \\
    由\textup{Cauthy}命题知,
    
    \begin{align*}
        & \lim\limits_{n \to \infty}{\dfrac{(a_2 - a_1) + (a_3 - a_2) + \cdots + (a_{n + 1} - a_n)}{n}} \\
        = & \ \lim\limits_{n \to \infty}{\dfrac{a_{n + 1} - a_1}{n}} \\
        = & \ \lim\limits_{n \to \infty}{\dfrac{n + 1}{n} \dfrac{a_{n + 1}}{n + 1}} - \lim\limits_{n \to \infty}{\dfrac{a_1}{n}} = d
    \end{align*}

    即
    \[\lim\limits_{n \to \infty}{\dfrac{a_n}{n}} = d\]

\end{proof}

\begin{corollary}[Cauthy命题推论]

    设$\{a_n\}$是正数列,且收敛于$A$,
    则
    \[\lim\limits_{n \to \infty}{\sqrt[n]{a_1a_2\cdots a_n}} = A\]

\end{corollary}

\begin{proof}

    显然有
    \[\ln{\sqrt[n]{a_1 a_2 \cdots a_n}} = \dfrac{\ln{a_1} + \ln{a_2} + \cdots + \ln{a_n}}{n}\]
    由\textup{Cauthy}命题知,因为$\{a_n\}$收敛于$A$,即数列$\{\ln{a_n}\}$收敛于$\ln{A}$. \\    
    所以
    \[\lim\limits_{n \to \infty}{\dfrac{\ln{a_1} + \ln{a_2} + \cdots + \ln{a_n}}{n}} = \ln{A}\]
    也即
    \[\lim\limits_{n \to \infty}{\sqrt[n]{a_1 a_2 \cdots a_n}} = A\]

\end{proof}


\begin{corollary}[Cauthy命题推论]

    设$\{a_n\}$是正数列,且存在极限$\lim\limits_{n \to \infty}{\dfrac{a_{n + 1}}{a_n}} = l$,
    则
    \[\lim\limits_{n \to \infty}{\sqrt[n]{a_n}} = l\]

\end{corollary}

\begin{proof}

    因为$\lim\limits_{n \to \infty}{\dfrac{a_{n + 1}}{a_n}} = l$,所以数列$\left\{ \ln{\dfrac{a_n}{a_{n - 1}}} \right\}$收敛于$\ln{l}$ \\
    由\textup{Cauthy}命题知
    
    \begin{align*}
        & \lim\limits_{n \to \infty}{\dfrac{\ln{\dfrac{a_2}{a_1}} + \ln{\dfrac{a_3}{a_2}} + \cdots + \ln{\dfrac{a_{n + 1}}{a_n}}}{n}} \\
        = & \ \lim\limits_{n \to \infty}{\dfrac{\ln{\dfrac{a_{n + 1}}{a_1}}}{n}} \\
        = & \ \lim\limits_{n \to \infty}{\dfrac{\ln{a_{n + 1} - \ln{a_1}}}{n}} \\
        = & \ \lim\limits_{n \to \infty}{\dfrac{n + 1}{n}\dfrac{\ln{a_{n + 1}}}{n + 1}} - \lim\limits_{n \to \infty}{\dfrac{\ln{a_1}}{n}} = \ln{l}
    \end{align*}

    即
    \[\lim\limits_{n \to \infty}{\dfrac{\ln{a_n}}{n}} = \ln{l}\]
    也即
    \[\lim\limits_{n \to \infty}{\sqrt[n]{a_n}} = l\]

\end{proof}

\begin{proposition}[Cauthy命题$\infty$时的特例]

    $\lim\limits_{n \to \infty}{x_n} = +\infty$. 证明:
    \[\lim\limits_{n \to \infty}{\dfrac{x_1 + x_2 + \cdots + x_n}{n}} = +\infty\]

\end{proposition}

\begin{proof} 

    $\forall A > 0$,由于$\lim\limits_{n \to \infty}{a_n} = +\infty$,故$\exists N_1 \in \mathbb{N}$,$n > N$时,$a_n > 2A + 2$. \\
    又
    \[\lim\limits_{n \to \infty}{\dfrac{n - N_1}{n}} = 1 > \dfrac{1}{2}\]
    \[\lim\limits_{n \to \infty}{\dfrac{x_1 + x_2 + \cdots + x_n}{n}} = 0 > -1\]
    于是

    \begin{align*}
        \dfrac{x_1 + x_2 + \cdots + x_n}{n} & = \dfrac{x_1 + x_2 + \cdots + x_{N_1}}{n} + \dfrac{x_{N_1 + 1} + x_{N_1 + 2} + \cdots + x_n}{n} \\
        & > -1 + \dfrac{n - N_1}{n}(2A + 2) \\
        & > -1 + \dfrac{1}{2}(2A + 2) \\
        & = A 
    \end{align*}

    故
    \[\lim\limits_{n \to \infty}{\dfrac{x_1 + x_2 + \cdots + x_n}{n}} = +\infty\]
    
\end{proof} 

\begin{proposition}[$\dfrac{0}{0}$型的stolz公式]

    设$\{a_n\}$和$\{b_n\}$都是收敛于$0$的数列,其中$\{a_n\}$是严格单调递减数列,且存在
    \[\lim\limits_{n \to \infty}{\dfrac{b_{n + 1} -  b_n}{a_{n + 1} - a_n}} = l \]
    其中$l$为有限值或无穷。则有
    \[\lim\limits_{n \to \infty}{\dfrac{b_n}{a_n}} = l\]

\end{proposition}

\begin{proof}
    
    $\forall \varepsilon > 0$,$\exists N \in \mathbb{N}$,$n > N$时
    \[\left| \dfrac{b_n - b_{n + 1}}{a_n - a_{n + 1}} - l \right| < \varepsilon\]
    \[(l - \varepsilon)(a_n - a_{n + 1}) < b_n - b_{n + 1} < (l + \varepsilon)(a_n - a_{n + 1})\]
    任取$m > n$,将上式中$n$替换为$n + 1$,$n + 2$,$n + 3$,$\cdots$,$m - 1$,有

    \begin{align*}
        &(l - \varepsilon)(a_{n + 1} - a_{n + 2}) < b_{n + 1} - b_{n + 2} < (l + \varepsilon)(a_{n + 1} - a_{n + 2}) \\
        &(l - \varepsilon)(a_{n + 2} - a_{n + 3}) < b_{n + 2} - b_{n + 3} < (l + \varepsilon)(a_{n + 2} - a_{n + 3}) \\
        & \vdots \\
        &(l - \varepsilon)(a_{m - 1} - a_{m}) < b_{m - 1} - b_{m} < (l + \varepsilon)(a_{m - 1} - a_{m})
    \end{align*}

    将上式相加得到
    \[(l - \varepsilon)(a_{n} - a_{m}) < b_{n} - b_{m} < (l + \varepsilon)(a_{n} - a_{m})\]
    \[\left| \dfrac{b_n - b_{m}}{a_n - a_{m}} - l \right| < \varepsilon\]
    对于$m$,令$m \to +\infty$,有
    \[\lim\limits_{m \to \infty}{a_m} = \lim\limits_{m \to \infty}{b_m} =  0\]
    又有
    \[\lim\limits_{m \to \infty}{\left| \dfrac{b_n - b_{m}}{a_n - a_{m}} - l \right|} = \left| \dfrac{b_n}{a_n} - l \right|\]
    则
    \[\lim\limits_{n \to \infty}{\dfrac{b_n}{a_n}} = l\]

\end{proof}

\begin{proposition}[$\dfrac{*}{\infty}$型的stolz公式]
    
    设有数列$\{a_n\}$和$\{b_n\}$,其中$\{a_n\}$是严格单调增加的发散数列,且存在
    \[\lim\limits_{n \to \infty}{\dfrac{b_{n + 1} -  b_n}{a_{n + 1} - a_n}} = l \]
    其中$l$为有限值或无穷。则有
    \[\lim\limits_{n \to \infty}{\dfrac{b_n}{a_n}} = l\]

\end{proposition}

\begin{proof}

    \begin{enumerate}

        \item
            $l$为有限值时。\\
            $\forall \varepsilon > 0$,$\exists N \in \mathbb{N}$,$n > N$时
            \[\left| \dfrac{b_n - b_{n + 1}}{a_n - a_{n + 1}} - l \right| < \varepsilon\]
            \[(l - \varepsilon)(a_n - a_{n + 1}) < b_n - b_{n + 1} < (l + \varepsilon)(a_n - a_{n + 1})\]
            取定$n$,将上式中$n$替换为$N$,$N + 1$,$N + 2$,$\cdots$,$n - 1$,有

            \begin{align*}
                & (l - \varepsilon)(a_{N} - a_{N + 1}) < b_{N} - b_{N + 1} < (l + \varepsilon)(a_{N} - a_{N + 1}) \\
                & (l - \varepsilon)(a_{N + 1} - a_{N + 2}) < b_{N + 1} - b_{N + 2} < (l + \varepsilon)(a_{N + 1} - a_{N + 2}) \\
                & \vdots \\
                & (l - \varepsilon)(a_{n - 1} - a_{n}) < b_{n} - b_{n} < (l + \varepsilon)(a_{n - 1} - a_{n})
            \end{align*}

            将上式相加得到
            \[(l - \varepsilon)(a_{n} - a_{N}) < b_{n} - b_{N} < (l + \varepsilon)(a_{n} - a_{N})\]
            \[\left| \dfrac{b_n - b_{N}}{a_n - a_{N}} - l \right| < \varepsilon\]
            \[\dfrac{b_n}{a_n} - l = \left( 1 - \dfrac{a_N}{a_n} \right)\left( \dfrac{b_n - b_N}{a_n - a_N} - l \right) + \dfrac{b_N - la_N}{a_N}\]
            由于$\lim\limits_{n \to \infty}{a_n} = +\infty$,$\exists N_1 \in \mathbb{N}$,使得当$n > N_1$时成立
            \[0 < 1 - \dfrac{a_N}{a_n} < 2\]
            \[\left| \dfrac{b_N - la_N}{a_n} \right| < \varepsilon\]
            则当$n > \max\{N_1, N\}$时有
            \[\left| \dfrac{b_n}{a_n} - l \right| < 3\varepsilon\]

        \item
            $l$为$+\infty$时。\\
            $\forall \varepsilon > 0$,$\exists N \in \mathbb{N}$,$n > N$时
            \[b_{n + 1} - b_n > a_{n + 1} - a_n\]
            由\textup{(1)}可知
            \[b_{n + 1} - b_N > a_{n + 1} - a_n\]
            则$\lim\limits_{n \to \infty}{b_n} = +\infty$,且$\{b_n\}$严格单调增加。\\
            所以
            \[\lim\limits_{n \to \infty}{\dfrac{b_n}{a_n}} = \lim\limits_{n \to \infty}{\dfrac{a_n - a_{n - 1}}{b_n - b_{n - 1}}} = 0\]
            故
            \[\lim\limits_{n \to \infty}{\dfrac{b_n}{a_n}} = \lim\limits_{n \to \infty}{\dfrac{\frac{1}{a_n}}{b_n}} = +\infty\]

        \item $l$为$-\infty$时同\textup{(2)}.

    \end{enumerate}

\end{proof}

\begin{proposition}[Carleman不等式]

    设$\sum\limits_{n = 1}^{\infty}{a_n}$为收敛的正项级数,
    则不等式
    \[\sum\limits_{n = 1}^{\infty}{\sqrt[n]{a_1 a_2 \cdots a_n}} \leq \euler \sum\limits_{n = 1}^{\infty}{a_n}\]

\end{proposition}

\begin{proof}

    \begin{align*}
        \sum\limits_{n = 1}^{\infty}{\sqrt[n]{a_1 a_2 \cdots a_n}} & = \sum\limits_{n = 1}^{\infty}{\dfrac{\sqrt[n]{a_1 (2a_2) \cdots (na_n)}}{\sqrt[n]{n!}}} \\
        & \leq \sum\limits_{n = 1}^{\infty}{\dfrac{a_1 + 2a_2 + \cdots + na_n}{n\sqrt[n]{n!}}} \\
        & \leq \euler \sum\limits_{n = 1}^{\infty}{\dfrac{a_1 + 2a_2 + \cdots + na_n}{n(n + 1)}} \\
        & = \euler \sum\limits_{n = 1}^{\infty}{\left[ \dfrac{1}{n(n + 1)} \sum\limits_{k = 1}^{n}{ka_k} \right]} \\
        & = \euler \sum\limits_{k = 1}^{N}{\left\{ ka_k\left[ \sum\limits_{n = k}^{N}{\dfrac{1}{n(n + 1)}} \right] \right\}} \\
        & = \euler \sum\limits_{k = 1}^{N}{\left[ ka_k \left( \dfrac{1}{k} - \dfrac{1}{N + 1} \right) \right]} \\
        & \leq \euler \sum\limits_{k = 1}^{N}{a_k}
    \end{align*}

    令$N\to+\infty$,即可得到\textup{Carleman}不等式。\\
    下证不等式右边的系数$\euler$不能在改进。\\
    对于每个$N$构造一个数列
    $$b_n = \left\{
        \begin{aligned}
            & \dfrac{1}{n}, & n & = 1, 2, \cdots, N \\
            & 0, & n & > N
        \end{aligned}
    \right.
    $$
    然后作出两个级数和之比,令$n\to+\infty$,应用\textup{stolz}定理得
    \[\lim\limits_{N \to \infty}{\dfrac{\sum\limits_{n = 1}^{N}{\sqrt[n]{a_1 a_2 \cdots a_n}}}{\sum\limits_{n = 1}^{N}{b_n}}} = \lim\limits_{N \to \infty}{\dfrac{N}{\sqrt[N]{N!}}} = \euler\]

\end{proof}

\begin{theorem}[Sapagof判别法]

设正数列$\{a_n\}$单调减少,则$\lim\limits_{n \to \infty}{a_n} =  0$的充分必要条件是正项级数$\sum\limits_{n = 1}^{\infty}{\left( 1 - \dfrac{a_{n + 1}}{a_n} \right)}$发散。
    
\end{theorem}

\begin{proof}
    
    先证充分性。\\
    正数列$\{a_n\}$单调减少,则$\{a_n\}$存在极限,即
    \[\lim\limits_{n \to \infty}{a_n} = a(a > 0)\]
    设
    \[b_n = 1 - \dfrac{a_{n + 1}}{a_n} = \dfrac{a_n - a_{n + 1}}{a_n} \leq \dfrac{a_n - a_{n + 1}}{a}\]
    则

    \begin{align*}
        \sum\limits_{n = 1}^{m}{b_n} & = \sum\limits_{n = 1}^{m}{(1 - \dfrac{a_{n + 1}}{a_{n}})} \\
        & \leq \dfrac{1}{a} \sum\limits_{n = 1}^{m}{(a_n - a_{n + 1})} \\
        & = \dfrac{1}{a}(a_1 - a_{m + 1})
    \end{align*}

    \begin{align*}
        \sum\limits_{n = 1}^{\infty}{b_n} & = \lim\limits_{m \to \infty}{\sum\limits_{n = 1}^{m}{b_n}} \\
        & \leq \lim\limits_{m \to \infty}{\dfrac{1}{a}(a_1 - a_{m + 1})} \\
        & = \dfrac{1}{a}(a_1 - a)
    \end{align*}

    可知$\sum\limits_{n = 1}^{\infty}{b_n}$收敛。\\
    再证必要性。若$a = 0$,则由柯西准则得

    \begin{align*}
        \sum\limits_{k = n + 1}^{n + p}{b_k} & = \sum\limits_{k = n + 1}^{n + p}{(1 - \dfrac{a_{k + 1}}{a_k})} \\
        & \geq \sum\limits_{k = n + 1}^{n + p}{\dfrac{a_k - a_{k + 1}}{a_k}} \\
        & = \dfrac{1}{a_{n + 1}}(a_{n + 1} - a_{n + p + 1}) \\
        & = \dfrac{a_{n + p + 1}}{a_{n + 1}}
    \end{align*}

    因为$\lim\limits_{n \to \infty}{a_n} =  0$,则总存在$p \in \mathbb{N}$,
    使得
    \[\sum\limits_{k = n + 1}^{n + p}{b_k} \geq \dfrac{1}{2}\]
    所以$\sum\limits_{n = 1}^{\infty}{b_n}$发散。

\end{proof}

\begin{theorem}[Sapagof判别法等价形式\romannum{1}]

    设$\{a_n\}$是单调增加的正数列,则该数列与级数
    \[\sum\limits_{n = 1}^{\infty}{\left( 1 - \dfrac{a_{n + 1}}{a_n} \right)}\]
    同敛散。
        
\end{theorem}

\begin{proof}

    $\{a_n\}$单调增加,若$\{a_n\}$收敛,则
    \[\lim\limits_{n \to \infty}{a_n} = a \ (a \neq +\infty)\]

    \begin{align*}
        \sum\limits_{k = n + 1}^{n + p}{b_k} & = \sum\limits_{k = n + 1}^{n + p}{\left( 1 - \dfrac{a_k}{a_{k + 1}} \right)} \\
        & \leq \sum\limits_{k = n + 1}^{n + p}{\dfrac{a_{k + 1} - a_k}{a_{n + 1}}} \\
        & = \dfrac{a_{n + p + 1} - a_{n + 1}}{a_{n + 1}}
    \end{align*}

    因为$\lim\limits_{n \to \infty}{a_n} = a$,则$\forall \varepsilon > 0$,$\exists N \in \mathbb{N}$,$n > N$时,$|a_n - a| < \varepsilon$. \\
    所以

    \begin{align*}
        \left| \dfrac{a_{n + p + 1} - a_{n + 1}}{a_{n + 1}} \right| & = \left| \dfrac{a_{n + p + 1} - a + a - a_{n + 1}}{a_{n + 1}} \right| \\
        & = \dfrac{|a_{n + p + 1} - a| + |a_{n + 1} - a|}{a_{n + 1}} \\
        & = \dfrac{2\varepsilon}{a_{n + 1}}
    \end{align*} 

    所以$\sum\limits_{n = 1}^{\infty}{b_n}$收敛。\\
    若$\lim\limits_{n \to \infty}{a_n} = +\infty$,则

    \begin{align*}
        \sum\limits_{k = n + 1}^{n + p}{b_k} & = \sum\limits_{k = n + 1}^{n + p}{\left( 1 - \dfrac{a_k}{a_{k + 1}} \right)} \\
        & \leq \sum\limits_{k = n + 1}^{n + p}{\dfrac{a_{k + 1} - a_k}{a_{n + p + 1}}} \\
        & = \dfrac{a_{n + p + 1} - a_{n + 1}}{a_{n + p + 1}} \\
        & = 1 - \dfrac{a_{n + 1}}{a_{n + p + 1}}
    \end{align*}

    由柯西准则,因为$\lim\limits_{n \to \infty}{a_n} = +\infty$,则存在$p \in \mathbb{N}$,
    \[ \dfrac{a_{n + 1}}{a_{n + p + 1}} \leq \dfrac{1}{2}\]
    则
    \[ 1 - \dfrac{a_{n + 1}}{a_{n + p + 1}} \geq \dfrac{1}{2}\]
    所以$\sum\limits_{n = 1}^{\infty}{b_n}$发散。

\end{proof}

\begin{theorem}
    
    $\{S_n\}$是正项级数$\sum\limits_{n = 1}^{\infty}{a_n}$的部分和数列,则
    $\sum\limits_{n = 1}^{\infty}{a_n}$和$\sum\limits_{n = 1}^{\infty}{\dfrac{a_n}{S_n}}$同敛散。

\end{theorem}

\begin{proof}
    
    $\sum\limits_{n = 1}^{\infty}{a_n}$收敛,则

    \begin{align*}
        \sum\limits_{k = n + 1}^{n + p}{\dfrac{a_k}{S_k}} & \leq \sum\limits_{k = n + 1}^{n + p}{\dfrac{a_k}{S_{n + 1}}} \\
        & \leq \dfrac{S_{n + p + 1} - S_{n}}{S_{n + 1}} \\
        & \leq \dfrac{S_{n + p + 1} - S_{n}}{S_{n}}
    \end{align*}

    设$\sum\limits_{n = 1}^{\infty}{a_n} = S$,则$\forall \varepsilon > 0$,$\exists N\in\mathbb{N}$,$n > N$时,$|S_n - S| < \varepsilon$. \\
    则有

    \begin{align*}
        \left| \dfrac{S_{n + p + 1} - S_{n}}{S_{n}} \right| & = \left| \dfrac{S_{n + p + 1} - S + S - S_{n}}{S_{n}} \right| \\
        & \leq \dfrac{|S_{n + p + 1} - S| + |S_{n + 1} - S|}{S_n} \\
        & = \dfrac{2\varepsilon}{S_n}
    \end{align*}

    所以$\sum\limits_{n = 1}^{\infty}{\dfrac{a_n}{S_n}}$收敛。\\
    $\sum\limits_{n = 1}^{\infty}{a_n}$发散,则$\lim\limits_{n \to \infty}{\sum\limits_{k  = 1}^{n}{a_k}} = +\infty$,且有

    \begin{align*}
        \sum\limits_{k = n + 1}^{n + p}{\dfrac{a_k}{S_k}} & \geq \sum\limits_{k = n + 1}^{n + p}{\dfrac{a_k}{S_{n + p + 1}}} \\
        & \geq \dfrac{S_{n + p + 1} - S_{n}}{S_{n + p + 1}} \\
        & = 1 - \dfrac{S_{n}}{S_{n + p + 1}}
    \end{align*}

    因为$\lim\limits_{n \to \infty}{\sum\limits_{k  = 1}^{n}{a_k}} = +\infty = \lim\limits_{n \to \infty}{S_n} = +\infty$,则存在$p \in \mathbb{N}$,
    \[1 - \dfrac{S_{n}}{S_{n + p + 1}} \geq \dfrac{1}{2}\]
    所以$\sum\limits_{n = 1}^{\infty}{\dfrac{a_n}{S_n}}$发散。

\end{proof}

\begin{theorem}
    
    \begin{enumerate}

        \item 
            正项级数$\sum\limits_{n = 1}^{\infty}{a_n}$收敛的充分必要条件时存在正数列$\{b_n\}$和正数$\delta$,使得当$n$充分大时有
            \[b_n \cdot \dfrac{a_n}{a_{n + 1}} - b_{n + 1} \geq \delta > 0\]

        \item 
            正项级数$\sum\limits_{n = 1}^{\infty}{a_n}$发散的充分必要条件时存在发散的正项级数$\sum\limits_{n = 1}^{\infty}{\dfrac{1}{b_n}}$,使得当$n$充分大时有
            \[b_n  \cdot \dfrac{a_n}{a_{n + 1}} - b_{n + 1} \leq 0\]

    \end{enumerate}

\end{theorem}

\begin{proof}
    
    \begin{enumerate}

        \item   
            先证充分性。\\
            由$b_n \cdot \dfrac{a_n}{a_{n + 1}} - b_{n + 1} \geq \delta > 0$得
            \[a_nb_n - a_{n + 1}b_{n + 1} \geq \delta a_{n + 1}\]
            \[a_{n + 1} \leq \dfrac{1}{\delta}(a_nb_n - a_{n + 1}b_{n + 1})\]
            不妨设$n \geq 1$时上式成立,则

            \begin{align*}
                \sum\limits_{k  = 1}^{n}{a_k} & = a_1 + \sum\limits_{k = 2}^{n}{a_k} \\
                & \leq a_1 + \dfrac{1}{\delta} \sum\limits_{k  = 1}^{n - 1}{(a_k b_k - a_{k + 1}b_{k + 1})} \\
                & = a_1 + \dfrac{1}{\delta}(a_1 b_1 - a_{n}b_{n}) \\
                & = a_1 + \dfrac{1}{\delta}a_1 b_1
            \end{align*}

            所以$\sum\limits_{n = 1}^{\infty}{a_n}$收敛。
            再证必要性。\\
            令
            \[b_n = \dfrac{R_n}{a_n}\]
            其中$R_n$为$\sum\limits_{n = 1}^{\infty}{a_n}$的余项,即
            \[R_n = S - S_n\]
            \[b_n \cdot \dfrac{a_n}{a_{n + 1}} - b_{n + 1} = \dfrac{R_n}{a_{n + 1}} - \dfrac{R_{n + 1}}{a_{n + 1}} = 1 > 0\]
            必要性得证。

        \item   
            充分性即为比较判别法的比值形式。\\
            下证必要性。\\
            令
            \[b_n = \dfrac{S_n}{a_n}\]
            易知$\sum\limits_{n = 1}^{\infty}{\dfrac{1}{b_n}}$发散
            \[b_n \cdot \dfrac{a_n}{a_{n + 1}} - b_{n + 1} = \dfrac{S_{n} - S_{n + 1}}{a_{n + 1}} = -1 \leq 0\]
            必要性得证。

\end{enumerate}

\end{proof}

\begin{proposition}
    
    $\lim\limits_{n \to \infty}{(a_{n + p} - a_n)} = \lambda_1$,$p$为固定的正整数,$\lambda$是常数,则
    \[\lim\limits_{n \to \infty}{\dfrac{a_n}{\lambda}} = \dfrac{\lambda}{p}\]

\end{proposition}

\begin{proof}
    
    对固定的$i \in \mathbb{N}$,$\{a_{np + i}\}$是$\{a_{n}\}$的子列 \\
    由$\lim\limits_{n \to \infty}{a_{n + p} - a_n} = \lambda$,则
    \[\lim\limits_{n \to \infty}{a_{(n + 1)p + i} - a_{np + i}} = \lambda\]
    令$A^{(i)}_{n} = a_{(n + 1)p + i} - a_{np + i}$,由\textup{Cauthy}命题知
    \[\lim\limits_{n \to \infty}{\dfrac{A_1 + A_2 + \cdot + A_n}{n}} = \lambda\]
    即
    \[\lim\limits_{n \to \infty}{\dfrac{a_{(n + 1)p + i} - a_{p + i}}{n}} = \lim\limits_{n \to \infty}{\dfrac{a_{(n + 1)p + i}}{n}} = \lambda\]
    则
    \[\lim\limits_{n \to \infty}{\dfrac{a_n}{n}} = \lim\limits_{n \to \infty}{\dfrac{a_{np + i}}{np + i}} = \dfrac{n}{np + i} \lim\limits_{n \to \infty}{\dfrac{a_{np + i}}{n}} = \dfrac{\lambda}{p}\]

\end{proof}

\begin{proposition}

    将二项式系数$\binom{n}{0}$,$\binom{n}{1}$,$\cdots$,$\binom{n}{n}$的算术平均数和几何平均数分别记作$A_n$和$G_n$. 证明:

    \begin{enumerate}

        \item \[\lim\limits_{n \to \infty}{\sqrt[n]{A_n}} = 2\]
                
        \item \[\lim\limits_{n \to \infty}{\sqrt[n]{G_n}} = \euler^{\frac{1}{2}}\]

    \end{enumerate}

\end{proposition}

\begin{proof}

    \begin{enumerate}
        
        \item 
            因为\[A_n = \sum\limits_{k = 0}^{n}{\binom{n}{k}} = 2^n\]则
            \[\lim\limits_{n \to \infty}{\sqrt[n]{A_n}} = \lim\limits_{n \to \infty}{\sqrt[n]{\dfrac{2^n}{n}}} = 2\]

        \item 
            因为\[G_n = \sqrt[n]{\binom{n}{0}\binom{n}{1}\cdots\binom{n}{n}}\]则
            \[\lim\limits_{n \to \infty}{\sqrt[n]{G_n}}  = \lim\limits_{n \to \infty}{\sqrt[n^2]{\binom{n}{0} \binom{n}{1} \cdots \binom{n}{n}}} = \lim\limits_{n \to \infty}{\dfrac{1}{n^2} \sum\limits_{k = 0}^{n}{\ln{\binom{n}{k}}}} = \euler^{\frac{1}{2}}  \]
    
        \end{enumerate}
    
\end{proof}

\begin{proposition}
    
    设
    \[x_n = \dfrac{1}{n^2} \sum\limits_{k = 0}^{n}{\ln{\binom{n}{k}}}, \quad n \in \mathbb{N}\]
    求$\lim\limits_{n \to \infty}{x_n}$.

\end{proposition}

\begin{proof}
    
    应用\textup{stolz}公式,知

    \begin{align*}
        \lim\limits_{n \to \infty}{x_n}  = & \lim\limits_{n \to \infty}{\dfrac{\sum\limits_{k = 0}^{n}{\ln{\binom{n + 1}{k}}} - \sum\limits_{k = 0}^{n}{\ln{\binom{n}{k}}}}{(n + 1)^2 - n^2}} \\
        = & \lim\limits_{n \to \infty}{\dfrac{\sum\limits_{k = 0}^{n}{\ln{\dfrac{n + 1}{n + 1 - k}}}}{2n + 1}} \\
        = & \lim\limits_{n \to \infty}{\dfrac{\ln{(1 + \frac{1}{n})^{n - 1}}}{2}} \\
        = & \ \dfrac{1}{2}
    \end{align*}

\end{proof}

\begin{proposition}
    
    设数列$\{a_n\}$满足$\lim\limits_{n \to \infty}{a_n \sum\limits_{k  = 1}^{n}{a_k^2}} = 1$. 证明:
    \[\lim\limits_{n \to \infty}{\sqrt{3na_n}} = 1\]

\end{proposition}

\begin{proof}

    令$S_n = \sum\limits_{k  = 1}^{n}{a_k^2}$,则$S_n$单调增加,且$S_n \to \infty$,否则若$\lim\limits_{n \to \infty}{S_n} = S$,\\
    则$\lim\limits_{n \to \infty}{a_n^2} = 0$,即
    \[\lim\limits_{n \to \infty}{a_n} = 0\]
    所以
    \[\lim\limits_{n \to \infty}{a_nS_n} = \lim\limits_{n \to \infty}{a_nS} = 0\]
    与题设矛盾。故$\lim\limits_{n \to \infty}{S_n} = +\infty$,又
    \[\lim\limits_{n \to \infty}{a_n} = \lim\limits_{n \to \infty}{a_n \cdot S_n \cdot \dfrac{1}{S_n}} = \lim\limits_{n \to \infty}{a_nS_n} \cdot \lim\limits_{n \to \infty}{\dfrac{1}{S_n}} = 0\]

    \begin{align*}
        S_n^3 - S_{n - 1}^3 & = (S_n - S_{n - 1}) (S_n^2 + S_{n - 1}S_n + S_{n - 1}^2) \\
        = & a_n^2 (2S_n^2 - 3a_n^2S_n + a_n^4) \\
        = & 2a_n^2S_n^2 - 3a_n^4S_n + a_n^6 
    \end{align*}
    
    \[\lim\limits_{n \to \infty}{S_n^3 - S_{n - 1}^3} =  \lim\limits_{n \to \infty}{\big( 3(a_nS_n)^2 - 3a_n^3(a_nS_n) + a_n^6 \big)} = 3\]
    \[\lim\limits_{n \to \infty}{\dfrac{1}{3na_n^3}} = \lim\limits_{n \to \infty}{\dfrac{1}{a_n^3S_n^3} \cdot \dfrac{S_n^3}{3n}} = \lim\limits_{n \to \infty}{\dfrac{S_n^3}{3n}} = \lim\limits_{n \to \infty}{\dfrac{S_n^3 - S_{n - 1}^3}{3}} = 1\]
    故
    \[\lim\limits_{n \to \infty}{\sqrt{3na_n}} = 1\]

\end{proof}

\begin{proposition}

    $\lim\limits_{n \to \infty}{a_n} = a $,$\lim\limits_{n \to \infty}{b_n} = b $. 证明:
    \[\lim\limits_{n \to \infty}{\dfrac{a_0b_n + a_1b_{n - 1} + \cdots + a_nb_0}{n}} = ab\]

\end{proposition}

\begin{proof}

    因$\lim\limits_{n \to \infty}{a_n} = a$,$\lim\limits_{n \to \infty}{b_n} = b $,故数列$\{a_n\}$,$\{b_n\}$有界,$\exists M > 0$,
    \[|a_n| < M, \quad |b_n| < M, \quad \forall n \in \mathbb{N}\]
    $\forall \varepsilon > 0 $,$\exists N \in \mathbb{N}$,$n > N$时有
    \[|a_n - a| < \dfrac{\varepsilon}{4M}, \quad |b_n - b| < \dfrac{\varepsilon}{4M}\]
    取自然数$N_1 > \max\left\{ N, \dfrac{2M}{\varepsilon}\left( \sum\limits_{k = 0}^{n}{|a_k - a|} + \sum\limits_{k = 0}^{n}{(|b_k - a|)} + |b| \right) \right\}$ \\
    则当$n > N_1$时有

    \begin{align*}
        & \left| \dfrac{a_0b_n + a_1b_{n - 1} + \cdots + a_nb_0}{n} \right| \\
        = \ & \dfrac{1}{n} \left| \sum\limits_{k = 0}^{n}{b_{n - k}(a_k - a)} + \sum\limits_{k = 0}^{n}{a_k(b_{n - k} - b)} + ab \right| \\
        = \ & \dfrac{M}{N} \left| \sum\limits_{k = 0}^{N_1}{|a_k - a|} + \sum\limits_{k = 0}^{N_1}{|b_{n - k} - b|} + |b| \right| \\
        \ & + \ \dfrac{M}{n} \left| \sum\limits_{k = N_1 + 1}^{n}{|a_k - a|} + \sum\limits_{k = N_1}^{n}{|b_{n - k} - b|} \right| \\
        \leq \ & \dfrac{\varepsilon}{2} + \dfrac{2M}{n}(n - N_1) - \dfrac{\varepsilon}{4M} \\
        < \ & \varepsilon 
    \end{align*}

    故
    \[\lim\limits_{n \to \infty}{\dfrac{a_0b_n + a_1b_{n - 1} + \cdots + a_nb_0}{n}} = ab\]
    
\end{proof}

\begin{proposition}

    设$a_1 = 1$,$a_n = n(a_{n - 1} + 1), \quad n = 2, 3, \cdots $,且
    \[x_n = \prod_{k  = 1}^{n}{\left( 1 + \dfrac{1}{a_n} \right)}\]
    求$\lim\limits_{n \to \infty}{x_n}$

\end{proposition}

\begin{proof}

    由$\dfrac{a_{n - 1} + 1}{a_n} = \dfrac{1}{n}$得

    \begin{align*}
        \prod_{k  = 1}^{n}{\left( 1 + \dfrac{1}{a_n} \right)} = & \left( 1 + \dfrac{1}{a_1} \right) \left( 1 + \dfrac{1}{a_2} \right) \cdots \left( 1 + \dfrac{1}{a_n} \right) \\
        = & \dfrac{a_1 + 1}{a_1} \cdot \dfrac{a_2 + 1}{a_2} \cdots \dfrac{a_n + 1}{a_n} \\
        = & \dfrac{1}{a_1} \cdot \dfrac{a_1 + 1}{a_2} \cdot \dfrac{a_2 + 1}{a_3} \cdots \dfrac{a_{n - 1} + 1}{a_n} \cdot a_{n + 1} \\
        = & \dfrac{1}{1} \cdot \dfrac{1}{2} \cdots \dfrac{1}{n} \\
        = & \dfrac{a_{n + 1}}{n!} 
    \end{align*}

    \begin{align*}
        \dfrac{a_n}{n!} = & \dfrac{n(a_{n - 1} + 1)}{n!} = \dfrac{a_{n - 1}}{(n - 1)!} + \dfrac{1}{(n - 1)!} \\
        = & \dfrac{a_{n - 2}}{(n - 2)!} + \dfrac{1}{(n - 2)!} + \dfrac{1}{(n - 1)!} \\
        & \cdots \\ 
        = & \dfrac{a_1}{1!} + \dfrac{1}{1!} + \dfrac{1}{2!} + \cdots + \dfrac{1}{(n - 1)!} \\
        = & \sum\limits_{k  = 1}^{n - 1}{\dfrac{1}{k!}} 
    \end{align*}

    则
    \[\dfrac{a_n}{n!} = \sum\limits_{k  = 1}^{n}{\dfrac{1}{k!}}\]
    所以
    \[\lim\limits_{n \to \infty}{x_n} = \lim\limits_{n\infty}{\sum\limits_{k = 0}^{n}{\dfrac{1}{k!}}} = \euler\]

\end{proof}

\begin{proposition}

    设$a_1, a_2 > 0$,$a_{n + 1} = \dfrac{2}{a_n + a_{n - 1}}$. 证明:$\{a_n\}$收敛。

\end{proposition}

\begin{proof}

    选取$\alpha$,使得
    \[\alpha < \min\{a_0, a_1\} \leq \max\{a_0, a_1\} < \dfrac{1}{\alpha}\]
    由数学归纳法可证
    \[\alpha < \min\{a_{n - 1}, a_n\} \leq \max\{a_{n - 1}, a_n\} < \dfrac{1}{\alpha}\]
    即
    \[\alpha < a_n < \dfrac{1}{\alpha}, \quad \forall n \in \mathbb{N}\]
    所以$\{a_n\}$有界。\\
    令
    \[L = \limsup_{n \to \infty}{a_n}, \quad l = \liminf_{n \to \infty}{a_n}\]
    且易知$L, l$均是有限数,则$L \leq \dfrac{1}{\alpha}$,$l \geq \alpha$.

    \begin{align*}
        L & = \limsup_{n \to \infty}{a_n} = \limsup_{n \to \infty}{\dfrac{2}{a_{n - 1} + a_{n - 2}}} \\
         & = \dfrac{1}{\liminf\limits_{n \to \infty}{(a_{n - 1} + a_{n - 2})}} = \dfrac{2}{\liminf\limits_{n \to \infty}{a_{n - 1}} + \liminf\limits_{n \to \infty}{a_{n - 2}}} \\ 
         & = \dfrac{1}{l}
    \end{align*}

    同理$l = \dfrac{1}{L}$,则$lL = 1$. \\
    可设$\{a_{nk + 3}\}$收敛到$L$,$\{a_{nk + 2}\}$,$\{a_{nk + 1}\}$,$\{a_{nk}\}$分别收敛到$a$,$b$,$c$,则$l < a, b, c < L$,又 \\
    $L = \dfrac{2}{a + b} $,则$a + b = 2l $,同理,$b = c = L$ \\
    所以$l = L = 1$,$\{a_n\}$收敛,极限为$1$.

\end{proof}

\begin{proposition}

    证明:数列$x_n = \dfrac{1}{1 + 1} + \dfrac{1}{2 + \frac{1}{2}} + \cdots + \dfrac{1}{n + \frac{1}{n}} - \ln{\dfrac{n}{\sqrt{2}}} \quad (n = 1, 2, \cdots)$ \\
    有位于$\left[ 0, \dfrac{1}{2} \right]$的极限。
    
\end{proposition}

\begin{proof}

    设
    \[f(x) = \dfrac{1}{x + \frac{1}{x}} = \dfrac{x}{x^2 + 1}\]
    则
    \[f'(x) = \dfrac{1 - x^2}{(x^2 + 1)^2} < 1, \quad x \in [1, +\infty)\]
    所以$f(x)$在$[1, +\infty)$上单调减少。

    \begin{align*}
        x_n \geq & \int_{1}^{2}{\dfrac{1}{x + \frac{1}{x}}}\diff x + \int_{2}^{3}{\dfrac{1}{x + \frac{1}{x}}}\diff x + \cdots + \int_{2}^{3}{\dfrac{1}{x + \frac{1}{x}}}\diff x - \ln{\dfrac{n}{\sqrt{2}}} \\
        = & \int_{1}^{n + 1}{\dfrac{x}{x^2 + 1}}\diff x - \ln{\dfrac{n}{\sqrt{2}}} \\
        = & \dfrac{1}{2} \ln{(x^2 + 1)} \Big|_{1}^{n + 1} - \ln{\dfrac{n}{\sqrt{2}}} \\
        = & \dfrac{1}{2} \ln{\dfrac{(n + 1)^2 + 1}{2}} - \ln{\dfrac{n}{\sqrt{2}}} \\
        = & \ln{\sqrt{\dfrac{(n + 1)^2 + 1}{n}}} \\
        \geq & \ln1 = 0 
    \end{align*}

    $\{x_n\}$有下界,又
    \[x_{n + 1} - x_n = \dfrac{1}{n + 1 + \frac{1}{n + 1}} - \ln{\dfrac{1}{n} + 1} < \dfrac{1}{n + 1} + \dfrac{1}{n + 1} = 0\]
    $\{x_n\}$单调减少,则$\{x_n\}$收敛。
    由$x_n \geq 0$,$x_1 = \dfrac{1}{2} $知,$0 \leq x_n \leq \dfrac{1}{2}$,所以
    \[0 \leq \lim\limits_{n \to \infty}{a_n} \leq \dfrac{1}{2}\]

\end{proof}

\begin{proposition}

    设$a_n > 0$,且
    \[\lim\limits_{n \to \infty}{\dfrac{a_n}{a_{n + 1} + a_{n + 2}}} = 0\]
    证明:数列$\{a_n\}$无界。
    
\end{proposition}

\begin{proof}
    
    反证法。假设$\{a_n\}$有界,即$\exists M > 0$,$a_n \leq M$,$\forall n \in \mathbb{N}$,则有
    \[0 < \dfrac{a_n}{2M} \leq \dfrac{a_n}{a_{n + 1} + a_{n + 2}}, \quad \forall n \in \mathbb{N}\]
    两边对$n$取极限,得到
    \[\lim\limits_{n \to \infty}{a_n} = 0\]
    设
    \[A(n) = \dfrac{a_n}{a_{n + 1} + a_{n + 2}}, \quad B(n) = \dfrac{a_{n + 1}}{a_{n + 1} + a_{n + 2}}\]
    显然$B(n) \in (0, 1]$,$ \forall n \in \mathbb{N}$,则
    
    \begin{align*}
        0 & \leq \limsup_{n \to \infty}{\dfrac{a_n}{a_{n + 2} + a_{n + 3}}} \\
        & = \liminf_{n \to \infty}{\left( \dfrac{a_n}{a_{n + 1} + a_{n + 2}} \cdot \dfrac{a_{n + 1} + a_{n + 2}}{a_{n + 2} + a_{n + 3}} \right)} \\
        & \leq \limsup_{n \to \infty}{\dfrac{a_n}{a_{n + 1} + a_{n + 2}}} - \limsup_{n \to \infty}{\dfrac{a_n}{a_{n + 1} + a_{n + 2}}} \\
        & = 0
    \end{align*}
    
    即
    \[\lim\limits_{n \to \infty}{\dfrac{a_n}{a_{n + 2} + a_{n + 3}}} = 0\]
    同理可证,对任意$p \in \mathbb{N}$
    \[\lim\limits_{n \to \infty}{\dfrac{a_n}{a_{n + p} + a_{n + p + 1}}} = 0\]
    则$\forall p \in \mathbb{N}$,$\exists N_p \in \mathbb{N}$,$n > N_p$时
    \[\dfrac{a_n}{a_{n + p} + a_{n + p + 1}} < \dfrac{1}{4}\]
    即$n > N_p$时,$a_n < 2a_{n + p}$和$a_n < 2a_{n + p + 1}$中至少有一个成立。\\
    显然我们可以选取一个子列,使其单调增加且无界。
    
\end{proof}

\begin{proposition}

    证明:
    \[\lim\limits_{n \to \infty}{\left( \dfrac{1 + a_{n + 1}}{a_n} \right)^n} \geq \euler\]
    对所有正数数列$\{a_n\}$成立,且$\euler$不能再改进。

\end{proposition}

\begin{proof}

    设$s_k = \left( \dfrac{1 + a_{k + 1}}{a_k} \right)^k$,则$s_k \geq 0$,且$\sup\limits_{k \geq n}\{s_k\}$对$n$不增。\\
    则$\sup\limits_{k \geq n}\{s_k\}$对$n$极限存在,设
    \[\lim\limits_{n \to \infty}{\sup_{k \geq n}{\{a_k\}}} = l > \euler\]
    则对任意$0 < \varepsilon < l - \euler$,选取$N \in \mathbb{N}$,$n > N$时
    \[l \leq \sup_{k \geq n }{\{s_k\}} < \euler - \varepsilon\]
    我们有
    \[\lim\limits_{n \to \infty}{\left( 1 + \dfrac{1}{n} \right)^n} = \euler\]
    令
    \[r_n = \left( 1 + \dfrac{1}{n} \right)^n\]
    选取$N' \in \mathbb{N}$,$n > N'$时
    \[\euler - \varepsilon < r_n < \euler + \varepsilon\]
    令$M = \max\{N, N'\}$,
    \[s_n \leq \sup_{k \geq n}\{s_k\} < r_n, \quad \forall n > M\]
    进一步有
    \[s_n \leq \sup_{k \geq n}\{a_k\} \leq r_n, \quad \forall n > M\]
    则有
    \[\dfrac{1}{n + 1} < \dfrac{a_n}{n} - \dfrac{a_{n + 1}}{n + 1} < \dfrac{a_{M + 1}}{M + 1}, \quad n > M\]
    \[\sum\limits_{k = M + 1}^{m - 1}{\dfrac{1}{k + 1}} < \dfrac{a_{M + 1}}{M + 1} - \dfrac{a_m}{m} < \dfrac{a_{M + 1}}{M + 1}\]
    令$m \to \infty$,则$\sum\limits_{k = M + 1}^{\infty}{\dfrac{1}{k + 1}}$收敛,而调和级数不收敛。所以
    \[\lim\limits_{n \to \infty}{\left( \dfrac{1 + a_{n + 1}}{a_n} \right)^n} \geq \euler\]

\end{proof}

\begin{proposition}
    
    设
    \[\lim\limits_{n \to \infty}{a_n} = q, \quad |q| < 1\]
    证明:
    \[\lim\limits_{n \to \infty}{(a_n + a_{n - 1}q + \cdots + a_1q^{n - 1})} = \dfrac{a}{1 - q}\]

\end{proposition}

\begin{proof}

    $\forall \varepsilon > 0$,$\exists N_1 \in \mathbb{N}$,$n > N_1$时
    \[|a_n - a| < \dfrac{1 - |q|}{3(1 - q)}\varepsilon\]
    因为$\lim\limits_{n \to \infty}{q^n} = 0$,则存在$N_2 \in \mathbb{N}$,$n > N_2$时
    \[|q|^n < \dfrac{\varepsilon}{3N_1M|1 - q|}, \quad |aq^n| < \dfrac{\varepsilon}{3}\]
    因此当$n > N = N_1 + N_2 + 1$时,有

    \begin{align*}
        & |(1 - q)(a_n + a_{n - 1}q + \cdots + a_1q^{n - 1} - a)| \\
        = \ & \Big|(1 - q)[(a_n - a) + (a_{n - 1} - a)q + \cdots + (a_{N_1 + 1} - a)q^{n - N_1 - 1} + \cdots + (a_1 - a)q^{n - 1} - aq^n] \Big| \\
        < \ & |1 - q| \left[ \dfrac{(1 - |q|)\varepsilon}{3(1 - q)} \cdot \dfrac{1 - |q|^{n - N_1}}{1 - |q|} + N_1M \dfrac{\varepsilon}{3N_1M|1 - q|} \right] + \dfrac{\varepsilon}{3} \\
        < \ & \dfrac{\varepsilon}{3} + \dfrac{\varepsilon}{3} + \dfrac{\varepsilon}{3} \\
        = \ & \varepsilon 
    \end{align*}

    所以
    \[\lim\limits_{n \to \infty}{(1 - q)(a_n + a_{n - 1}q + \cdots + a_1q^{n - 1})} = a\]
    又$1 - q \neq 0$,所以
    \[\lim\limits_{n \to \infty}{(a_n + a_{n - 1}q + \cdots + a_1q^{n - 1})} = \dfrac{a}{1 - q}\]

\end{proof}

\begin{proposition}

    设数列$\{a_n\}$满足
    \[a_n = \sum\limits_{k = 0}^{n}{\dfrac{1}{\binom{n}{k}}}, \quad n = 1, 2, \cdots\]
    证明:
    
    \begin{enumerate}

        \item 
            当$n \geq 2$时,
            \[a_n = \dfrac{n + 1}{2n}a_n + 1\]
        
        \item \[\lim\limits_{n \to \infty}{a_n} = 2\]
        
    \end{enumerate}

\end{proposition}

\begin{proof}

    \begin{enumerate}

        \item 
            由$a_n = \dfrac{n + 1}{2n}a_n + 1$知,
            \[2a_n - a_{n - 1} - 2 = \dfrac{a_{n - 1}}{n}\]
            \[a_n = \sum\limits_{k = 0}^{n - 1}{\dfrac{1}{\binom{n}{k}}} + 1\]
            所以

            \begin{align*}
                2a_n - a_{n - 1} - 2 & = \sum\limits_{k = 0}^{n}{\dfrac{1}{\binom{n}{k}}} + \sum\limits_{k = 0}^{n}{\binom{n}{k + 1}} - \sum\limits_{k = 0}^{n}{\binom{n - 1}{k}} \\
                & = \dfrac{k!(n - k)!}{n!} + \dfrac{(k + 1)!(n - k + 1)!}{n!} - \dfrac{(k - 1)!(n - k - 1)!}{(n - 1)!} \\
                & = \dfrac{k!(n - k - 1)!}{n!} \\
                & = \dfrac{1}{n} a_{n - 1} 
            \end{align*}


        \item 
            \begin{align*}
                2(a_n - a_{n - 1}) & = \dfrac{n + 1}{n}a_{n - 1} - \dfrac{n + 2}{n + 1} a_n \\
                & = a_{n - 1} - a_n + \dfrac{1}{n}a_{n - 1} - \dfrac{1}{n + 1}a_n \\
                & > (a_{n - 1} - a_n) + \dfrac{1}{n + 1}(a_{n - 1} - a_n) \\
                & = \dfrac{n + 2}{n + 1}  (a_{n - 1} - a_n) 
            \end{align*}

            因为$a_1 = 2$, $a_2 = \dfrac{5}{2}$, $a_3 = \dfrac{8}{3}$,所以$n \geq 4$时,$b_n \geq a_{n + 1}$ \\
            即$\{a_n\}$单调减少,且$a_n \geq 0$,$n \in \mathbb{N}$,所以$\{a_n\}$存在极限
            \[\lim\limits_{n \to \infty}{a_n} = \lim\limits_{n \to \infty}{\dfrac{n + 1}{2n}a_{n - 1} + 1}\]
            得
            \[\lim\limits_{n \to \infty}{a_n} = 2\]

    \end{enumerate}

\end{proof}

\begin{lemma}\label{lemma:decimal}

    设集合
    \[S = \{n\alpha - \lfloor n \alpha \rfloor \big| n \in \mathbb{N}\}\]
    其中$\{n\alpha\} = n\alpha - \lfloor n \alpha \rfloor$,$\alpha \in \mathbb{R} \backslash \mathbb{Q}$,即$\alpha$为无理数。\\
    证明:集合$S$在区间$[0, 1]$上稠密。

\end{lemma}

\begin{proof}

    先证集合$S$为无穷集合。设任意$i, j \in \mathbb{N}$
    \[\{i\alpha\} \neq \{j\alpha\}\]
    否则$\exists i, j \in \mathbb{N}$使得
    \[\{i\alpha\} = i\alpha - \lfloor i\alpha \rfloor = j\alpha - \lfloor j\alpha \rfloor = \{j\alpha\}\]
    则有
    \[\alpha = \dfrac{\lfloor i\alpha \rfloor - \lfloor j\alpha \rfloor}{i - j} \in \mathbb{Q}\]
    与题设矛盾,所以显然集合$S$是无穷集,且$S \subseteq [0, 1]$. \\
    由\textup{Bolzano-Weierstrass}定理知,集合$S$至少存在一个聚点。因此易知,$\forall n \in \mathbb{N}$,$\exists i, j \in \mathbb{N}$使得
    \[0 < \{i\alpha\} - \{j\alpha\} < \dfrac{1}{n}\]
    则存在$M \in \mathbb{N}$使得
    \[M(\{i\alpha\} - \{j\alpha\}) \leq 1 < (M + 1)(\{i\alpha\} - \{j\alpha\})\]
    又因为$\alpha$是无理数,所以不存在$M \in \mathbb{N}$, 使$M(\{i\alpha\} - \{j\alpha\}) = 1$,则
    \[M(\{i\alpha\} - \{j\alpha\}) < 1 < (M + 1)(\{i\alpha\} - \{j\alpha\})\]
    因为$\{i\alpha\} - \{j\alpha\} < \dfrac{1}{n}$,则对任意$m \in \{0, 1, \cdots, n - 1\}$,存在$k \in \{1, 2, \cdots, M\}$
    \[k(\{i\alpha\} - \{j\alpha\}) \in \left[ \dfrac{m}{n}, \dfrac{m + 1}{n} \right]\]

    \begin{align*}
        k (\{ i \alpha \} - \{ j \alpha \}) & = \{ k (\{ i \alpha \} - \{ j \alpha \}) \} \\
        & = \{ k [(i \alpha - \lfloor i \alpha \rfloor) - (j \alpha - \lfloor j \alpha \rfloor)] \} \\
        & = \{ k (i - j) \alpha + k (\lfloor j \alpha \rfloor - \lfloor i \alpha \rfloor) \} \\
        & = \{ k (i - j) \alpha \}
    \end{align*}

    因此
    \[\{ k (i - j) \alpha \} \in \! \left[ \dfrac{m}{n}, \dfrac{m + 1}{n} \right] \cap S\]
    即集合$S$在区间$[0, 1]$上稠密。

\end{proof}

\begin{proposition}

    证明:数列$\{\sin{n}\}$在区间$[-1, 1]$上稠密。

\end{proposition}

\begin{proof}

    $\forall v \in [-1, 1]$,$\exists u \in [0, 2\pi]$
    \[\sin{u} = v\]
    由$y = \sin{x}$的连续性可知,$\forall \varepsilon > 0$,$\exists \delta > 0$,使得当$|x - u| < \delta$时
    \[| \sin{u} - \sin{x} | = | v - \sin{x} | < \varepsilon\]
    由引理\ref{lemma:decimal}知,对任意无理数$\alpha$,集合
    \[\{n\alpha - \lfloor n\alpha \rfloor \big| n \in \mathbb{N}\}\]
    在区间$[0, 1]$上稠密。所以令$\alpha = \dfrac{u}{2\pi}$,$\exists n \in \mathbb{N}$,使得
    \[\left| \left \{\dfrac{n}{2\pi} \right\} - \dfrac{u}{2\pi} \right| < \dfrac{\delta}{2\pi}\]
    其中
    \[\left \{\dfrac{n}{2\pi} \right\} = \dfrac{n}{2\pi} - \left\lfloor \dfrac{n}{2\pi} \right\rfloor\]
    两边乘以$2\pi$,得
    \[\left|  2\pi \left \{\dfrac{n}{2\pi} \right\} - u \right| < \delta\]
    所以
    \[\left| \sin{\left( 2\pi \left \{\dfrac{n}{2\pi} \right\} \right)} - \sin{u} \right| < \varepsilon\]
    又因为
    \[2\pi \left \{\dfrac{n}{2\pi} \right\} = \dfrac{n}{2\pi} - \left\lfloor \dfrac{n}{2\pi} \right\rfloor\]
    则有
    \[\sin{\left( 2\pi \left \{\dfrac{n}{2\pi} \right\} \right)} = \sin{\left[ 2\pi \left( \dfrac{n}{2\pi} - \left\lfloor \dfrac{n}{2\pi} \right\rfloor \right) \right]} = \sin{\left( 2\pi \cdot \dfrac{n}{2\pi} \right)} = \sin{n}\]
    即
    \[|\sin{n} - \sin{u}| = |\sin{n} - v| < \varepsilon\]

\end{proof}

\begin{proposition}

    求数列$a_n = \sqrt[n]{1 + \sqrt[n]{2 + \cdots + \sqrt[n]{n}}}$的极限。

\end{proposition}

\begin{proof}

    由算术几何平均值不等式知

    \begin{align*}
        \sqrt[n]{1 + \sqrt[n]{2 + \cdots + \sqrt[n]{n}}} & = \sqrt[n]{\left( 1 + \sqrt[n]{2 + \cdots + \sqrt[n]{n}} \right) \times 1 \times \cdots \times 1} \\
        & \leq \dfrac{n - 1 + 1 + \sqrt[n]{2 + \cdots + \sqrt[n]{n}}}{n} \\
        & = 1 + \dfrac{1}{n} \sqrt[n]{2 + \cdots + \sqrt[n]{n}} \\
        & \leq 1 + \dfrac{1}{n} \sqrt[n]{2 + 3 + \cdots + n} \\
        & = 1 + \dfrac{1}{n} \sqrt[n]{\dfrac{(n + 2)(n - 1)}{2}}
    \end{align*}

    易知
    \[\lim\limits_{n \to \infty}{\dfrac{1}{n} \sqrt[n]{\dfrac{(n + 2)(n - 1)}{2}}} = 0\]
    又$a_n \leq 0$,则由夹逼定理得
    \[0 \leq \lim\limits_{n \to \infty}{a_n} \leq \lim\limits_{n \to \infty}{\dfrac{1}{n} \sqrt[n]{\dfrac{(n + 2)(n - 1)}{2}}} = 0\]
    即
    \[\lim\limits_{n \to \infty}{a_n} = 0\]
    
\end{proof}

\begin{proposition}
    
    证明:若$p_k > 0$,$k = 1, 2, \cdots$,且$\lim\limits_{n\to \infty}{\dfrac{p_n}{p_1 + p_2 + \cdots + p_n}} = 0$,$\lim\limits_{n \to \infty}{a_n} = a$,则
    \[\lim\limits_{n \to \infty}{\dfrac{p_1 a_n + p_2 a_{n - 1} + \cdots + p_n a_1}{p_1 + p_2 + \cdots + p_n}} = a\]

\end{proposition}

\begin{proof}

    由题设知,$\forall \varepsilon > 0$,$\exists N_1 \in \mathbb{N}$,使得
    \[\left| \dfrac{p_n}{p_1 + p_2 + \cdots + p_n} \right| < \varepsilon\]
    $\exists N_2 \in \mathbb{N}$,使得
    \[|a_n - a| \varepsilon\]
    且$\exists M > 0$,使得
    \[|a_n| < M, \quad n \in \mathbb{N}\]
    则当$n > N = N_1 + N_2$时有

    \begin{align*}
        \left| \dfrac{p_1 a_n + p_2 a_{n - 1} + \cdots + p_n a_1}{p_1 + p_2 + \cdots + p_n} - a \right| & = \left| \dfrac{p_1 (a_n - a) + p_2 (a_{n - 1} - a) + \cdots + p_n (a_1 - a)}{p_1 + p_2 + \cdots + p_n} \right| \\
        & \leq \left| \dfrac{p_1 (a_n - a)}{p_1 + p_2 + \cdots + p_n} \right| + \cdots + \left| \dfrac{p_{n - N_2} (a_{N_2 + 1} - a)}{p_1 + p_2 + \cdots + p_n} \right| \\
        & \quad + \left| \dfrac{p_{n - N_2 + 1} (a_{N_2} - a)}{p_1 + p_2 + \cdots + p_n} \right| + \cdots + \left| \dfrac{p_n (a_1 - a)}{p_1 + p_2 + \cdots + p_n} \right| \\
        & \leq \left| \dfrac{(p_1 + p_2 + \cdots + p_{n - N_2})\varepsilon}{p_1 + p_2 + \cdots + p_n} \right| + 2M N_2 \varepsilon \\
        & \leq (2M N_2 + 1) \varepsilon
    \end{align*}

    即
    \[\lim\limits_{n \to \infty}{\dfrac{p_1 a_n + p_2 a_{n - 1} + \cdots + p_n a_1}{p_1 + p_2 + \cdots + p_n}} = a\]
    
\end{proof}

\begin{proposition}

    计算
    \[\lim\limits_{n \to \infty}{n \left( \int_{0}^{\frac{\pi}{4}}{\tan^n{\left( \dfrac{x}{n} \right)}}\diff x \right)^{\frac{1}{n}}}\]

\end{proposition}

\begin{proof}

    易知
    \[\int_{0}^{\frac{\pi}{4}}{\left( \dfrac{x}{n} \right)^n}\diff x \leq \int_{0}^{\frac{\pi}{4}}{\tan^n{\left( \dfrac{x}{n} \right)}}\diff x \leq \dfrac{\pi}{4} \tan^n {\left( \dfrac{\pi}{4n} \right)}\]
    因为
    \[\lim\limits_{n \to \infty}{n \left( \int_{0}^{\frac{\pi}{4}}{\left( \dfrac{x}{n} \right)^n}\diff x \right)^{\frac{1}{n}}} = \lim\limits_{n \to \infty}{\dfrac{\pi}{4} \sqrt{\dfrac{\pi}{n + 1}}} = \dfrac{\pi}{4}\]
    又有
    \[\lim\limits_{n \to \infty}{n \sqrt[n]{\dfrac{\pi}{4}\tan^n{\left( \dfrac{\pi}{4n} \right)}}} = \lim\limits_{n \to \infty}{\left( n \tan{\left( \dfrac{\pi}{4n} \right)} \cdot \sqrt[n]{\dfrac{\pi}{4}} \right)} = \dfrac{\pi}{4}\]
    则由夹逼定则得
    \[\lim\limits_{n \to \infty}{n \left( \int_{0}^{\frac{\pi}{4}}{\tan^n{\left( \dfrac{x}{n} \right)}}\diff x \right)^{\frac{1}{n}}} = \dfrac{\pi}{4}\]

\end{proof}

\begin{proposition}

    已知$\lim\limits_{n \to \infty}{x_n} = 0$,用定义证明
    \[\lim\limits_{n \to \infty}{\dfrac{\left( x_n + \frac{x_{n - 1}}{2} + \cdots + \frac{x_1}{n} \right)}{\ln n}} = 0\]

\end{proposition}

\begin{proof}

    设$\max|x_n| = M$,且对任意$\varepsilon ' > 0$,存在一个$N_1(\varepsilon ')$,使得当$N > N_1(\varepsilon ')$时,$|x_k| \leq \varepsilon '$成立。则有
    
    \begin{align*}
        \dfrac{\left| x_n + \frac{x_{n - 1}}{2} + \cdots + \frac{x_1}{n} \right|}{\ln n} & = \dfrac{\left| \sum\limits_{k = 1}^{N_1(\varepsilon')}{\frac{1}{n + 1 - k}x_k} + \sum\limits_{k = N_1(\varepsilon') + 1}^{n}{\frac{1}{n + 1 - k}x_k} \right|}{\ln n} \\
        & \leq \dfrac{\left| \sum\limits_{k = 1}^{N_1(\varepsilon ')}{\frac{1}{n + 1 - k}x_k} \right| + \left| \sum\limits_{k = N_1(\varepsilon') + 1}^{n}{\frac{1}{n + 1 - k}x_k} \right|}{\ln n} \\
        & \leq \dfrac{ \sum\limits_{k = 1}^{N_1(\varepsilon ')}{\frac{1}{n + 1 - k}\left| x_k \right|} + \sum\limits_{k = N_1(\varepsilon ) + 1}^{n}{\frac{1}{n + 1 - k} \left| x_k \right|}}{\ln n} \\
        & \leq \dfrac{M \varepsilon' + \left( \sum\limits_{k = N_1(\varepsilon ') + 1}^{n}{\frac{1}{n + 1 - k}} \right)\varepsilon'}{\ln n} \\
        & < (M + 1) \varepsilon
    \end{align*}

    因此令$\varepsilon = (M + 1)\varepsilon'$,对于任意$\varepsilon >  0$,存在一个$N(\varepsilon) = N_1(\varepsilon') = N_1(\dfrac{\varepsilon}{M + 1})$,使得当$N > N(\varepsilon)$时
    \[\dfrac{\left| x_n + \frac{x_{n - 1}}{2} + \cdots + \frac{x_1}{n} \right|}{\ln n} < \varepsilon\]

\end{proof}

\begin{proposition}
    
    已知正数列$\{x_n\}$满足$x_{n + 2} = \sqrt{x_{n + 1}} + \sqrt{x_{n}}$,$n \geq 1$. 证明:$\{x_n\}$收敛并求出该极限。

\end{proposition}

\begin{proof}

    \begin{enumerate}

        \item 
            如果$0 < x_1 < x_2 < 1$,则有
            \[x_3 = \sqrt{x_1} + \sqrt{x_2} \geq x_2\]
            \[x_{n + 2} - x_{n + 1} = (\sqrt{x_{n + 1}} + \sqrt{x_{n}}) - (\sqrt{x_{n}} + \sqrt{x_{n - 1}}) = (\sqrt{x_{n + 1}} - \sqrt{x_{n}}) + (\sqrt{x_{n}} - \sqrt{x_{n - 1}}) \geq 0\]
            所以$\{x_n\}$单调增加。且由数学归纳法可证,若$x_n \leq 4$,则
            \[x_{n + 2} = \sqrt{x_{n + 1}} + \sqrt{x_{n}} \leq 4\]
            所以$\{x_n\}$收敛,且极限为$4$.

        \item 
            如果$0 < x_2 < x_1 < 1$,则有
            \[x_3 = \sqrt{x_1} + \sqrt{x_2} \geq x_1\]
            \[x_{n + 2} - x_{n + 1} = \sqrt{x_{n + 1}} - \sqrt{x_{n - 1}} \geq 0\]
            同上。

        \item 
            如果$x_1 \geq 1$或$x_2 \geq 1$,则
            \[x_3 = \sqrt{x_1} + \sqrt{x_2} \geq 1 \Longrightarrow x_{n + 2} = \sqrt{x_{n + 1}} + \sqrt{x_{n}} \geq 1\]
            
            \begin{align*}
                |x_{n + 2} - 4| & = |\sqrt{x_{n + 1}} - 2 + \sqrt{x_{n}} - 2| \\
                & \leq |\sqrt{x_{n + 1}} - 2| + |\sqrt{x_{n}} - 2| \\
                & = \left| \dfrac{x_{n + 1} - 4}{\sqrt{x_{n + 1}} + 2} \right| + \left| \dfrac{x_{n} - 4}{\sqrt{x_{n}} + 2} \right| \\
                & \leq \dfrac{1}{3} |x_{n + 1} - 4| + \dfrac{1}{3}|x_n - 4| 
            \end{align*}

            由递推公式易知,
            \[\lim\limits_{n \to \infty}{|x_{n + 2} - 4|} = 0\]

    \end{enumerate}

\end{proof}

\begin{proposition}

    设$x_0 \in \left( 0, \dfrac{\pi}{2} \right]$,定义$x_{n + 1} = \sin{x_n}$,求$\lim\limits_{n \to\infty}{nx_n^2}$的值。

\end{proposition}

\begin{proof}

    因为$nx_n^2 = \dfrac{n}{\frac{1}{x_n^2}}$,由\textup{Stolz}定理知
    \[\lim\limits_{n \to\infty}{nx_n^2} = \lim\limits_{n \to \infty}{\dfrac{n}{\frac{1}{x_n^2}}} = \lim\limits_{n \to \infty}{\dfrac{(n + 1) - n}{\frac{1}{x_{n + 1}^2} - \frac{1}{x_n^2}}} = \lim\limits_{n \to \infty}{\dfrac{1}{\frac{1}{x_{n + 1}^2} - \frac{1}{x_n^2}}}\]
    \[\dfrac{1}{x_{n + 1}^2} - \dfrac{1}{x_n^2} = \dfrac{1}{\sin^2{x_n}} - \dfrac{1}{x_n^2} = \dfrac{x_n^2 - \sin^2{x_n}}{x_n^2 \sin^2{x_n}}\]
    易知$\lim\limits_{n \to \infty}{x_n} = 0$,所以
    \[\lim\limits_{n \to \infty}{\dfrac{x_n^2 - \sin^2{x_n}}{x_n^2 \sin^2{x_n}}} = \lim\limits_{x \to 0}{\dfrac{x^2 - \sin^2{x}}{x^2 \sin^2{x}}} = \dfrac{1}{3}\]
    因此
    \[\lim\limits_{n \to\infty}{nx_n^2} = 3\]
    同时可知
    \[x_n = O\left( \dfrac{1}{\sqrt{n}} \right)\]
    级数$\sum\limits_{n = 1}^{\infty}{x_n^2}$发散。

\end{proof}

\begin{proposition}

    设$a_1 = \sqrt{2}$,$a_{n + 1} = {\sqrt{2}}^{a_n}$. 证明:数列$\{a_n\}$收敛并求出该极限。

\end{proposition}

\begin{proof}

    由数学归纳法易证
    \[a_n \leq 2, \quad n \in \mathbb{N}\]
    设函数$f(x) = {\sqrt{2}}^x - x$,则
    \[f'(x) = \dfrac{1}{2}\ln2 {\sqrt{2}}^x - 1, \quad f''(x) = \left( \dfrac{1}{2}\ln 2 \right)^2 {\sqrt{2}}^x \geq 0\]
    易知$x \leq 2$时,$f'(x) < 0$,且$f(2) = 0$. 因此
    \[f(x) \geq 0, \quad 0 \leq x \leq 2\]
    即$a_{n + 1} \leq a_n$. 所以数列$\{a_n\}$收敛,且极限$a$满足
    \[{\sqrt{2}}^{a} = a\]
    即$a = 2$.

\end{proof}

\begin{proposition}

    设\[S_{n} = \sum\limits_{k = 1}^{n} \left[ \dfrac{1}{4k - 3} + \dfrac{1}{4k - 1} - \dfrac{1}{2k} \right]\]
    证明$S_{n}$收敛并求其值。

\end{proposition}

\begin{proof}

    易知
    
    \begin{align*}
        \lim\limits_{n \to +\infty } S_{n} & =  \lim\limits_{n \to +\infty} \sum\limits_{k = 1}^{n} \left[ \dfrac{1}{4k - 3} - \dfrac{1}{4k - 2} + \dfrac{1}{4k - 1} - \dfrac{1}{4k}  + \dfrac{1}{2} \left[ \dfrac{1}{2k - 1} - \dfrac{1}{2k} \right] \right] \\
        & = \dfrac{3}{2} \lim\limits_{n \to + \infty} \sum\limits_{k = 1}^{n}  \left[  \dfrac{1}{2k - 1} -\dfrac{1}{2k} \right] \\
        & = \dfrac{3}{2} \ln 2
    \end{align*}

\end{proof}

\begin{proposition}

    证明:
    \[\lim\limits_{n \to \infty}{\left( 1 + \int_{0}^{1}{\dfrac{\sin^n{x}}{x^n}}\diff x \right)^n} = + \infty\]

\end{proposition}

\begin{proof}

    注意到,$\sin{x} \geq x - \dfrac{x^3}{6}$,$\forall x \geq 0$,得到

    \begin{align*}
        \left( 1 + \int_{0}^{1}{\dfrac{\sin^n{x}}{x^n}}\diff x \right)^n & \geq \left( 1 + \int_{0}^{1}{\dfrac{\sin^n x}{x^n}}\diff x \right)^n \\
        & \geq \left( 1 + \int_{0}^{\frac{1}{\sqrt{n}}}{\dfrac{\sin^n{x}}{x^n}}\diff x \right)^n \\
        & \geq \left( 1 + \dfrac{5}{6\sqrt{n}} \right)^n \\
        & \geq \dfrac{5\sqrt{n}}{6}
    \end{align*}

    因此有
    \[\lim\limits_{n \to \infty}{\left( 1 + \int_{0}^{1}{\dfrac{\sin^n{x}}{x^n}}\diff x \right)^n} = + \infty\]

\end{proof}

\begin{proposition}

    求$x \in [1, 2)$,使得对任意自然数$n \in \mathbb{N}$,满足$\lfloor 2^n x \rfloor$被$4$除后余$1$或者余$2$.

\end{proposition}

\begin{proof}

    $x$可以表示为二进制形式,$\lfloor 2^n x \rfloor$即为$x$小数点右移$n - 2$位(若$n < 2$时,即为左端补$0$后小数点左移相应位数)后保留整数部分。\\
    $\lfloor 2^n x \rfloor$被$4$除后的余数即为$\lfloor 2^n x \rfloor$二进制表示的最后两位数,因此$x$的二进制形式中,任意连续两位数必为$01$或$10$,则
    \[x = 1.0101 \cdots\]
    即$x = \dfrac{4}{3}$.
    
\end{proof}

\section{不等式}

\begin{theorem}[Bernoulli不等式]

    设$h > -1$,$n \in \mathbb{N}$,则成立不等式
    \[(1 + h)^n \geq 1 + nh\]
    其中$n > 1$时等号成立的充分必要条件时$h = 0$.

\end{theorem}

\begin{proof}
    
    $n = 1$或$h = 0$时不等式显然成立,下面只证明$n > 1$和$h\neq 0$时的情况。\\
    将$(1 + h)^n - 1$作因式分解,可以得到
    \[(1 + h)^n - 1 = h[1 + (1 + h) + (1 + h)^2 + \cdots + (1 + h)^{n - 1}]\]
    当$h > 0$时,右边方括号内第二项起都大于$1$,因此$(1 + h)^n - 1 > nh$. \\
    当$-1 < h < 0$时,右边方括号内第二项起都小于$1$,则$(1 + h)^n - 1 < nh$,又因为$h < 0$,\\
    所以得到$(1 + h)^n - 1 > nh$. \\
    下证等号成立的条件,易知$A > 0$,$A + B > 0$,$n \in \mathbb{N}$时成立不等式
    \[(A + B)^n \geq A^n + nA^{n - 1}B\]
    且$n > 1$时等号成立的充分必要条件是$B = 0$.

\end{proof}

\begin{theorem}[算术平均值-几何平均值不等式]
    
    设$a_1, a_2, \cdots, a_n$是$n$个非负实数,则成立不等式
    \[\dfrac{a_1 + a_2 + \cdots + a_n}{n} \geq \sqrt[n]{a_1, a_2, \cdots, a_n}\]
    其中等号成立的充分必要条件是$a_1 = a_2 = \cdots = a_n$.

\end{theorem}

\begin{proof}
    
    显然,若$a_1, a_2, \cdots, a_n$中有$0$,则不等式成立,且此时等号成立的充分必要条件是$a_1 = a_2 = \cdots = a_n = 0$. \\
    下面只证$a_1, a_2, \cdots, a_n$全为正数时的情况。
    $n = 1$时不等式成立。\\
    假设$n = k$时不等式成立,则$n = k + 1$时有以下分解
    \[\dfrac{a_1 + a_2 + \cdots + a_{k + 1}}{k + 1} = \dfrac{a_1 + a_2 + \cdots + a_{k}}{k} + \dfrac{ka_{k + 1} - (a_1 + a_2 + \cdots + a_k)}{k(k + 1)}\]
    令
    
    \begin{align*}
        A & = \dfrac{a_1 + a_2 + \cdots + a_{k}}{k} \\
        B & = \dfrac{ka_{k + 1} - (a_1 + a_2 + \cdots + a_k)}{k(k + 1)}
    \end{align*}

    \begin{align*}
        & \left( \dfrac{a_1 + a_2 + \cdots + a_{k + 1}}{k + 1} \right)^{k + 1} \\
        = & \  (A + B)^{k + 1} \\
        \geq & \  A^{k + 1} + (k + 1) A^k B \\
        = & \ A^k(A + (k + 1)B) \\
        = & \ A^k a_{k + 1} \\
        \geq & \ a_1 a_2 \cdots a_{k + 1}
    \end{align*}

    不等式等好成立时条件也可以由数学归纳法得到。\\
    $n = 1$时显然成立。\\
    假设$n = k$时成立,则$n = k + 1$时可以由以上推导过程观察得到等号成立的充分必要条件是

    \begin{align*}
        & a_1 = a_2 = \cdots = a_k \\
        & ka_{k + 1} = a_1 + a_2 + \cdots + a_k
    \end{align*}

    也就是
    \[a_1 = a_2 = \cdots = a_k = a_{k + 1}\]
\end{proof}

\begin{theorem}[Young不等式]

    $p, q \in \mathbb{R}^+$,且$\dfrac{1}{p} + \dfrac{1}{q} = 1$,则

    \begin{align*}
        a^{\frac{1}{p}} b^{\frac{1}{q}} & \leq \dfrac{1}{p}a + \dfrac{1}{q}b \quad (p > 1) \\
        a^{\frac{1}{p}} b^{\frac{1}{q}} & \geq \dfrac{1}{p}a + \dfrac{1}{q}b \quad (p < 1)
    \end{align*}

    其中$a, b$为任意实数。

\end{theorem}

\begin{proof}

    设函数
    \[f(x) = x^{\alpha} - \alpha x + \alpha - 1\]
    易证$\alpha < 1 $时,$f(x) \leq 0$;$\alpha \geq 1$时,$f(x) \geq 0$.
    令$x = \dfrac{a}{b}$,则
    
    \begin{align*}
        & \left( \dfrac{a}{b} \right)^{\alpha} - \alpha \left( \dfrac{a}{b} \right) + \alpha - 1 \leq 0 \quad (\alpha < 1) \\
        & \left( \dfrac{a}{b} \right)^{\alpha} - \alpha \left( \dfrac{a}{b} \right) + \alpha - 1 \geq 0 \quad (\alpha \geq 1)
    \end{align*}

    令$\alpha = \dfrac{1}{p}$,有

    \begin{align*}
        a^{\frac{1}{p}} b^{\frac{1}{q}} & \leq \dfrac{1}{p}a + \dfrac{1}{q}b \quad(p > 1) \\
        a^{\frac{1}{p}} b^{\frac{1}{q} }& \geq \dfrac{1}{p}a + \dfrac{1}{q}b \quad(p < 1)
    \end{align*}

\end{proof}

\begin{theorem}[H{\H o}lder不等式]
    
    $p, q \in \mathbb{R}^+$,且$\dfrac{1}{p} + \dfrac{1}{q} = 1$,则
    \[\sum{x_i y_i} \leq \left( \sum{{x_i}^{p}} \right)^{\frac{1}{p}} \left( \sum{{y_i}^{q}} \right)^{\frac{1}{q}}\]
    其中,$\{x_i\}$和$\{y_i\}$均为非负数。

\end{theorem}

\begin{proof}

    令$X = \sum{{x_i}^p}$,$Y = \sum{{y_i}^q}$.
    由\textup{Young}不等式得
    \[\dfrac{x_i y_i}{X^{\frac{1}{p}}Y^{\frac{1}{q}}} = \left( \dfrac{{x_i}^p}{X} \right)^{\frac{1}{p}}\left( \dfrac{{y_i}^q}{Y} \right)^{\frac{1}{q}} \leq \dfrac{1}{p}\dfrac{{x_i}^p}{X} + \dfrac{1}{q}\dfrac{{y_i}^q}{Y}\]
    \[\sum{\dfrac{x_i y_i}{X^{\frac{1}{p}} Y^{\frac{1}{q}}}} \leq \dfrac{1}{p}\dfrac{\sum{{x_i}^p}}{X} + \dfrac{1}{q}\dfrac{\sum{{y_i}^q}}{Y} = \dfrac{1}{p} + \dfrac{1}{q} = 1 \]
    \[\sum{x_i y_i} \leq X^{\frac{1}{p}} Y^{\frac{1}{q}} = (\sum{{x_i}^{p}})^{\frac{1}{p}} (\sum{{y_i}^{q}})^{\frac{1}{q}}\]

\end{proof}

\begin{theorem}[Minkowski不等式]

    设$p \in \mathbb{R}^+$
    \[\left( \sum{(x_i + y_i)}^p \right) ^{\frac{1}{p}} \leq \left( \sum{{x_i}^p} \right)^{\frac{1}{p}} + \left( \sum{{y_i}^{p}} \right)^{\frac{1}{p}}\]
    其中,$\{x_i\}$和$\{y_i\}$均为非负数。

\end{theorem}

\begin{proof}

    \[\sum{(x_i + y_i)}^p = \left( \sum{(x_i + y_i)}^{p - 1} \right) (x_i + y_i) = x_i \sum{(x_i + y_i)}^{p - 1} + y_i \sum{(x_i + y_i)^{p - 1}}\]
    由\textup{H{\H o}lder}不等式得
    
    \begin{align*}
        \sum{(x_i + y_i)}^p & \leq \left( \sum{{x_i}^p} \right)^{\frac{1}{p}} \left( \sum{(x_i + y_i)^{q(p - 1)}} \right)^{\frac{1}{q}} + \left( \sum{{y_i}^q} \right)^{\frac{1}{p}} \left( \sum{(x_i + y_i)^{q(p - 1)}} \right)^{\frac{1}{q}} \\
        & = \left( \sum{(x_i + y_i)}^p \right) ^{\frac{1}{q}} \left( \left( \sum{{x_i}^p} \right)^{\frac{1}{p}}  + \left( \sum{{y_i}^{p}} \right)^{\frac{1}{p}} \right)
    \end{align*}

\end{proof}

\begin{theorem}

    \begin{enumerate}
        
        \item 
            设$F \in C[a, b]$,处处大于$0$,且单调减少,则有
            \[\int_{a}^{b}{F(x)}\diff x \int_{a}^{b}{xF^2(x)}\diff x \leq \int_{a}^{b}{F^2(x)}\diff x \int_{a}^{b}{xF(x)}\diff x\]

        \item 
            设$f$,$g$在区间$[a, b]$上对于任何$x$, $y$有
            \[(f(x) - f(y))(g(x) - g(y)) \geq 0 \]
            又设$p(x)$是区间$[a, b]$上的黎曼可积函数,且处处大于$0$. 则有
            \[\int_{a}^{b}{p(x)f(x)}\diff x \int_{a}^{b}{p(x)g(x)}\diff x \leq \int_{a}^{b}{p(x)}\diff x \int_{a}^{b}{p(x)f(x)g(x)}\diff x\]

    \end{enumerate}

\end{theorem}

\begin{proof}
    
    \begin{enumerate}

        \item
            令两边第二项积分变量替换为$y$,化为二重积分,得
        
            \begin{align*}
                I_1 & = \int_{a}^{b}{\int_{a}^{b}{F(x)yF^2(y) - F^(x)yF(y)}\diff x}\diff y \\
                & = \int_{a}^{b}{\int_{a}^{b}{yF(x)F(y)(F(y) - F(x))}\diff x}\diff y
            \end{align*}

            再令$x$与$y$交换,得
            \[I_2 = \int_{a}^{b}{\int_{a}^{b}{xF(x)F(y)(F(x) - F(y))}\diff x}\diff y\]
            因为积分区域关于$x = y$对称,则
            \[2I = I_1 + I_2 = \int_{a}^{b}{\int_{a}^{b}{F(x)F(y)(F(x) - F(y))}\diff x}\diff y\]
            因为$F(x)$单调减少且处处大于$0$,所以$F(x)F(y)(F(x) - F(y)) \leq 0$,即$I \leq 0$.
            则
            \[ \int_{a}^{b}{F(x)}\diff x\int_{a}^{b}{xF^2(x)}\diff x \leq \int_{a}^{b}{F^2(x)}\diff x \int_{a}^{b}{xF(x)}\diff x \]

        \item 同\textup{(1)},先化为二重积分,再利用题设证明。
        
    \end{enumerate}

\end{proof}

\begin{theorem}[Bellman-Gronwall不等式]

    设当$x \geq 0$时,$f(x)$,$g(x)$为非负连续函数,且有
    \[f(x) \leq A + \int_{0}^{x}{f(t)g(t)}\diff t\]
    其中$A > 0$. \\
    则当$x \geq 0$时
    \[ f(x) \leq A \euler^{\int_{0}^{x}{g(t)}\diff t} \]

\end{theorem}

\begin{proof}

    由题设得

    \begin{align*}
        \dfrac{f(x)}{A + \int_{0}^{x}{f(t)g(t)}\diff t} & \leq 1 \\
        \dfrac{f(x)g(x)}{A + \int_{0}^{x}{f(t)g(t)}\diff t} & \leq g(x) \\
        \int_{0}^{x}{\dfrac{f(x)g(x)}{A + \int_{0}^{x}{f(t)g(t)}\diff t}}\diff x & \leq \ln A + \int_{0}^{x}{g(t)}\diff t
    \end{align*}

    即
    \[f(x) \leq A+\int_{0}^{x}{f(t)g(t)}\diff t \leq A \euler^{\int_{0}^{x}{g(t)}\diff t} \]

\end{proof}

\begin{theorem}
    
    设$f$是区间$[a, b]$上的黎曼可积函数,且$0 < m \leq f(x) \leq M$,则有
    \[\int_{a}^{b}{\dfrac{1}{f(x)}}\diff x \int_{a}^{b}{f(x)}\diff x \leq \dfrac{(M + m)^2}{4Mm}(b - a)^2 \]

\end{theorem}

\begin{proof}
    
    左右两边同乘$Mm$,得
    \[\int_{a}^{b}{\dfrac{Mm}{f(x)}}\diff x \int_{a}^{b}{f(x)}\diff x \leq \dfrac{(M + m)^2}{4}(b - a)^2\]
    由均值不等式得

    \begin{align*}
        \sqrt{\int_{a}^{b}{\dfrac{Mm}{f(x)}}\diff x} \sqrt{\int_{a}^{b}{f(x)}\diff x} & \leq \dfrac{1}{2} \int_{a}^{b}{\left[ \dfrac{Mm}{f(x)} + f(x) \right]}\diff x \\
        & = \dfrac{1}{2} \int_{a}^{b}{\left[ \dfrac{(f(x) - M)(f(x) - m)}{f(x)} + (M + m) \right]}\diff x \\
        & \leq \dfrac{1}{2} \int_{a}^{b}{(M + m)}\diff x \\
        & = \dfrac{1}{2}(b - a)(M + m)
    \end{align*}

    则有
    \[\int_{a}^{b}{\dfrac{Mm}{f(x)}}\diff x \int_{a}^{b}{f(x)}\diff x \leq \dfrac{1}{4} (b - a)^2(M + m)^2\]
    即
    \[ \int_{a}^{b}{\dfrac{1}{f(x)}}\diff x \int_{a}^{b}{f(x)}\diff x \leq \dfrac{(M + m)^2}{4Mm}(b - a)^2 \]

\end{proof}