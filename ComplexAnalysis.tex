\chapter{复分析}

\section{复数}

\begin{proposition}
    
    证明:对于$|z| \leq 1$,$|w| \leq 1$,有
    \[|w - z| \leq |1 - \overline{w}z|\]

\end{proposition}

\begin{proof}

    设$z = x_1 + y_1\imag$,$w = x_2 + y_2\imag$.
    \[|w - z| = | (x_2 - x_1) + (y_2 - y_1)\imag| \]
    \[|1 - \overline{w}z| = |1 - (x_2 - y_2\imag) (x_1 - y_1\imag)| = | (1 - x_1x_2 - y_1y_2) + (x_1y_2 - x_2y_1) |\]

    \begin{align*}
        &|w - z|^2 - |1 - \overline{w}z|^2 \\
        = \ & x_1^2 + x_2^2 + y_1^2 + y_2^2 - 1 - x_1^2x_2^2 -  y_1^2y_2^2 - x_1^2 y_2^2 -  x_2^2y_1^2 \\
        = \ & (1 - x_1^2 - y_1^2) (1 - x_2^2 - y_2^2) \\
        \leq \ & 0
    \end{align*}

    其中
    \[0 \leq x_1^2 + y_1^2 \leq 1, \quad 0 \leq x_2^2 + y_2^2 \leq 1\]
    则
    \[|w - z| \leq |1 - \overline{w}z|\]

\end{proof}

\begin{proposition}

    设数列$\{a_n\}$由$-1$,$0$,$1$组成。证明:
    \[a_0\sqrt{2 + \sqrt{a_1 \sqrt{2 + a_2\sqrt{2 + \cdots}}}} = 2 \sin{\left( \dfrac{\pi}{4} \sum\limits_{n = 0}^{\infty}{\dfrac{a_0 a_1 \cdots a_n}{2^n}} \right)}\]

\end{proposition}

\begin{proof}

    设
    \[z_n = 2 \sin{\left( \dfrac{\pi}{4} \sum\limits_{k = 0}^{n}{\dfrac{a_0 a_1 \cdots a_k}{2^n}} \right)}\]
    易知
    \[\mathrm{sgn}{z_n} = \mathrm{sgn}\left( 2 \sin{\left( \dfrac{\pi}{4} \sum\limits_{k = 0}^{n}{\dfrac{a_0 a_1 \cdots a_k}{2^n}} \right)} \right) = a_0\]
    对$a_0 \neq 0$,有
    \[z_n^2 - 2 = a_1\sqrt{2 + a_2\sqrt{2 + \cdots + a_n\sqrt{2}}}\]
    且有

    \begin{align*}
        z_n^2 - 2 & = -2 \cos{\left( \dfrac{\pi}{2}\sum\limits_{k = 0}^{n}{\dfrac{a_0 a_1 \cdots a_k}{2^k}} \right)} \\
        & = 2 \cos{\left( \dfrac{\pi}{2} + \dfrac{\pi}{2}\sum\limits_{k = 0}^{n}{\dfrac{a_1 \cdots a_k}{2^k}} \right)} \\
        & = 2 \sin{\left( \dfrac{\pi}{4}\sum\limits_{k = 1}^{n}{\dfrac{a_1 \cdots a_k}{2^{k - 1}}} \right)}
    \end{align*}

    得证。

\end{proof}

\section{复积分}

\begin{proposition}

    证明:对任意$ \xi \in \mathbb{C}$,有
    \[\euler^{-\pi\xi^2} = \int_{-\infty}^{+\infty}{\euler^{-\pi x^2}\euler^{-2\pi \imag x\xi}}\diff x\]
    
\end{proposition}

\begin{proof}

    \[\euler^{-\pi x^2}\euler^{-2\pi x\xi} = \euler^{-\pi(x^2 + 2\imag x\xi)}\]
    因为
    \[-\pi(x^2 + 2\imag x\xi) = -\pi (x^2 + 2\pi\imag x - \xi^2 + \xi^2) = -\pi(x + \imag \xi)^2 - \pi\xi^2\]
    则
    \[\int_{-\infty}^{+\infty}{\euler^{-\pi x^2}\euler^{-2\pi \imag x\xi}}\diff x = \euler^{\pi\xi^2} \int_{-\infty}^{+\infty}{\euler^{-\pi (x + \imag\xi)^2}}\diff x\]
    令$z = x + \imag\xi$,$\diff x = \diff z$,则有

    \begin{align*}
        & \euler^{\pi\xi^2} \int_{-\infty}^{+\infty}{\euler^{-\pi (x + \imag\xi)^2}}\diff x \\
        = \ & \euler^{\pi\xi^2} \int_{-\infty}^{+\infty}{\euler^{-\pi z^2}}\diff x \\
        = \ & \euler^{\pi\xi^2} \dfrac{\sqrt{\pi}}{\sqrt{\pi}} \\
        = \ & \euler^{\pi\xi^2}
    \end{align*}

\end{proof}

\begin{proposition}

    证明:
    \[\left| \int_{\gamma}{\euler^{\imag z^2}}\diff z \right| \leq \dfrac{\pi(1 - \euler^{-r^2})}{4r}\]
    其中$\gamma(t) = r\euler^{\imag t}$,$t \in \left[ 0, \dfrac{\pi}{4} \right]$,$r \in \mathbb{R}^+$

\end{proposition}

\begin{proof}

    \begin{align*}
        \left| \int_{\gamma}{\euler^{\imag z^2}}\diff z \right| & = \left| \int_{0}^{\frac{\pi}{4}}{\euler^{\imag r^2(\cos{2t} + \imag\sin{2t})}\imag r\euler^{\imag t}}\diff t \right| \\
        & \leq \int_{0}^{\frac{\pi}{4}}{\left| \euler^{\imag r^2(\cos{2t} + \imag\sin{2t})}\imag r\euler^{\imag t} \right|}\diff t \\
        & = \int_{0}^{\frac{\pi}{4}}{r\euler^{-r^2\sin{2t}}}\diff t
    \end{align*}

    因为$\sin{2t} \geq \dfrac{4t}{\pi}$,$t \in \left[ 0, \dfrac{\pi}{4} \right)$,则
    \[\int_{0}^{\frac{\pi}{4}}{r\euler^{-r^2\sin{2t}}}\diff t \leq \int_{0}^{\frac{\pi}{4}}{r\euler^{\frac{4r^2t}{\pi}}}\diff t = \dfrac{\pi}{4r}(1 - \euler^{-r^2}) \]

\end{proof}

\begin{proposition}

    设$f(z)$在复平面上处处解析,并且不等式
    \[\int_{0}^{2\pi}{|f(\mathrm{e^{\imag \theta}})|}\diff \theta \leq r^{\frac{16}{5}} \]
    对所有$r > 0$成立。证明:$f(z) \equiv 0$

\end{proposition}

\begin{proof}

    由题设$f(z)$在复平面$\mathrm{C}$内都能\textup{Taylor}展开,则
    \[f(z) = \sum\limits_{n = 0}^{\infty}{\dfrac{f^{(n)}(z_0)}{n!}(z - z_0)^n}\]
    应用\textup{Cauthy}积分公式,得
    \[f(0) = \dfrac{1}{2\pi} \int_{0}^{2\pi}{f(\euler^{\imag \theta})}\diff \theta\]
    从而$2\pi|f(0)| \leq 6{\frac{16}{5}}$,令$r \to 0^{+}$,有$f(0) = 0$
    再由
    \[f'(0) = \dfrac{1}{2\pi} \int_{0}^{2\pi}{\dfrac{f(\euler^{\imag \theta})}{r\euler^{\imag \theta}}}\diff \theta\]
    则可知$f'(0) = 0$,同理$f''(0) = f^{(3)}(0) = 0$ \\
    当$n \geq 4$时
    \[f^{(n)}(0) =  \dfrac{n!}{2\pi} \int_{0}^{2\pi}{\dfrac{f(\euler^{\imag\theta})}{r^n \euler^{\imag\theta}}}\diff \theta\]
    \[|f^{(n)}(0)| \leq \dfrac{n!}{2\pi} \dfrac{r^{\frac{16}{5}}}{r^n} \to 0 \quad (r \to + \infty)\]
    从而
    \[f^{(n)}(0) = 0, \quad \forall n \in \mathbb{N}\]
    即
    \[f(z) \equiv 0\]
    
\end{proof}

\begin{proposition}

    设函数$f$在$z = 0$的邻域上连续。证明:

    \begin{enumerate}

        \item \[\lim\limits_{r \to 0}{\int_{0}^{2 \pi}{f(r\euler^{\imag t})}\diff t} = 2 \pi f(0)\]
        
        \item \[ \lim\limits_{r \to 0}{\int_{L}{\dfrac{f(z)}{z}}\diff z} = 2 \pi \imag f(0)\]
        
    \end{enumerate}

    其中$L$是圆$|z| = r$.

\end{proposition}

\begin{proof}

    \begin{enumerate}

        \item 
            因为函数$f$在$z = 0$的邻域上连续,则$\forall \varepsilon > 0$,$\exists \delta$,使得当$r < \delta$时
            \[|f(r\euler^{\imag t}) - f(0| < \dfrac{\varepsilon}{2 \pi}, \quad 0 \leq t \leq 2 \pi\]
            因此

            \begin{align*}
                \left| \int_{0}^{2 \pi}{f(r\euler^{\imag t})}\diff t - 2 \pi f(0) \right| & = \left| \int_{0}^{2 \pi}{f(r\euler^{\imag t}) - f(0)}\diff t \right| \\
                & \leq \int_{0}^{2 \pi}{\left| f(r\euler^{\imag t}) - f(0) \right|}\diff t \\
                & < \dfrac{\varepsilon}{2 \pi}\int_{0}^{2 \pi}\diff t \\
                & = \varepsilon
            \end{align*}

        \item 
            设$z = r\euler^{\imag t}$,$0 \leq t \leq 2 \pi$,则$\forall \varepsilon > 0$,$\exists \delta$,使得当$r < \delta$时

            \begin{align*}
                \left| \int_{L}{\dfrac{f(z)}{z}}\diff t - 2 \pi \imag f(0) \right| & = \left| \int_{0}^{2 \pi}{\dfrac{f(r\euler^{\imag t})}{r\euler^{\imag t}}r\euler^{\imag t} \imag - f(0)\imag}\diff t \right| \\
                & = \left| \imag \int_{0}^{2 \pi}{f(r\euler^{\imag t}) - f(0)}\diff t \right| \\
                & \leq \int_{0}^{2 \pi}{|f(r\euler^{\imag t}) - f(0)|}\diff t \\
                & < \dfrac{\varepsilon}{2 \pi}\int_{0}^{2 \pi}\diff t \\
                & = \varepsilon
            \end{align*}

    \end{enumerate}

\end{proof}

\begin{proposition}

    设$f(z) = c_0 + c_1 z + \cdots + c_n z^n$是一多项式。

    \begin{enumerate}

        \item 
            如果系数$c_k\ (k = 0, 1, \cdots, n)$是实数,证明:
            \[\int_{-1}^{1}{f^2(x)}\diff x \leq \pi \int_{0}^{2\pi}{|f(\euler^{\imag \theta})|^2}\dfrac{\diff \theta}{2 \pi} = \pi \sum\limits_{k = 0}^{n}{c_k^2}\]

        \item 
            如果系数$c_k\ (k = 0, 1, \cdots, n)$是复数,证明:
            \[\int_{-1}^{1}{|f(x)|^2}\diff x \leq \pi \int_{0}^{2\pi}{|f(\euler^{\imag \theta})|^2}\dfrac{\diff \theta}{2 \pi} = \pi \sum\limits_{k = 0}^{n}{|c_k|^2}\]
        
        \item 
            证明:
            \[\left| \sum\limits_{j, k = 0}^{n}{\dfrac{c_j c_k}{j + k + 1}} \right| \leq \pi \sum\limits_{k = 0}^{n}{c_k|^2}\]

    \end{enumerate}

\end{proposition}

\begin{proof}

    \begin{enumerate}

        \item 
        先证右边等号成立,注意到
        \[|f(\euler^{\imag \theta})|^2 = (c_0 + c_1 \cos{\theta} + \cdots + c_n \cos{n\theta})^2 + (c_0 + c_1 \sin{\theta} + \cdots + c_n \sin{n\theta})^2\]
        易知
        $$\int_{0}^{2\pi}{\cos{mx}\cos{nx}}\diff x = \left\{ 
        \begin{aligned}
            & \pi, & m = n, \\
            & 0, & m \neq n,
        \end{aligned} \right.$$
        $$\int_{0}^{2\pi}{\sin{mx}\sin{nx}}\diff x = \left\{ 
        \begin{aligned}
            & \pi, & m = n, \\
            & 0, & m \neq n,
        \end{aligned} \right.$$
        所以
        \[\pi \int_{0}^{2\pi}{|f(\euler^{\imag \theta})|^2}\dfrac{\diff \theta}{2 \pi} = \pi \sum\limits_{k = 0}^{n}{c_k^2}\]
        再证不等号。应用\textup{Cauthy}积分公式得
        
        \begin{align*}
            0 & = \int_{\Gamma_1}{f^2(x)}\diff x + \int_{\Gamma_2}{f^2(z)}\diff z \\
            & = \int_{-1}^{1}{f^2(x)}\diff x + \int_{0}^{\pi}{f^2(\euler^{\imag \theta})\imag \euler^{\imag \theta}}\diff \theta 
        \end{align*}

        其中,$\Gamma_1$:从$(-1, 0)$到$(1, 0)$的直径,$\Gamma_2$:从$(1, 0)$沿单位圆逆时针到$(-1, 0)$. \\
        所以
        \[\int_{-1}^{1}{f^2(x)}\diff x \leq \left| \int_{0}^{\pi}{f^2(\euler^{\imag \theta})\imag \euler^{\imag \theta}}\diff \theta \right| \leq \int_{0}^{\pi}{|f(\euler^{\imag \theta})|^2}\diff \theta\]
        再把积分曲线变为下半圆可得
        \[\int_{-1}^{1}{f^2(x)}\diff x \leq \int_{\pi}^{2\pi}{|f(\euler^{\imag \theta})|^2}\diff \theta\]
        因此
        \[\int_{-1}^{1}{f^2(x)}\diff x \leq \dfrac{1}{2} \int_{0}^{2\pi}{|f(\euler^{\imag \theta})|^2}\diff \theta\]

    \item 
        设$c_k = x_k + \imag y_k \ (x_k, y_k \in \mathbb{R})$,并且

    \begin{align*}
        f(\euler^{\imag \theta}) & = [(x_0 + x_1 \cos{\theta} + \cdots + x_n \cos{n\theta}) - (y_1 \sin{\theta} + y_2 \sin{2\theta} + \cdots + y_n \sin{n\theta})] \\
        & \ + \imag[(x_1 \sin{\theta} + \cdots + x_n \sin{n\theta}) + (y_0 + y_1 \cos{\theta} + y_2 \cos{2\theta} + \cdots + y_n \cos{n\theta})]
    \end{align*}

    易知
    \[\int_{0}^{2\pi}{\cos{mx}\sin{nx}} \diff x = 0, \quad \forall m,n \in \mathbb{N}\]
    同\textup{(1)}可证等号部分成立。 \\
    令$x = \cos{\theta}$,$\theta \in [0, \pi]$得
    \[\int_{-1}^{1}{|f(x)|^2}\diff x = \dfrac{1}{2} \int_{0}^{2\pi}{|f(\cos{\theta})|^2|\sin{\theta}|}\diff \theta\]
    令$x = \sin{\theta}$,$\theta \in \left[ 0, \dfrac{\pi}{2} \right]$,并作换元$t = \pi -\theta$可得
    \[\int_{-1}^{1}{|f(x)|^2}\diff x = \dfrac{1}{2} \int_{0}^{2\pi}{|f(\sin{\theta})|^2|\cos{\theta}|}\diff \theta\]
    令$x = \sin{\theta}$,$\theta \in \left[ \pi, \dfrac{3\pi}{2} \right]$,并作换元$t = 3\pi -\theta$可得
    \[\int_{-1}^{1}{|f(x)|^2}\diff x = \dfrac{1}{2} \int_{0}^{2\pi}{|f(\sin{\theta})|^2|\cos{\theta}|}\diff \theta\]
    所以
    \begin{align*}
        \int_{-1}^{1}{|f(x)|^2}\diff x & \leq \dfrac{1}{4} \int_{0}^{2\pi}{|f(\cos{\theta})|^2 + |f(\sin{\theta})|^2} \diff \theta \\
        & \dfrac{1}{4} (2\pi) \left[ x_0^2 + y_0^2 + \sum\limits_{k = 0}^{n}{(x_k^2 + y_k^2)} \right] \\
        & \leq \dfrac{1}{4} (2\pi) \left[ 2\sum\limits_{k = 0}^{n}{(x_k^2 + y_k^2)} \right] \\
        & = \pi \sum\limits_{k = 0}^{n}{|c_k|^2}
    \end{align*}

    \item 
        对$f^2(x)$展开,可得$x^m$的系数是$\sum\limits_{j + k = m}{c_j c_k}$,我们把它记为$a_m \ (m = 0, 1, \cdots, 2n)$. \\
        注意到下面的等式成立
        
        \begin{align*}
            \left| \int_{0}^{1}{f^2(x)}\diff x \right| & =  \left| \int_{0}^{1}{\sum\limits_{m = 0}^{2n}{a_m x^m}}\diff x \diff x \right| \\
            & =  \left| \sum\limits_{m = 0}^{2n}{\dfrac{a_m}{m + 1}} \right| \\
            & =  \left| \sum\limits_{j, k = 0}^{n}{\dfrac{c_j c_k}{j + k + 1}} \right|
        \end{align*}

        由\textup{(2)}结论可得

        \begin{align*}
            \left| \int_{0}^{1}{f^2(x)}\diff x \right| & \leq \int_{0}^{1}{|f(x)|^2}\diff x \\
            & \leq \int_{-1}^{1}{|f(x)|^2}\diff x \\
            & \leq \pi \sum\limits_{k = 0}^{n}{|c_k|^2}
        \end{align*}

        上面的等式成立当且仅当在$-1 < x < 0$上有$|f(x)| \equiv 0$,即$c_0 = c_1 = \cdots = c_n = 0$. 

    \end{enumerate}

    
\end{proof}

\section{复级数}

\begin{proposition}

    设$\lim\limits_{n \to \infty}{a_nz^n}$收敛。证明:
    \[\lim\limits_{r \to 1, \ r < 1}{\sum\limits_{n = 1}^{\infty}{a_nr^n}} = \sum\limits_{n = 1}^{\infty}{a_n}\]

\end{proposition}

\begin{proof}
    
    记$A_k = \sum\limits_{n = 1}^{k}{A_n}$,$A_0 = 0$,则

    \begin{align*}
        \sum\limits_{n = 1}^{N}{(1 - r^n)a_n} & = (1 - r^N)A_N + \sum\limits_{n = 1}^{N}{(r^n - r^{n + 1})A_n} \\
        & = (1 - r)A_N + \sum\limits_{n = 1}^{N}{(r^n - r^{n + 1})(A_N - A_n)}
    \end{align*}

    因为
    \[\lim\limits_{r \to 1, \ r < 1}{(1 - r)A_N} = 0\]
    所以只需证
    \[\lim\limits_{r \to 1, \ r < 1}{(r^n - r^{n + 1})(A_N - A_n)} = 0\]
    $\forall \varepsilon > 0 $,$\exists N' \in \mathbb{N}$,$n > N'$时
    \[|A_n - A| < \varepsilon \]
    其中$A = \lim\limits_{n \to \infty}{A_N}$ \\
    取$(1 - \varepsilon)^{\frac{1}{N'}} < r < 1$,则
    \[0 < 1 - r < 1 - r^{N'} < \varepsilon \]
    由于$A_n$有界,设$|A_n| \leq M $,则$n > N'$时
    
    \begin{align*}
        \sum\limits_{n = 1}^{N - 1}{(r^n - r^{n + 1})(A_N - A_n)} & \leq (1 - r)\sum\limits_{n = 1}^{N'}{r^n|A_n - A|} + \sum\limits_{n = N' + 1}^{N - 1}{r^n(|A_n - A| + |A_N - A|)} \\
        & \leq (\sum\limits_{n = 1}^{N'}{2r^nM} + \sum\limits_{n = N' + 1}^{N - 1}{r^n2\varepsilon}) \\
        & = (1 - r)\left( 2M\dfrac{r - r^{N' + 1}}{1 - r} + 2\varepsilon r^{N' + 1}\dfrac{1 - R^{N - N' + 1}}{1 - r} \right) \\
        & < 2M\varepsilon + 2\varepsilon \\
        & = 2(M + 1)\varepsilon
    \end{align*}

    令$n \to +\infty$,再$r \to 1$ \\
    即
    \[\lim\limits_{r \to 1, \ r < 1}{\sum\limits_{n = 1}^{\infty}{a_nr^n}} = \sum\limits_{n = 1}^{\infty}{a_n}\]

\end{proof}

\begin{proposition}
    
    设$f$在$\mathbb{C}$上解析,且满足对于任一点$z_0\in\mathbb{C}$,$f$在$z_0$的\textup{Tarloy}展开式
    \[f(z) = \sum\limits_{n = 0}^{+\infty}{a_n(z - z_0)}\]
    满足至少有一个$c_n$为$0$. 证明$f$是多项式。

\end{proposition}

\begin{proof}
    
    考虑开单位圆盘$D = \Delta(0, 1)$,则$D$内有不可数的点,所以存在正整数$p$,使得$D$内有无穷多点$\{z_n\}$,
    在这些点处的\textup{Taylor}展开式中第$p$项系数$c_p = 0$ \\
    可以在这些点中选取一个收敛子列$\{z_{n_k}\}$,且该子列收敛到$z_0 \in D$,
    由唯一性定理知,$f^{(p)}\equiv 0$在$\mathbb{C}$上恒成立,故$f$为多项式。

\end{proof}

\begin{proposition}

    设
    \[S_n = \sum\limits_{k = 0}^{\infty}{a_k}\]
    证明:如果幂级数$A = \sum\limits_{k = 0}^{\infty}{a_kz^k}$的收敛半径$R(A)$为$1$,
    则幂级数$C = \sum\limits_{k = 0}^{\infty}{S_k z^k}$的收敛半径$R(C)$也为$1$.

\end{proposition}

\begin{proof}
    
    因为
    \[\sum\limits_{k = 0}^{\infty}{z^k} \cdot \sum\limits_{k = 0}^{\infty}{a_kz^k} = \sum\limits_{k = 0}^{\infty}{\sum\limits_{m = 0}^{\infty}{u_mz^m}} = \sum\limits_{k = 0}^{\infty}{S_kz^k} = C\]
    则有$R(C) \geq 1$,因为左边乘积中幂级数的收敛半径都是$1$,右边幂级数$C$的收敛半径$R(C)$不可能大于$1$,即$R(C) \leq 1$. 则对任意$\varepsilon > 0$,存在$N \in \mathbb{N}$,当$n > N$时,对任意$p \in \mathbb{N}^{+}$有
    \[\varepsilon > \sum\limits_{k = n + 1}^{n + p}{|S_k||z|^k} \geq \sum\limits_{k = n + 1}^{n + p}{(n + p - k + 1)|a_k||z|^k}, \quad 1 < |z| < R\]
    所以
    \[\sum\limits_{k = n + 1}^{n + p}{(n + p - k)|a_k||z|^k} \leq \sum\limits_{k = n + 1}^{n + p}{(n + p - k + 1)|a_k||z|^k} < \varepsilon, \quad 1 < |z| < R \eqno{(1)}\]
    因此
    \[\varepsilon > \sum\limits_{k = n + 1}^{n + p}{(n + p - k + 1)|a_k||z|^k} = \sum\limits_{k = n + 1}^{n + p}{(n + p - k)|a_k||z|^k} + \sum\limits_{k = n + 1}^{n + p}{|a_k||z|^k}\]
    由\textup{(1)}式知,
    \[\sum\limits_{k = n + 1}^{n + p}{|a_k||z|^k} < \varepsilon, \quad 1 < |z| < R\]
    若$R(C) < 1$,则必有$R(A) < 1$,与题设矛盾。得证。

\end{proof}

\begin{theorem}[Tauber定理]

    幂级数$\sum\limits_{n = 0}^{\infty}{a_k z^n}$的收敛半径为$1$,且$\lim\limits_{n \to \infty}{na_n} = 0$. 有

    \begin{enumerate}
        
        \item \[\lim\limits_{m \to \infty}{\dfrac{\sum\limits_{n = 1}^{m}{n|a_n|}}{m}} = 0\]
        
        \item 
            设
            \[f(z) = \sum\limits_{n = 0}^{m}{a_n z^n}, \quad |z| < 1\]
            如果存在$\lim\limits_{x \to 1}{f(x)} = A$,那么级数$\sum\limits_{n = 0}^{n}{a_n}$收敛到$A$.
        
    \end{enumerate}

\end{theorem}

\begin{proof}

    \begin{enumerate}

        \item 
            $\forall \varepsilon > 0$,$\exists N \in \mathbb{N}$,使得当$n > N$时,$n|a_n| < \dfrac{\varepsilon}{2}$.
            选择$n = n_0$满足上述条件。则对任意$m > n_0$,有
            
            \begin{align*}
                \dfrac{\sum\limits_{n = 1}^{m}{n|a_n|}}{m} & = \dfrac{\sum\limits_{n = 1}^{n_0}{n|a_n|}}{m} + \dfrac{\sum\limits_{n = n_0 + 1}^{m}{n|a_n|}}{m} \\
                & < \dfrac{\sum\limits_{n = 1}^{n_0}{n|a_n|}}{m} + \dfrac{m - n_0}{2m} \varepsilon \\
                & < \dfrac{\sum\limits_{n = 1}^{n_0}{n|a_n|}}{m} + \dfrac{\varepsilon}{2}
            \end{align*}

            对足够大的$m > M$,有
            \[\dfrac{\sum\limits_{n = 1}^{n_0}{n|a_n|}}{m} < \dfrac{\varepsilon}{2}\]
            因此对$m > \max{(N, M)}$有
            \[\dfrac{\sum\limits_{n = 1}^{m}{n|a_n|}}{m} < \varepsilon\]

        \item 
            \begin{align*}
                \left| \sum\limits_{k = 0}^{n}{a_k} - f(x) \right| & = \left| \sum\limits_{k = 0}^{n}{a_k(1 - x^k)} - \sum\limits_{k = n + 1}^{\infty}{a_k x^k} \right| \\
                & \leq (1 - x)\sum\limits_{k = 0}^{n}{\left[ |a_k|(1 + x + \cdots + x^{k - 1}) \right]} + \sum\limits_{k = n + 1}^{\infty}{|a_k|x^k} \\
                & < n(1 - x)\dfrac{\sum\limits_{k = 0}^{n}{k|a_k|}}{n} + \sum\limits_{k = n + 1}^{\infty}{k|a_k|\dfrac{x^k}{k}}        
            \end{align*}

            令$m|a_m| < \dfrac{\varepsilon}{2}$,由\textup{(a)}知,$\forall \varepsilon > 0$,$\exists M \in \mathbb{N}$,使得当$m > M$时有 
            \[\dfrac{\sum\limits_{k = 0}^{n}{k|a_k|}}{m} < \dfrac{\varepsilon}{2}\]
            我们由上述不等式得

            \begin{align*}
                \left| \sum\limits_{k = 0}^{n}{a_k} - f(x) \right| & < n(1 - x)\dfrac{\varepsilon}{2} + \dfrac{\varepsilon}{2n}\sum\limits_{k = n + 1}^{\infty}{x^k} \\
                & = \dfrac{\varepsilon}{2}\left[ n(1 - x) + \dfrac{1}{n} \dfrac{x^{n + 1}}{1 - x} \right] \\
                & < \dfrac{\varepsilon}{2}\left[ n(1 - x) + \dfrac{1}{n} \dfrac{1}{1 - x} \right]
            \end{align*}

            即
            \[\sum\limits_{k = 0}^{n}{a_k} = A\]

        \end{enumerate}

\end{proof}

\begin{lemma}\label{lemma:radius}

    设函数
    \[\varphi(z) = \sum\limits_{n = 0}^{\infty}{a_n (z - z_0)^n}\]
    的收敛半径为$R$,$w$是收敛圆边界上的点,$z_1$是半径$\vec{z_0w}$上不同于$z_0$和$w$的任意一点。
    证明:如果
    \[\Delta = R - |z_1 - z_0| = \dfrac{1}{\limsup\limits_{n \to \infty}{\sqrt[n]{\dfrac{|\varphi^{(n)}(z_1)|}{n!}}}}\]
    则$w$是$\varphi(z)$的发散点。且如果
    \[\Delta < \dfrac{1}{\limsup\limits_{n \to \infty}{\sqrt[n]{\dfrac{|\varphi^{(n)}(z_1)|}{n!}}}}\]
    则$w$是$\varphi(z)$的收敛点。

\end{lemma}

\begin{proof}

    由题设知,$\varphi(z)$在$|z - z_0| < R$内收敛,固定$w$,选择$z_1$,并将$\varphi(z)$在$z = z_1$展开,得到
    \[\varphi(z) = \sum\limits_{n = 0}^{\infty}{b_n (z - z_1)^n}, \quad |z - z_1| < r\]
    其中$r$是收敛半径,且有

    \begin{align*}
        b_n & = \dfrac{\varphi^{(n)}{z_1}}{n!} \\
        & = a_n + \dfrac{n + 1}{1}a_{n + 1}(z_1 - z_0) + \dfrac{(n + 1)(n + 2)}{1 \cdot 2}a_{n + 2}(z_1 - z_0)^2 + \cdots \\
        & = \sum\limits_{m = n}^{\infty}{\binom{m}{n}a_m(z_1 - z_0)^{m - n}}
    \end{align*}

    易知,若$w$是$\varphi(z)$的发散点,则$\varphi(z)$在$z = z_1$处的\textup{Taylor}展开式的收敛半径为$r$,因此对$b_n$有
    \[\dfrac{1}{\limsup\limits_{n \to \infty}{\sqrt[n]{\dfrac{|\varphi^{(n)}(z_1)|}{n!}}}} = \dfrac{1}{\limsup\limits_{n \to \infty}{\sqrt[n]{|b_n|}}} = r\]
    即
    \[\Delta = R - |z_1 - z_0| = r = \dfrac{1}{\limsup\limits_{n \to \infty}{\sqrt[n]{\dfrac{|\varphi^{(n)}(z_1)|}{n!}}}}\]
    反之,若$w$是$\varphi(z)$的收敛点,则$\varphi(z)$在$z = z_1$处的\textup{Taylor}展开式的收敛半径大于$r$,因此对$b_n$有
    \[\Delta < \dfrac{1}{\limsup\limits_{n \to \infty}{\sqrt[n]{\dfrac{|\varphi^{(n)}(z_1)|}{n!}}}}\]

\end{proof}

\begin{theorem}[Pringshajm定理]
    
    设幂级数$\varphi(z) = \sum\limits_{n = 0}^{\infty}{a_n z^n}$收敛半径$R = 1$,且$a_n \geq n$,$n \in \mathbb{N}$,
    则$z = 1$是$\varphi(z)$的发散点。

\end{theorem}

\begin{proof}

    反证法。假设$z_0 = 1$不是$\varphi(z)$的发散点。设$x$是区间$(0, 1)$上的任意点,由引理\ref{lemma:radius}可知
    \[\Delta = R - |x - z_0| = 1 - x < \dfrac{1}{\limsup\limits_{n \to \infty}{\sqrt[n]{\dfrac{|\varphi^{(n)}(z_1)|}{n!}}}}\]
    设$w$是单位圆上的任意一点,且$z_1$是半径$\vec{z_0w}$和圆$|z| = x$的交点,则有
    \[R - |z_1| = 1 - x\]
    且有
    
    \begin{align*}
        |\varphi^{(n)}(z_1)| & = \left| a_n + \dfrac{n + 1}{1} a_{n + 1}z_1 + \dfrac{(n + 1)(n + 2)}{1 \cdot 2}a_{n + 2}z_1^2 + \cdots \right| \\
        & \leq a_n + \dfrac{n + 1}{1} a_{n + 1}x + \dfrac{(n + 1)(n + 2)}{1 \cdot 2}a_{n + 2}x^2 + \cdots \\
        & = \varphi^{(n)}(x)
    \end{align*}

    因此
    \[\dfrac{1}{\limsup\limits_{n \to \infty}{\sqrt[n]{\dfrac{|\varphi^{(n)}(z_1)|}{n!}}}} \geq \dfrac{1}{\limsup\limits_{n \to \infty}{\sqrt[n]{\dfrac{|\varphi^{(n)}(x)|}{n!}}}}\]
    从最后一个不等式知
    \[\Delta < \dfrac{1}{\limsup\limits_{n \to \infty}{\sqrt[n]{\dfrac{|\varphi^{(n)}(z_1)|}{n!}}}}\]
    由$w$的任意性和引理\ref{lemma:radius}可知,$\varphi(z)$在$|z| = 1$上没有发散点,这与$\varphi(z)$的收敛半径$R = 1$矛盾。得证。

\end{proof}

\begin{proposition}

    设$\sum\limits_{m = -\infty}^{+\infty}{|a_m|} < \infty$,求
    \[\lim\limits_{n \to \infty}{\dfrac{1}{2n + 1}\sum\limits_{m = -\infty}^{+\infty}{|a_{m - n} + a_{m - n + 1} + \cdots + a_{m + n}|}}\]

\end{proposition}

\begin{proof}

    因为$\sum\limits_{m = -\infty}^{+\infty}{|a_m|} < \infty$,所以存在$\sum\limits_{m = -\infty}^{+\infty}{a_m} = S$ \\
    令
    \[C_n = \dfrac{1}{2n + 1}\sum\limits_{m = -\infty}^{+\infty}{|a_{m - n} + a_{m - n + 1} + \cdots + a_{m + n}|}\]
    下证$\lim\limits_{n \to \infty}{C_n} = |S|$. $\forall \varepsilon > 0$,$\exists M \in \mathbb{N}$,使得
    \[\sum\limits_{|m| > M}{|a_m|} < \varepsilon\]
    则对$m > M$有

    \begin{align*}
        (2n + 1)C_n & = \sum\limits_{|m| > n + M}{|a_{m - n} + \cdots + a_{m + n}|} \\
        & \quad + \sum\limits_{n + M \leq |m| < n - M}{|a_{m - n} + \cdots + a_{m + n}|} \\
        & \quad + \sum\limits_{|m| < n - M}{|a_{m - n} + \cdots + a_{m + n}|}
    \end{align*}

    因为

    \begin{align*}
        \sum\limits_{|m| > n + M}{|a_{m - n} + \cdots + a_{m + n}|} & \leq \sum\limits_{|m| > n + M}{|a_{m - n}| + \cdots + |a_{m + n}|} \\
        & \leq (2n + 1)\sum\limits_{|m| > M}{|a_m|} \\
        & \leq (2n + 1)\varepsilon
    \end{align*}

    且
    \[\sum\limits_{|m| < n - M}{|a_{m - n} + \cdots + a_{m + n}|} \leq \sum\limits_{|m| < n - M}{\sigma} \leq 4M\sigma\]
    因为$|m| \leq n - M$即$m - n \leq -M \leq M \leq m + n$,则对$|m| \leq n - M$有
    \[\Big| |a_{m - n} + \cdots + a_{m + n}| - |S| \Big| \leq \sum\limits_{|m| \leq M}{|a_m|} < \varepsilon\]
    所以

    \begin{align*}
        & \left| \sum\limits_{|m| < n - M}{|a_{m - n} + \cdots + a_{m + n}|} - (2n - 2M + 1)|S| \right| \\
        \leq & \sum\limits_{|m| < n - M}{|a_{m - n} + \cdots + a_{m + n}| - |S|} \\
        \leq & \ (2n - 2M + 1)\varepsilon        
    \end{align*}

    因此
    \[\Big| (2n + 1)C_n - (2n - 2M + 1)|S| \Big| \leq (2n + 1)\varepsilon + 4M\sigma + (2n - 2M + 1)\varepsilon\]
    两边除以$(2n + 1)$并令$n \to \infty$得到
    \[\limsup\limits_{n \to \infty}{\Big|C_n - |S|\Big|} \leq \varepsilon + \varepsilon + 2\varepsilon\]
    由$\varepsilon$的任意性知
    \[\lim\limits_{n \to \infty}{C_n} = |S|\]

\end{proof}

\begin{proposition}

    设
    \[f(z) = \dfrac{\sum\limits_{n = 0}^{\infty}{a_n(z - a)^n}}{\sum\limits_{n = 0}^{\infty}{b_n(z - a)^n}}, \quad |z - a| < R\]
    其中$b_0 \neq 0$. 证明:$f(z)$可以表示为下列形式
    \[f(z) = \sum\limits_{n = 0}^{\infty}{c_n(z - a)^n}, \quad |z - a| < R\]
    并用系数$a_n$和$b_n$来表示上述系数$c_n$.

\end{proposition}

\begin{proof}

    易知
    \[\dfrac{\sum\limits_{n = 0}^{\infty}{a_n(z - a)^n}}{\sum\limits_{n = 0}^{\infty}{b_n(z - a)^n}}\]
    在$|z - a| < R$内解析,因此存在\textup{Taylor}级数展开式
    \[f(z) = \sum\limits_{n = 0}^{\infty}{c_n(z - a)^n}, \quad |z - a| < R\]
    则有

    \begin{align*}
        \sum\limits_{n = 0}^{\infty}{a_n(z - a)^n} & = \sum\limits_{n = 0}^{\infty}{c_n(z - a)^n} \cdot \sum\limits_{n = 0}^{\infty}{b_n(z - a)^n} \\
        & = \sum\limits_{n = 0}^{\infty}{\sum\limits_{k = 0}^{\infty}{a_kb_{n - k}(z - a)^n}}
    \end{align*}

    由幂级数展开的唯一性可得

    \begin{align*}
        & c_0 b_0 = a_0 \\
        & c_0 b_1 + c_1 b_0 = a_1 \\
        & c_0 b_2 + c_1 b_1 + c_2 b_0 = a_2 \\
        & \vdots \\
        & c_0 b_n + c_1 b_{n - 1} + c_2 b_{n - 2} + \cdots + c_n b_0 = a_n
    \end{align*}

    可得线性方程组,解得
    $$c_n = \dfrac{1}{b_{n + 1}}
    \begin{vmatrix} 
        b_0 & 0 & 0 & \cdots & a_0 \\
        b_1 & b_0 & 0 & \cdots & a_1 \\
        b_2 & b_1 & b_0 & \cdots & a_2 \\
        \vdots & \vdots & \vdots & \cdots & \vdots \\
        b_n & b_{n - 1} & b_{n - 2} & \cdots & a_n
    \end{vmatrix}$$
    且
    $$\begin{vmatrix} 
        b_0 & 0 & 0 & \cdots & 0 \\
        b_1 & b_0 & 0 & \cdots & 0 \\
        b_2 & b_1 & b_0 & \cdots & 0 \\
        \vdots & \vdots & \vdots & \cdots & \vdots \\
        b_n & b_{n - 1} & b_{n - 2} & \cdots & b_0
    \end{vmatrix} = b_0^{n + 1} \neq 0$$

\end{proof}

\section{解析函数}

\begin{proposition}
    
    设$f$在一个包含单位圆盘的开集上(除去单位圆盘上的唯一一个极点$z_0$)解析。\\
    证明:若$\sum\limits_{n = 1}^{\infty}{a_nz^n}$表示$f$在开单位圆盘上的\textup{Taylor}级数,那么
    \[\lim\limits_{n \to \infty}{\dfrac{a_n}{a_{n + 1}}} = z_0\]

\end{proposition}

\begin{proof}
    
    函数$f$在单位圆内可以表示为
    \[f(z) = \dfrac{c}{z - z_0} + h(z)\]
    其中$h(z)$是解析函数且$c \neq 0$. 因为$h(z)$解析,因此在单位圆内有\textup{Taylor}级数
    \[h(z) = \sum\limits_{n = 0}^{\infty}{b_n z^n}\]
    又$\dfrac{c}{z - z_0}$能展开为幂级数
    \[\dfrac{c}{z - z_0} = -\dfrac{c}{z_0}\dfrac{1}{1 - \frac{z}{z_0}} = \sum\limits_{n = 0}^{\infty}{\left( \dfrac{z}{z_0} \right)^n}\]
    所以
    \[f(z) = \sum\limits_{n = 0}^{\infty}{\left( b_n - \dfrac{c}{z^{n + 1}} \right)z^n} = \sum\limits_{n = 0}^{\infty}{a_n z^n}\]
    则有
    $$\dfrac{a_n}{a_{n + 1}} = 
    \dfrac{b_n - \frac{c}{z_0^{n + 1}}} {b_{n + 1} - \frac{c}{z_0^{n + 2}}} = 
    z_0 \dfrac{(b_n z_0^n) z_0 - c} {(b_{n + 1}z_0^{n + 1}) z_0 - c} =
    z_0 \dfrac{z_0 h_n - c}{z_0 h_{n + 1} - c}$$
    其中$h_n$是$h(z)$在$z = z_0$处\textup{Taylor}级数第$n$项的系数。因为$h(z)$在单位圆内解析,有$\lim\limits_{n \to \infty}{h_n} = 0$. 且
    \[\lim\limits_{n \to \infty}{\dfrac{a_n}{a_{n + 1}}} = \lim\limits_{n \to \infty}{\left( a_0 \dfrac{a_0 h_n - c}{z_0h_{n + 1} - c} \right)} = z_0\]

\end{proof}

\begin{proposition}

    证明:函数$\dfrac{d_r}{r}$是$r$的单调不减函数。其中
    \[d_r = \sup\{|f(z) - f(w)|\big| z, w \in D(0, r), |z| = |w|\}\]

\end{proposition}

\begin{proof}

    考虑$0 < r < R < 1$,易知$d_r = \sup\{|f(z) - f(zu)| \big| z \in D(0, 1), |z| = 1 \}$,注意到对于任意的$|u| = 1$,$z \mapsto  \dfrac{f(z) - f(zu)}{z}$是单位圆$D(0, 1)$上的解析函数,因此
    
    \begin{align*}
        \dfrac{d_r}{r} & = \dfrac{1}{r} \sup_{z \in D(0, r)}{|f(z) - f(zu)|} \\
        & = \dfrac{1}{r} \sup_{|z| = r}{|f(z) - f(zu)|} \\
        & = \sup_{|z| = r}{\left| \dfrac{f(z) - f(zu)}{z} \right|} \\
        & \leq \sup_{z \in \overline{D}(0, R)}{\left| \dfrac{f(z) - f(zu)}{z} \right|} \\
        & = \sup_{|z| = R}{\left| \dfrac{f(z) - f(zu)}{z} \right|} \\
        & = \dfrac{1}{R} \sup_{|z| = R}{|f(z) - f(zu)|} \\
        & = \dfrac{1}{R} \sup_{z \in D(0, R)}{|f(z) - f(zu)|} \\
        & = \dfrac{1}{R} d_{R} \leq \dfrac{1}{R} d_{1}
    \end{align*}

    令$R \to 1$,则有
    \[\dfrac{d_{r}}{r} \leq \dfrac{1}{R} d_{R} \leq d_1\]

\end{proof}

\begin{proposition}

    已知$D$是单位圆盘,设$f:D \to \mathbb{C}$是解析函数。证明:\\
    函数$f$的直径$d = \sup\limits_{z, w \in D}{|f(z) - f(w)|}$满足
    \[2|f'(0)| \leq d\]

\end{proposition}

\begin{proof}

    由\textup{Cauthy}公式得,
    \[f'(0) = \dfrac{1}{2\pi \imag} \oint_{|z| = r}{\dfrac{f(z)}{z^2}}\diff z\]
    \[f'(0) = \dfrac{1}{2\pi \imag} \oint_{|z| = r}{\dfrac{-f(-z)}{z^2}}\diff z\]
    其中$0 < r < 1$,即
    \[2f'(0) = \dfrac{1}{2\pi \imag} \oint_{|z| = r}{\dfrac{f(z) - f(-z)}{z^2}}\diff z\]

    \begin{align*}
        2|f'(0)| & = \dfrac{1}{2 \pi} \left| \oint_{|z| = r}{\dfrac{f(z) - f(-z)}{z^2}}\diff z \right| \\
        & \leq \dfrac{1}{2 \pi} \oint_{|z| = r}{\left| \dfrac{f(z) - f(-z)}{z^2} \right|}\diff z \\
        & \leq \dfrac{1}{2 \pi} \dfrac{d}{r^2} \cdot 2 \pi r \\
        & = \dfrac{d}{r}
    \end{align*}

    则显然有$2|f'(0)| \leq d$.

\end{proof}

\begin{theorem}\label{theorem:complex}

    对任意的$0 < r < 1$,
    \[2|f'(0)| = \dfrac{d_r}{r}\]
    成立当且仅当
    \[f(z) = f'(0)z + f(0), \quad \forall z \in D(0, 1)\]
    其中
    \[d_r = \sup\{|f(z) - f(w)|\big| z, w \in D(0, r), |z| = |w|\}\]
    
\end{theorem}

\begin{proof}
    
    充分性显然。\\
    必要性。易知
    \[d_r^{*} = \sup\limits_{z \in D(0, r)}{|f(z) - f(-z)|} \leq d_r\]
    所以$2|f'(0)| \leq \dfrac{d_r^{*}}{r}$,因此
    \[f(z) - f(-z) = 2f'(0)z\]
    假设存在$|a| = r$,使得
    \[\mathrm{Im}\{f'(a)\} \neq 0\]
    函数$f^{*}(z) = \overline{f(\overline{z})}$是单位圆上的解析函数。对$\theta \in \mathbb{R}$,有

    \begin{align*}
        \varphi(\theta) & = |f(a\euler^{\imag \theta}) - f(-a)|^2 \\
        & = \left| f(a\euler^{\imag \theta}) - f(a) + \dfrac{d_r^{*}}{r}a \right|^2 \\
        & = \left( f(a\euler^{\imag \theta}) - f(a) + \dfrac{d_r^{*}}{r}a \right) \left( f^{*}(\overline{a}\euler^{-\imag \theta}) - f^{*}(\overline{a}) + \dfrac{d_r^{*}}{r}\overline{a} \right)
    \end{align*}

    由\textup{Cauthy}定理知,
    \[\varphi'(\theta) = 2 \mathrm{Re}\left\{ a\imag \euler^{\imag \theta}f'(a\euler^{\imag \theta}) \left( f^{*}(\overline{a}\euler^{-\imag \theta}) - f^{*}(\overline{a}) + \dfrac{d_r^{*}}{r}\overline{a} \right) \right\}\]
    所以
    \[\varphi'(0) = 2 \mathrm{Re}\left\{ a\imag f'(a)\dfrac{d_r^{*}}{r}\overline{a} \right\} = -2\dfrac{d_r^{*}}{r}|a|^2\mathrm{Im}\{f'(a)\} \neq 0\]
    因此$\varphi(\theta)$在$\theta = 0$处要么单调增加要么单调减少,所以存在$\theta_0$,使得$\varphi(\theta_0) > \varphi(0)$. 也即
    \[d_r \geq |f(a\euler^{\imag \theta}) - f(-a)| = \sqrt{\varphi(\theta)} > \sqrt{\varphi(0)} = |f(a) - f(-a)| = \dfrac{d_r^{*}}{r}r = d_r\]
    与$d_r^{*} \leq d_r$矛盾。所以
    \[f'(z) = 0, \quad \forall |z| = r\]
    由最大模定理知,
    \[f'(z) = 0, \quad \forall z \in D(0, r)\]
    因此由\textup{Cauthy-Riemann}条件知,$f'(z)$是常值函数,即
    \[f(z) = f'(0)z + f(0), \quad \forall z \in D(0, 1)\]
    成立。
    
\end{proof}

\begin{theorem}

    设函数$f(z)$是单位圆上的解析函数。则
    \[2|f'(0)| = d\]
    当且仅当
    \[f(z) = f'(0)z + f(0)\]
    其中
    \[d = \sup\limits_{z, w \in D(0, 1)}{|f(z) - f(w)|}\]
    
\end{theorem}

\begin{proof}

    充分性显然。\\
    必要性。易知
    \[\dfrac{d_r^{*}}{r} \leq \dfrac{d_r}{r} \leq d\]
    因此
    \[2|f'(0)| = \dfrac{d_r}{r}\]
    由定理\ref{theorem:complex}知,
    \[f(z) = f'(0)z + f(0)\]
    得证。

\end{proof}

\begin{proposition}
    
    设$f$在$0 < |z| < R$解析,存在$M > 0$,使得任意$r$,$0 < r < R$,有
    \[r \int_{0}^{2 \pi}{|2f(\mathrm{e^{i\theta}})|}\diff \theta < M\]
    证明:$z = 0$是一个可去发散点,否则是简单极点。

\end{proposition}

\begin{proposition}

    设$f : D(0, 1) \to \mathbb{C}$是解析函数。证明:对于任意$0 < r < 1$,
    \[2|f'(0)| \leq \dfrac{d_r^{*}}{r}\]
    成立。其中
    \[d_r^{*} = \sup\limits_{z \in D(0, r)}{|f(z) - f(-z)|}\]
    且等号成立当且仅当
    \[f(z) - f(-z) = 2f'(0)z, \quad \forall z \in D(0, 1)\]

\end{proposition}

\begin{proof}

    不等式自证不难。下证等号成立时的情况。充分性显然。\\
    必要性。应用反证法,不妨设
    \[f(z) - f(-z) = 2f'(0)z + 2f^{(3)}(0)z^3, \quad \forall z \in D(0, 1)\]
    其中$f'(0) = r_1\euler^{\imag \theta_1}$,$f^{(3)}(0) = r_2\euler^{\imag \theta_2} \neq 0$. 则
    \[d_r^{*} = \sup\limits_{z \in D(0, r)}{(|2f'(0)z + 2f^{(3)}(0)z^3|)} = \sup\limits_{z \in D(0, r)}{(2|z||f'(0) + f^{(3)}(0)z^2|)}\]
    总存在$|z| = r$,使得$f'(0)$与$f^{(3)}(0)z^2$同向,也即
    \[\dfrac{d_r^{*}}{r} = 2|f'(0) + f^{(3)}(0)z^2| > 2|f'(0)|\]
    与题设矛盾,得证。

\end{proof}

\begin{proof}
    
    设$r_2 < R$,选取$z_0$,满足$0 < z_0 < \dfrac{r_2}{2}$,再选取$r_1$,使$0 < r_1 < \dfrac{|z_0|}{2}$ \\
    由\textup{Cauthy}积分公式得

    \begin{align*}
        |f(z_0)| & = \left| \dfrac{1}{2\mathrm{\pi i}} \int_{C_2}{\dfrac{f(z)}{z - z_0}}\diff z - \int_{C_1}{\dfrac{f(z)}{z - z_0}}\diff z \right| \\
        & \leq \dfrac{1}{2 \pi} \left( \int_{C_2}{\left| \dfrac{f(z)}{z - z_0} \right|}|\diff z| + \int_{C_1}{\left| \dfrac{f(z)}{z - z_0} \right|}|\diff z| \right)
    \end{align*}

    其中$C_1 = r_1\mathrm{e^{i\theta}}$,$C_2 = r_2\mathrm{e^{i\theta}}$
    由$|z - z_0| \geq \dfrac{r_2}{2}$,$z \in C_2$有
    \[ \dfrac{1}{2 \pi} \int_{C_2}{\left| \dfrac{f(z)}{z - z_0} \right|}|\diff z| \leq \dfrac{1}{ \pi r_2}\int_{0}^{2 \pi}{|f(\euler^{\imag\theta})}\diff \theta \leq \dfrac{M}{ \pi r_2^2}\]
    由$|z - z_0| \geq \dfrac{|z|}{2}$,$z \in C_1$有
    \[ \dfrac{1}{2 \pi} \int_{C_1}{\left| \dfrac{f(z)}{z - z_0} \right|}|\diff z| \leq \dfrac{1}{ \pi|z|}\int_{0}^{2 \pi}{|f(\euler^{\imag\theta})}\diff \theta \leq \dfrac{M}{ \pi|z|}\]
    因此,我们有
    % 似有不对
    \[|f(z)| \leq \dfrac{M}{ \pi r_2} + \dfrac{M}{ \pi|z|}, \quad 0 < |z| < \dfrac{r_2}{2}\]
    所以\[|zf(z)| \leq \dfrac{M|z|}{ \pi r_2} + \dfrac{M}{\pi}\]
    则极限是有限的。\\
    $z = 0$是可去极点,极限为$0$. \\
    $z = 0$是简单极点,极限非$0$.

\end{proof}

\begin{proposition}

    设$f(z)$在$|z| < 1 $内解析,并且$z = 0$为$f(z)$的$n(n \geq 1)$级零点,当$|z| < 1$时,$|f(z)| < 1$. \\
    证明:当$|z| < 1$时,$|f(z)| \leq |z|^n$

\end{proposition}

\begin{proof}
    
    $f(z)$在$|z| < 1$内解析,且$z = 0$为$n(n \geq 1)$级零点,则有
    \[f(z) = \sum\limits_{m = n}^{\infty}{a_mz^m}\]
    显然$\varphi(z) = \dfrac{f(z)}{z^{n - 1}}$在$|z| < 1$内解析,且$\varphi(0) = 0$,有
    \[ \max\varphi(z) \leq \max_{|z| = r}|\varphi(z)| = \dfrac{|f(z)|}{r^{n - 1}} \leq \dfrac{1}{r^{n - 1}} \]
    令$n \to 1$,则$|\varphi(z)| < 1$,于是$|\varphi(z)| \leq |z|$,即
    \[|f(z)| \leq z^n\]

\end{proof}

\begin{proposition}

    设$f(z) = \sum\limits_{n = 0}^{\infty}{a_n z^n} \ (a_n \neq 0)$的收敛半径$R > 0$,而且
    \[M = \max_{|z| = \rho}{|f(z)|}\ (\rho < R)\]
    则在圆盘$|z| < \dfrac{|a_0|}{|a_0| + M}$内,$f(z)$没有零点。

\end{proposition}

\begin{proof}

    由\textup{Cauchy}不等式知,$|a_n| \leq \dfrac{M}{\rho^n}$. 对于$|z| < \dfrac{|a_0|}{|a_0| + M}\ (\rho < R)$,有

    \begin{align*}
        |f(z) - a_0| & \leq |a_1||z| + |a_2||z|^2 + \cdots + |a_n||z|^n + \cdots \\
        & \leq \left( \dfrac{|z|}{\rho} + \dfrac{|z|^2}{\rho^2} + \cdots + \dfrac{|z|^n}{\rho^n} + \cdots \right) \\
        & = M \dfrac{|z|}{\rho - |z|} \\
        & < M \dfrac{\frac{|a_0|}{|a_0| + M}\rho}{\rho - \frac{|a_0|}{|a_0| + M}\rho} \\
        & = |a_0|
    \end{align*}

    因为
    \[|f(z)| = |f(z) - a_0 + a_0| \geq |a_0| - |f(z) - a_0| > 0\]
    所以$f(z)$没有零点。

\end{proof}

\begin{proposition}

    设$f(z)$是整函数,且$\lim\limits_{r \to \infty}{\dfrac{M(r)}{r^n}} < +\infty$,其中$M(r) = \max\limits_{|z| = r}{|f(z)|}$. 证明:$f(z)$是不高于$n$次的多项式。

\end{proposition}

\begin{proof}

    设$f(z) = \sum\limits_{n = 0}^{\infty}{c_n z^n}$,$|z| < + \infty$,由\textup{Cauchy}不等式得
    \[|c_k| \leq \dfrac{M(r)}{r^k}, k = 1, 2, \cdots\]
    下证当$k \geq n + 1$时,$c_k = 0$. 令$k = n + p \ (p \geq 1)$,由题设知
    \[\lim\limits_{r \to \infty}{\dfrac{M(r)}{r^k}} = \lim\limits_{r \to \infty}{\dfrac{1}{r^p} \dfrac{M(r)}{r^n}} = 0\]
    则$k \geq n + 1$时$c_k = 0$. 即$f(z)$是不高于$n$次的多项式。

\end{proof}

\section{共形映射}

\begin{proposition}

    设$f(z)$是单位圆盘$\mathbb{D} = \{z \big| |z| < 1 \}$到其自身的解析映射,且有$f(z) \neq z$. \\
    证明:$f(z)$至少有一个零点。

\end{proposition}

\begin{proof}
    
    反证法。若存在$a, b \in \mathbb{D}$,$a \neq b$,且
    \[f(a) = a, \quad  f(b) = b\]
    则设$\varphi(z) = \dfrac{a - z}{1 - \overline{a}z}$,于是$\varphi$将$\mathbb{D}$映射成$\mathbb{D}$,且有$\varphi(0) = a$. \\
    令$F(z) = \varphi^{-1} \circ f \circ \varphi(z)$,则有$F(0) = 0$,且
    \[F[(\varphi^{-1}(b))] = \varphi^{-1}(b)\]
    由\textup{Schwarz}引理知$F(z) = \euler^{\mathrm{i \theta}}z$,\\
    因为
    \[F[(\varphi^{-1}(b))] = \varphi^{-1}(b) = \euler^{\mathrm{i \theta}}\varphi^{-1}(b)\]
    所以$ \euler^{\mathrm{i \theta}} = 1$
    则
    \[ \varphi^{-1} \circ f \circ \varphi(z) = z\]
    即$ f[\varphi(z)] = \varphi(z)$, $f$为恒等映射,与题设矛盾,得证。

\end{proof}

\begin{proposition}
    
    设$f(z)$是单位圆盘$\mathbb{D} = \{z \big| |z| < 1 \}$到其自身的映射,且为解析函数,$f(0) = 0$. 证明:
    
    \begin{enumerate}
        
        \item   \[|f(z) + f(-z)| \leq 2 |z|^2 \quad (|z| < 1)\]
        
        \item   如果$\exists z_0 \neq 0$,使得上面不等式中的等号在$z = z_0$处成立,则
                \[f(z) = \euler^{\imag\theta}z^2(\theta \in \mathbb{{R}})\]

    \end{enumerate}

\end{proposition}

\begin{proof}
    
    \begin{enumerate}
        
        \item
            $f(z)$在$|z| < 1$内解析,则可以\textup{Taylor}展开,又因为$f(0) = 0$,\\
            所以
            \[ f(z) = \sum\limits_{n = 1}^{\infty}{a_nz^n}\]
            令$\varphi(z) = \dfrac{1}{2} (f(z) + f(-z)) = \sum\limits_{n = 1}^{\infty}{a_{2n}z^{2n}}$,
            显然$z = 0$是$\varphi(z)$的$k(k \geq 2)$级零点,\\
            所以
            \[|\varphi(z)| = |z|^2\]
            即
            \[|f(z) + f(-z)| \leq 2|z|^2\]

        \item 
            \[|\varphi(z) = \dfrac{1}{2} |f(z) + f(-z)| \leq \dfrac{1}{2} (|f(z)| + |f(-z)|) \leq 1\]
            由\textup{(1)}知,$\left| \dfrac{\varphi(z)}{z^2} \right| \leq 1$,且$\left| \dfrac{\varphi(z_0)}{z_0^2} \right| = 1, \quad |z_0| < 1$,则由\textup{Schwarz}引理得
            \[\varphi(z) = \euler^{\imag\theta} z^2\]
            即
            \[f(z) + f(-z) = 2\euler^{\imag\theta} z^2\]
            \[f(z) + f(-z) = \sum\limits_{n = 1}^{\infty}{a_{2n}z^{2n}}\]
            设$f(z) = \euler^{\imag\theta} z^2 + h(z)$,其中$h(z) = \sum\limits_{n = 1}^{\infty}{a_{2n - 1}z^{2n - 1}}$. \\
            显然有$h(z) = -h(-z)$,则由$|f(z)| \leq 1$得

            \begin{align*}
                &|\euler^{\imag\theta} z^2 + h(z)| \leq 1 \\
                &|\euler^{\imag\theta} z^2 - h(z)| \leq 1
            \end{align*}

            即有

            \begin{align*}
                &(\euler^{\imag\theta} z^2 + h(z)) \overline{(\euler^{\imag\theta} z^2 + h(z))} \leq 1 \\
                &(\euler^{\imag\theta} z^2 - h(z)) \overline{(\euler^{\imag\theta} z^2 - h(z))} \leq 1
            \end{align*}

            则
            \[|z|^4 + |H(z)| \leq 1 \]
            由最大模定理可知$h(z)\equiv0$,即
            \[f(z) = \euler^{\imag\theta} z^2 \]

    \end{enumerate}

\end{proof}

\begin{proposition}
    
    函数$f(z)$在可求面积得区域$D$内单叶解析,并且满足$|f(z)| \leq 1$. 证明:
    \[\iint_{D}{|f'(z)|^2}\diff x\diff y \leq  \pi\]

\end{proposition}

\begin{proof}
    
    $S = \iint_{D}{|f'(z)|^2}\diff x\diff y$为$f(z)$将$D$映射成的区域面积,又$|f(z)| \leq 1$,则显然有
    \[\iint_{D}{|f'(z)|^2}\diff x\diff y \leq  \pi\]

\end{proof}

\begin{proof}
    
    若$\lim\limits_{z \to z_0}{f(z)} = w_0$,其中$z_0, w_0 \in \mathbb{C}$,则
    \[\lim\limits_{z \to z_0}{|f(z)|} = |w_0|\]

\end{proof}

\begin{proposition}
    
    由极限定义知,$\forall \varepsilon > 0$,$\exists \delta > 0$,$|z - z_0| < \delta$时,$|f(z) - w_0 | < \varepsilon$. \\
    又因为
    \[\Big||a| - |b|\Big| \leq |a - b|\]
    则 
    \[\Big| |f(z) - |w| \Big| \leq | f(z) - w| < \varepsilon \]
    易知
    \[\lim\limits_{z \to z_0}{|f(z)} = |w_0|\]

\end{proposition}

\begin{proposition}
    
    设$f(z)$在$|z| \leq 1$内解析,在$|z| = 1$上有$|f(z)| > m$,并且$|f(0)| < m$,其中$m > 0$. 证明:$f(z)$在$|z| < 1$内至少有一个零点。

\end{proposition}

\begin{proof}
    
    反证法。假设$f(z)$在$|z| < 1$内无零点,又$|z| = 1$上$|f(0)| > m > 0$ \\
    则$f(z)$在$|z| \leq 1$内无零点,$\dfrac{1}{f(z)}$在$|z| \leq 1$解析,由最大模定理知
    \[\max_{|z| \leq 1}{\left| \dfrac{1}{f(z)} \right|} = \max_{|z| = 1}{\left| \dfrac{1}{f(z)} \right|}  < \dfrac{1}{m}\]
    然而$\left| \dfrac{1}{f(z)} \right| > \dfrac{1}{m}$,与题设矛盾,所以$f(z)$在$|z| < 1$内至少有一个零点。

\end{proof}

\begin{proposition}
    
    设单叶解析函数$f(z)$将$z$平面上可求面积的区域$D$映射成$w$平面上的区域$G$,设区域$G$的面积为$A$. 证明:
    \[ A = \iint_{D}{|f'(z)|^2}\diff x\diff y\]

\end{proposition}

\begin{proof}
    
    显然$A = \iint_{D}\diff x\diff y$,积分区域为$G$,则由积分换元公式知

    \begin{align*}
        A & = \iint_{G}\diff x\diff y \\
          & = \iint_{D}{\left| \dfrac{\partial(u, v)}{\partial(x, y)} \right|}\diff x\diff y \\
          & = \iint_{D}{|f'(z)|^2}\diff x\diff y
    \end{align*}

\end{proof}

\begin{proposition}
    
    设$f(z) = \sum\limits_{n = 0}^{\infty}{a_nz^n}$为单位圆盘$\mathbb{D} = \{z\big| |z| < 1\}$内的单叶解析函数,\\
    $G$为$f$将单位圆盘$\mathbb{D}$映射成的区域,$A$为区域$G$的面积。证明:
    \[A =  \pi\sum\limits_{n = 1}^{\infty}{n|a_n|^2}\]

\end{proposition}

\begin{proof}
    
    易证$A = \iint_{\mathbb{D}}{|f'(z)|}\diff x\diff y$,又$f'(z) = \sum\limits_{n = 1}^{\infty}{na_nz^{n - 1}}$,设$z = \euler^{\imag\theta}$,则有

    \begin{align*}
        A & = \iint_{\mathbb{D}}{|f'(z)|^2}\diff x\diff y \\ 
          & = \iint_{\mathbb{D}}{f'(z)\overline{{f'(z)}}}\diff x\diff y \\
          & = \iint_{\mathbb{D}}{\left( \sum\limits_{n = 1}^{\infty}{na_nz^{n - 1}} \right)\overline{\left( \sum\limits_{n = 1}^{\infty}{na_nz^{n - 1}} \right)}}\diff x\diff y \\
          & = \int_{0}^{1}{\int_{0}^{2 \pi}{\left( r\sum\limits_{n = 1}^{\infty}{na_nr^{n - 1}\euler^{(n - 1)\imag\theta}} \right)\left( \sum\limits_{n = 1}^{\infty}{n\overline{a_n}r^{n - 1}\euler^{-(n - 1)\imag\theta}} \right)}\diff r}\diff \theta \\
          & = \int_{0}^{1}{r\sum\limits_{n = 1}^{\infty}{2n^2|a_n|^2 \pi r^{2n - 2}}} \\
          & = \pi \sum\limits_{n = 1}^{\infty}{n|a_n|^2}
    \end{align*}

    其中
    $$
    \int_{o}^{2\pi}{\euler^{\imag n\theta}}\diff \theta = 
    \left\{
        \begin{aligned}
            &2\pi, & n = 0 \\
            &0, & n \neq 0 \\
        \end{aligned}
    \right.
    $$
    即
    \[A = \pi \sum\limits_{n = 1}^{\infty}{n|a_n|^2}\]

\end{proof}


\begin{theorem}[Area theorem]
    
    设$f(z) = \dfrac{1}{z} + c_0 + c_1 z + c_2 z
    ^2 + \cdots$在$0 < |z| < 1$内为单射且解析,则
    \[\sum\limits_{n = 1}^{\infty}{n|c_n|^2} \leq 1\]
    
\end{theorem}

\begin{proof}
    
    由题设知,$f(z)$将正向圆周$|z| = r\ (r < 1)$映射成简单闭曲线$\gamma_r$,即
    \[\gamma_r(\theta) = f(r\euler^{-\imag \theta}), \quad \theta \in [0, 2\pi]\]
    令$D_r$为简单闭曲线$\gamma$围成的有界区域,$S_r$为$D_r$的面积,易知,曲线$\gamma$正向包围$D_r$. \\
    因此
    \[S_r = \int_{\gamma_r}{x}\diff y - \int_{\gamma_r}{y}\diff x\]
    应用链式法则得
    \[S_r = \int_{0}^{2\pi}{\mathrm{Re}(f(r \euler^{-\imag \theta})) \mathrm{Im}(-\imag r \euler^{-\imag \theta} f'(r \euler^{-\imag \theta}))} \diff \theta = -\int_{0}^{2\pi}{\mathrm{Im}(f(r \euler^{-\imag \theta})) \mathrm{Re}(-\imag r \euler^{-\imag \theta} f'(r \euler^{-\imag \theta}))} \diff \theta\]
    因此$D$等于上式右边两项的平均值,即
    \[S_r = \dfrac{1}{2} \mathrm{Re}\int_{0}^{2\pi}{f(r \euler^{-\imag \theta}) \overline{r \euler^{-\imag \theta} f'(r \euler^{-\imag \theta})}} \diff \theta\]
    令$c_{-1} = 1$,幂级数展开得
    \[S_r = -\dfrac{1}{2}\mathrm{Re}\int_{0}^{2\pi}{\sum\limits_{n = -1}^{\infty}{\sum\limits_{n = -1}^{\infty}{m r^{n + m}a_n \overline{a_m} \euler^{\imag (m - n)\theta}}}}\diff \theta\]
    因为
    $$\int_{0}^{2\pi}{\euler^{\imag(m - n)\theta}}\diff \theta = \left\{
        \begin{aligned}
            & 2\pi, & m = n, \\
            & 0, & m \neq n,           
        \end{aligned}
    \right.$$
    所以
    \[S_r = - \pi\sum\limits_{n = -1}^{\infty}{nr^{2n}|a_n|^2} = \dfrac{\pi}{r^2} - \pi\sum\limits_{n = 1}^{\infty}{nr^{2n}|a_n|^2} \geq 0\]
    令$r \to 1$得
    \[\sum\limits_{n = 1}^{\infty}{n|a_n|^2} \leq 1\]
    下证$\gamma_r$正向围绕$D_r$. 设$r'$满足$r < r' < 1$,$z_0 = f(r')$. 存在足够小的正数$s$,使得$\gamma_s$环绕$z_0$,且环绕数为$1$. 当$t \neq r'$时,因为$f$是单射,所以$\gamma_t$不通过$z_0$. 因此由环绕数的同伦不变性可知,$\gamma_r$对$z_0$的环绕数为$1$,也即$z_0 \in D_r$,且$\gamma_r$正向环绕$D_r$.

\end{proof}

\begin{lemma}\label{lemma:complex}

    设解析函数$f(z) = z + a_2 z^2 + \cdots$为单射。证明:存在一个函数$g(z)$,满足$g^2(z) = f(z^2)$,$g(0) = 0$,$g'(0) = 1$,且$g(z)$为单射。

\end{lemma}

\begin{proof}

    显然$f(0) = 0$,$f'(0) = 1$,作$\dfrac{f(z)}{z}$没有零点,则存在函数$\varphi(z)$,使得$\varphi^2(z) = \dfrac{f(z)}{z}$. 令$g(z) = z\varphi(z^2)$,容易验证$g(0) = 0$,$g'(0) = 1$,且$g(z)$为单射。
    
\end{proof}

\begin{theorem}\label{theorem:
    injective1}

    设解析函数$f(z) = z + z_2 z^2 + \cdots$是单射,则$|a_2| \leq 2$,等号成立当且仅当
    \[f(z) = \dfrac{z}{(1 - \euler^{\imag \theta})^2}, \quad \theta \in \mathbb{R}\]

\end{theorem}

\begin{proof}

    由引理\ref{lemma:complex}得
    \[\dfrac{1}{g(z)} = \sqrt{\dfrac{1}{f(z^2)}} = \dfrac{1}{z} \sqrt{\dfrac{1}{1 + a_2 z^2 + a_3 z^3 + \cdots}} = \dfrac{1}{z} + b_0 + b_1 z + \cdots\]
    化简得
    \[1 = (1 + b_0 z + b_1 z^2 + \cdots)^2 (1 + a_2 z^2 + a_3 z^4 + \cdots)\]
    比较$z$,$z^2$系数可得$b_0 = 0$,$a_2 + 2 b_1 + b_0^2 = 0$,所以$b_1 = -\dfrac{a_2}{2}$. \\
    因此
    \[\dfrac{1}{g(z)} = \dfrac{1}{z} - \dfrac{a_2}{2}z + \cdots\]
    由引理\ref{lemma:complex}知$\dfrac{1}{g(z)}$单射,又由\textup{Area Theorem}得
    \[\left| \dfrac{a_2}{2} \right|^2 = |b_1|^2 \leq \sum\limits_{n = 1}^{\infty}{n|b_n|^2} \leq 1\]
    所以$|a_2| \leq 2$. \\
    下证等号成立的情况。如果等号成立即$|a_2| = 2$,则存在$\theta \in \mathbb{R}$,使得$a_2 = 2 \euler^{\imag \theta}$. 由\textup{Area Theorem}得,$\sum\limits_{n = 1}^{\infty}{n|b_n|^2} \leq 1$,可知其余系数$b_2 = b_3 = \cdots = 0$,因此
    \[\dfrac{1}{g(z)} = \dfrac{1}{z} - \euler^{\imag \theta}z\]
    由引理\ref{lemma:complex}知
    \[g^2(z) = \dfrac{1}{\left( \frac{1}{z} - \euler^{\imag \theta}z \right)^2} = \dfrac{z^2}{(1 - \euler^{\imag \theta} z^2)^2} = f(z^2)\]
    所以
    \[f(z) = \dfrac{z}{(1 - \euler^{\imag \theta})^2}, \quad \theta \in \mathbb{R}\]

\end{proof}

\begin{theorem}\label{theorem:injective2}

    设解析函数$f(z) = \dfrac{1}{z} + c_0 + c_1 z + c_2 z^2 + \cdots$是$\mathbb{D}$上的单射,且$h(z)$取不到$w_1$,$w_2$,则$|w_1 - w_2| \leq 4$.

\end{theorem}

\begin{proof}

    由题设得
    \[\dfrac{1}{f(z) - w_1} = \dfrac{z}{1 + (c_0 - w_1)z + c_1 z^2 + c_2 z^3 + \cdots}\]
    则
    \[z = (1 + (c_0 - w_1)z + c_1 z^2 + c_2 z^3 + \cdots) (b_0 + b_1 z + \cdots)\]
    比较系数得
    \[b_0 = 0, \ b_1 = 1, \ b_0 c_1 + b_1(c_0 - w_1) + b_2 = 0 \Rightarrow b_2 = w_1 - c_0\]
    因此函数$\varphi(z) = \dfrac{1}{f(z) - w_1}$满足$\varphi(0) = 0$,$\varphi'(0) = 1$,且为解析的单射。 \\
    由定理\ref{theorem:
    injective1}得,$|b_2| = |w_1 - c_0| \leq 2$,同理$|w_2 - c_0| \leq 2$. \\
    所以
    \[|w_1 - w_2| \leq |w_1 - c_0| + |c_0 - w_2| \leq 4\]
    
\end{proof}

\begin{theorem}[Koebe-Bieberbach定理]

    设$f: \mathbb{D} \to \mathbb{C}$是解析函数,$f(0) = 0$,$f'(0) = 1$,且$f$为单射,则必有$D\left( 0, \dfrac{1}{4} \right) \subset f(\mathbb{D})$,即$f(\mathbb{D})$包含以原点为圆心,半径为$\dfrac{1}{4}$的圆盘。

\end{theorem}

\begin{proof}

    设$f(z)$取不到$w$,则$\dfrac{1}{f(z)}$取不到$0$和$\dfrac{1}{w}$. \\
    由定理\ref{theorem:injective2}知$\left| 0 - \dfrac{1}{w} \right| \leq 4$, 即$|w| \geq 4$. \\
    因此$f(z)$取不到的值都位于圆盘$D\left( 0, \dfrac{1}{4} \right)$之外,所以$D\left( 0, \dfrac{1}{4} \right) \subset f(\mathbb{D})$.

\end{proof}

\begin{proposition}

    设$f:\mathbb{D} \to \mathbb{C}$时解析函数,$|f(z)| \leq 1$,并且$f(\alpha) = \beta$,$\alpha, \beta \in \mathbb{D}$,则
    \[|f'(\alpha)| \leq \dfrac{1 - |\beta|^2}{1 - |\alpha|^2}\]
    当且仅当等号成立时,有$f(z) = \varphi_{-\beta}(\varphi_{\alpha}(z))$. 其中
    \[\varphi_{\alpha} = \dfrac{z - \alpha}{1 - \overline{\alpha}{z}}, \quad \varphi_{\beta} = \dfrac{z - \beta}{1 - \overline{\beta}{z}}\]

\end{proposition}

\begin{proof}
    
    令$g = \varphi_{\beta}\circ f \circ \varphi_{-\alpha}$,于是$g$在$D$内解析,$|g(z)| \leq 1$,并且$g(0) = 0$,由\textup{Schwarz}引理得
    \[|g'(0)| \leq 1\]
    且
    
    \begin{align*}
        g'(0) & = \varphi_{\beta}'(\beta) f'(\alpha) \varphi_{-\alpha}(0) \\
        & =  (1 - |\beta|^2)^{-1}f'(\alpha (1 - |\alpha|^2))
    \end{align*}

    则有
    \[|f'(\alpha)| \leq \dfrac{1 - |\beta|^2}{1 - |\alpha|^2}\]

\end{proof}

\begin{proposition}
    
    设$f:\mathbb{D} \to \mathbb{C}$是解析函数,$f(0) = 0$,并且任意$z \in D$,$\mathrm{Re}(f(z)) \leq A$,\\
    其中$\mathbb{D} = \{z \big| |z| < 1\}$,$A$是正实数,则对任意$r \in (0, 1)$有
    \[M(r) \leq \dfrac{2Ar}{1 - r}\]
    其中$M(r) = \max\limits_{|z| = r}\{|f(z)|\}$

\end{proposition}

\begin{proof}
    
    $\forall r \in (0, 1)$,令
    \[A(r) = \max_{|z| = r}\{\mathrm{Re}f(z)\}\]
    因为
    \[\euler^{A(r)} = \max_{|z| = r}\{|\euler^{f(z)}|\}\]
    由最大模原理知$A(r)$是单调增加的非负函数,并且由假设知$A(r) \leq A$ \\
    令
    \[ g(z) = \dfrac{f(z)}{2A - f(z)} = \dfrac{P + Q\imag}{(2A - P) - Q\imag} \]
    其中
    \[ P = \mathrm{Re}f(z), \quad Q = \mathrm{Im}f(z) \]
    那么$f(z)$在$\mathbb{D}$内解析,对任意$x \in \mathbb{D}$有
    \[|g(z)|^2 = \dfrac{P^2 + Q^2}{(2A - P)^2 + Q^2} \leq \dfrac{R^2 + Q^2}{A^2 + Q^2} \leq 1\]
    从而$|g(z)| \leq 1$,$g(0) = 0$,于是由\textup{Schwarz}引理得
    \[|g(z) | \leq z\]
    由$g(z)$定义知
    \[f(z) = \dfrac{2Ag(z)}{1 + g(z)}\]
    则
    \[ |f(z)| \leq \dfrac{2A(z)}{1 - |z|} \leq \dfrac{2Ar}{1 - r}\]
    即
    \[M(r) \leq \dfrac{2Ar}{1 - r}\]

\end{proof}

\begin{lemma}[Schwarz引理]
    
    设$f(z)$在单位圆盘$\mathbb{D} = \{z \big| |z| < 1\}$内解析,在闭圆盘$\overline{\mathbb{D}} = \{z \big| |z| \leq 1\}$上是连续的,并且满足$f(0) = 0$,又在$\mathbb{D}$内处处有$|f(z)| < 1$,则

    \begin{enumerate}
        
        \item $|f(z)| \leq |z|$,且$|f'(0)| \leq 1$.
            
        \item 
            若在开圆盘内有一复数$z_0 \neq 0$,是$|f(z_0)| = |z_0|$,或者 $|f'(0)| = 1$,
            那么$f(z) = e^{\imag\alpha}z$,$\alpha$为一实数。
        
    \end{enumerate}

\end{lemma}

\begin{proof}
    
        
    由于$f(0) = 0$,且$f(z)$在$\mathbb{D}$内解析,则$f(z)$在$\mathbb{D}$内可以\textup{Taylor}展开,有
    \[f(z) = c_1 z + c_2 z^2 + \dots + \]
    令$\varphi(z) =  c_1 + c_2 z + \cdots $,则
    $$
    \varphi(z) = 
    \left\{
        \begin{aligned}
            & \dfrac{f(z)}{z}, & z \neq 0 \\
            &f'(0), & z = 0 \\
        \end{aligned}
    \right.
    $$
    显然$\varphi(z)$在$\mathbb{D}$内解析,对于圆内一点$z_0$,设$|z_0| < r < 1$,则由最大模定理知
    \[|\varphi(z + 0)| = \max_{|z| = r}|\varphi(z_0)| = \max_{|z| = r}{\left| \dfrac{f(z)}{z} \right|} \leq \dfrac{1}{r}\]
    令$r \to 1$,则有$|\varphi(z_0)| \leq 1$ \\
    若$z_0 = 0$,有$|f'(0)| = |\varphi(0)| \leq 1$ \\
    若$z_0 \neq 0$,有$|f'(0)| = \left| \dfrac{f(z_0)}{z_0} \right| \leq 1$,即
    \[|f(z_0)| \leq |z_0|\]
    由于$f(0) = 0$,上式当$z = 0$时成立,则
    \[|f(z_0)| \leq |z_0|, \quad x \in \mathbb{D}\]
    若有$z_0 \neq 0$,使$|f(z_0)| = |z_0| $,或者$|f'(0)| = 1$,则$\varphi(z)$在$\mathbb{D}$内取得最大模。\\
    从而$\varphi(z) \equiv c$, 又$|c| = 1$,则$\varphi(z) = \euler^{\imag\alpha}$,所以
    \[f(z) = \euler^{\imag\alpha}z\]

\end{proof}

\begin{theorem}[拟双曲型距离]

    定义拟双曲型距离
    \[\rho(z, w) = \left| \dfrac{z - w}{1 - \overline{w}z} \right|, \quad z, w \in \mathbb{D}\]
    设$f: \mathbb{D} \to \mathbb{D}$解析,则
    \[\rho(f(z), f(w)) \leq \rho(z, w), \quad \forall z, w \in \mathbb{D}\]
    另外,如果$f$是单位圆盘的自同构映射, 则f保持拟双曲型距离不变:
    \[\rho(f(z), f(w)) = \rho(z, w), \quad \forall z, w \in \mathbb{D}\]
    
\end{theorem}

\begin{proof}

    $\forall w \in \mathbb{D}$,记$g = \psi_{f(w)} \circ f \circ \psi^{-1}_{w}$,满足
    \[g(0) = \psi_{f(w)} \circ f \circ \psi^{-1}_{w}(0) = \psi_{f(w)} \circ f(w) = 0, \quad |g(z)| \leq 1,\ (|z| \leq 1)\]
    由\textup{Schwarz}引理知,
    \[|g(z)| = |\psi_{f(w)} \circ f \circ \psi^{-1}_{w}(z)| \leq |z|\]
    取$z = \psi_{w}(u)$,则$|\psi_{f(w)} \circ f(u)| \leq |\psi_{w}(w)|$,即
    \[\rho(f(u), f(w)) \leq \rho(w, w), \quad \forall w \in \mathbb{D}\]
    若f是单位圆盘的自同构映射, 则$g^{-1}$有定义, 且满足$g^{-1}(0) = 0$,$|g^{-1}(z)| \leq 1$,仿照上面步骤可得到
    \[|(\psi_{f(w)} \circ f \circ \psi^{-1}_{w}(z))^{-1}| \leq |z|\]
    从而
    \[\rho(w, w) \leq \rho(f(u), f(w)), \quad \forall w \in \mathbb{D}\]
    综上可知
    \[\rho(f(z), f(w)) = \rho(z, w), \quad \forall z, w \in \mathbb{D}\]

\end{proof}

\begin{theorem}[Schwarz-Pick定理\romannum{1}]

    设$f:\mathbb{D} \to \mathbb{C}$在$D$内解析,$|f(z)| \leq 1$,$z \in \mathbb{D}$,且$f(z_0) = w_0$. 则
    \[ \left| \dfrac{f(z) - w_0}{1 - \overline{w_0}f(z)} \right| \leq \left| \dfrac{z - z_0}{1 - z_0z} \right|\]
    等号当且仅当$f(z)$是分式线性映射时成立。

\end{theorem}

\begin{proof}
    
    设函数$w = f(z)$,考虑分式线性映射
    \[\zeta = T(z) = \dfrac{z - z_0}{1 - \overline{z_0}z}, \quad \tau = S(w) = \dfrac{w - w_0}{1 - \overline{w_0}w}\]
    分别$z$平面和$w$平面上的单位圆盘双方单值映射成$\zeta$和$\tau$平面上的单位圆盘,并且把$z_0, w_0$映射成$\zeta = 0$,$\tau = 0$. 则
    \[F(\zeta) = S(f(T^{-1}(\zeta)))\]
    在$|\zeta| < 1$内解析,$|F(\zeta)| \leq 1$,且$F(0) = 0$. \\
    于是在$|\zeta| < 1$内,由\textup{Schwarz}引理得
    \[|F(\zeta)| = |S((f(T^{-1}(\zeta))))| \leq |\zeta|\]
    从而在$|z| < 1$内有
    \[|S(f(z))| \leq |T(z)|\]
    即
    \[ \left| \dfrac{f(z) - w_0}{1 - \overline{w_0}f(z)} \right| \leq \left| \dfrac{z - z_0}{1 - z_0z} \right|\]

\end{proof}

\begin{theorem}[Schwarz-Pick定理\romannum{2}]

    设$f: \mathbb{D} \to \mathbb{D}$解析,则
    \[\dfrac{|f'(z)|}{1 - |f(z)|^2} \leq \dfrac{1}{1 - |z|^2}, \quad \forall z \in \mathbb{D}\]

\end{theorem}

\begin{proof}
    
    由拟双曲型距离定理可知
    \[\left| \dfrac{f(z) - f(w)}{1 - \overline{f(w)}f(z)} \right| \leq \left| \dfrac{z - w}{1 - \overline{w}z} \right|\]
    整理得
    \[\left| \dfrac{f(z) - f(w)}{z - w} \dfrac{1}{1 - \overline{f(w)}f(z)} \right| \leq \left| \dfrac{1}{1 - \overline{w}z} \right|\]
    令$w \to z$,并利用$\alpha \overline{\alpha} = | \alpha|^2$,可得
    \[\dfrac{|f'(z)|}{1 - |f(z)|^2} \leq \dfrac{1}{1 - |z|^2}, \quad \forall z \in \mathbb{D}\]
    
\end{proof}
