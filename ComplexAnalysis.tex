\chapter{复分析}

\section{复数}

\begin{proposition}
    
    证明:对于$|z| \leq 1, \ |w| \leq 1$,有
    $$|w - z| \leq |1 - \overline{w}z|$$

\end{proposition}

\begin{proof}

    设$z = x_1 + y_1\imag,\ w = x_2 + y_2\imag$
    $$|w - z| = | (x_2 - x_1) + (y_2 - y_1)\imag| $$
    $$|1 - \overline{w}z| = |1 - (x_2 - y_2\imag) (x_1 - y_1\imag)| = | (1 - x_1x_2 - y_1y_2) + (x_1y_2 - x_2y_1) |$$

    \begin{align*}
        &|w - z|^2 - |1 - \overline{w}z|^2 \\
        = \ & x_1^2 + x_2^2 + y_1^2 + y_2^2 - 1 - x_1^2x_2^2 -  y_1^2y_2^2 - x_1^2 y_2^2 -  x_2^2y_1^2 \\
        = \ & (1 - x_1^2 - y_1^2) (1 - x_2^2 - y_2^2) \\
        \leq \ & 0
    \end{align*}

    其中
    $$0 \leq x_1^2 + y_1^2 \leq 1, \quad 0 \leq x_2^2 + y_2^2 \leq 1$$
    则
    $$|w - z| \leq |1 - \overline{w}z|$$

\end{proof}

\section{复积分}

\begin{proposition}

    证明:对任意$ \xi \in \mathbb{C}$,有
    $$\euler^{-\pi\xi^2} = \int_{-\infty}^{+\infty}{\euler^{-\pi x^2}\euler^{-2\pi \imag x\xi}}\diff x$$
    
\end{proposition}

\begin{proof}

    $$\euler^{-\pi x^2}\euler^{-2\pi x\xi} = \euler^{-\pi(x^2 + 2\imag x\xi)}$$
    因为
    $$-\pi(x^2 + 2\imag x\xi) = -\pi (x^2 + 2\pi\imag x - \xi^2 + \xi^2) = -\pi(x + \imag \xi)^2 - \pi\xi^2$$
    则
    $$\int_{-\infty}^{+\infty}{\euler^{-\pi x^2}\euler^{-2\pi \imag x\xi}}\diff x = \euler^{\pi\xi^2} \int_{-\infty}^{+\infty}{\euler^{-\pi (x+\imag\xi)^2}}\diff x$$
    令$z = x + \imag\xi, \diff x = \diff z$,则有

    \begin{align*}
        &\euler^{\pi\xi^2} \int_{-\infty}^{+\infty}{\euler^{-\pi (x+\imag\xi)^2}}\diff x \\
        = \ & \euler^{\pi\xi^2} \int_{-\infty}^{+\infty}{\euler^{-\pi z^2}}\diff x \\
        = \ & \euler^{\pi\xi^2} \dfrac{\sqrt{\pi}}{\sqrt{\pi}} \\
        = \ & \euler^{\pi\xi^2}
    \end{align*}

\end{proof}

\begin{proposition}

    证明:
    $$\left| \int_{\gamma}{\euler^{\imag z^2}}\diff z \right| \leq \dfrac{\pi(1 - \euler^{-r^2})}{4r}$$
    其中$\gamma(t) = r\euler^{\imag t},\ t\in \left[0,\dfrac{\pi}{4} \right],\ r \in \mathbb{R}^+$

\end{proposition}

\begin{proof}

    \begin{align*}
        \left| \int_{\gamma}{\euler^{\imag z^2}}\diff z \right| & = \left| \int_{0}^{\frac{\pi}{4}}{\euler^{\imag r^2(\cos{2t} + \imag\sin{2t})}\imag r\euler^{\imag t}}\diff t \right| \\
        & \leq \int_{0}^{\frac{\pi}{4}}{\left| \euler^{\imag r^2(\cos{2t} + \imag\sin{2t})}\imag r\euler^{\imag t} \right|}\diff t \\
        & = \int_{0}^{\frac{\pi}{4}}{r\euler^{-r^2\sin{2t}}}\diff t
    \end{align*}

    因为$\sin{2t} \geq \dfrac{4t}{\pi}, \ t \in \left[0,\dfrac{\pi}{4} \right)$,则
    $$\int_{0}^{\frac{\pi}{4}}{r\euler^{-r^2\sin{2t}}}\diff t \leq \int_{0}^{\frac{\pi}{4}}{r\euler^{\frac{4r^2t}{\pi}}}\diff t = \dfrac{\pi}{4r}(1 - \euler^{-r^2}) $$

\end{proof}

\begin{proposition}

    设$f(z)$在复平面上处处解析,并且不等式
    $$\int_{0}^{2\pi}{|f(\mathrm{e^{\imag \theta}})|}\diff \theta \leq r^{\frac{16}{5}} $$
    对所有$r > 0$成立,证明:$f(z) \equiv 0$

\end{proposition}

\begin{proof}

    由题设$f(z)$在复平面$\mathrm{C}$内都能\textup{Taylor}展开,则
    $$f(z) = \sum\limits_{n=0}^{\infty}{\dfrac{f^{(n)}(z_0)}{n!}(z - z_0)^n}$$
    应用\textup{Cauthy}积分公式,得
    $$f(0) = \dfrac{1}{2\pi} \int_{0}^{2\pi}{f(\euler^{\imag \theta})}\diff \theta$$
    从而$2\pi|f(0)| \leq 6{\frac{16}{5}}$,令$r\to0^{+}$,有$f(0) = 0$
    再由
    $$f'(0) = \dfrac{1}{2\pi} \int_{0}^{2\pi}{\dfrac{f(\euler^{\imag \theta})}{r\euler^{\imag \theta}}}\diff \theta$$
    则可知$f'(0) = 0$,同理$f''(0) = f^{(3)}(0) = 0$\\
    当$n \geq 4$时
    $$f^{(n)}(0) =  \dfrac{n!}{2\pi} \int_{0}^{2\pi}{\dfrac{f(\euler^{\imag\theta})}{r^n \euler^{\imag\theta}}}\diff \theta$$
    $$|f^{(n)}(0)| \leq \dfrac{n!}{2\pi} \dfrac{r^{\frac{16}{5}}}{r^n} \to 0\quad(r\to+\infty)$$
    从而
    $$f^{(n)}(0) = 0,\quad \forall n \in \mathbb{N}$$
    即
    $$f(z) \equiv 0$$
    
\end{proof}

\section{复级数}

\begin{proposition}

    设$\lim\limits_{n\to\infty}{a_nz^n}$收敛,证明:
    $$\lim\limits_{r \to 1,\ r < 1}{\sum\limits_{n=1}^{\infty}{a_nr^n}} = \sum\limits_{n=1}^{\infty}{a_n}$$

\end{proposition}

\begin{proof}
    
    记$A_k = \sum\limits_{n = 1}^{k}{A_n},\ A_0 = 0$,则

    \begin{align*}
        \sum\limits_{n=1}^{N}{(1-r^n)a_n} & = (1-r^N)A_N + \sum\limits_{n=1}^{N}{(r^n-r^{n+1})A_n}\\
        & = (1 - r)A_N + \sum\limits_{n=1}^{N}{(r^n-r^{n+1})(A_N - A_n)}
    \end{align*}

    因为$\lim\limits_{r \to 1,\ r < 1}{(1 - r)A_N} = 0$,所以只需证$\lim\limits_{r \to 1,\ r < 1}{(r^n-r^{n+1})(A_N - A_n)} = 0$\\
    对任意$\varepsilon > 0 $,$\exists N' \in \mathbb{N}$,$n > N'$时
    $$|A_n - A| < \varepsilon $$
    其中$A = \lim\limits_{n\to\infty}{A_N}$\\
    取$(1-\varepsilon)^{\frac{1}{N'}} < r < 1$,则$0 < 1 - r < 1 - r^{N'} < \varepsilon $\\
    由于$A_n$有界,设$|A_n| \leq M $,则$n > N'$时
    
    \begin{align*}
        \sum\limits_{n=1}^{N-1}{(r^n - r^{n+1})(A_N - A_n)} & \leq (1 - r)\sum\limits_{n=1}^{N'}{r^n|A_n - A|} + \sum\limits_{n=N'+1}^{N-1}{r^n(|A_n - A| + |A_N -A|)}\\
        & \leq (\sum\limits_{n=1}^{N'}{2r^nM} + \sum\limits_{n=N'+1}^{N-1}{r^n2\varepsilon})\\
        & = (1 - r)\left(2M\dfrac{r - r^{N'+1}}{1 - r} + 2\varepsilon r^{N'+1}\dfrac{1 - R^{N-N'+1}}{1 - r}\right)\\
        & < 2M\varepsilon + 2\varepsilon\\
        & = 2(M+1)\varepsilon
    \end{align*}

    令$n\to+\infty$,再$r\to1$\\
    即
    $$\lim\limits_{r \to 1,\ r < 1}{\sum\limits_{n=1}^{\infty}{a_nr^n}} = \sum\limits_{n=1}^{\infty}{a_n}$$

\end{proof}

\begin{proposition}
    
    设$f$在$\mathbb{C}$上解析,且满足对于任一点$z_0\in\mathbb{C}$,$f$在$z_0$的\textup{Tarloy}展开式
    $$f(z) = \sum\limits_{n=0}^{+\infty}{a_n(z-z_0)}$$
    满足至少有一个$c_n$为$0$,证明$f$是多项式。

\end{proposition}

\begin{proof}
    
    考虑开单位圆盘$D = \Delta(0,1)$,则$D$内有不可数的点,所以存在正整数$p$,使得$D$内有无穷多点$\{z_n\}$,
    在这些点处的\textup{Taylor}展开式中第$p$项系数$c_p=0$\\
    可以在这些点中选取一个收敛子列$\{z_{n_k}\}$,且该子列收敛到$z_0\in D$,
    由唯一性定理知,$f^{(p)}\equiv 0$在$\mathbb{C}$上恒成立,故$f$为多项式。

\end{proof}

\section{解析函数}

\iffalse
\begin{proposition}
    
    设$f$在一个包含单位圆盘的开集上(除去单位圆盘上的一个极点$z_0$)解析,证明:\\
    若$\lim\limits_{n=1}^{\infty}{a_nz^n}$表示$f$在开单位圆盘上的\textup{Taylor}级数,那么
    $$\lim\limits_{n\to\infty}{\dfrac{a_n}{a_{n+1}}} = z_0$$

\end{proposition}

\begin{proof}
    
    考虑$f(z_0z)$的相同问题,则只需证明$z_0 = 1$的情况\\
    设$g(z)$是$f(z)$在$z=1$处\textup{Taylor}级数的主要部分,即
    $$g(z) = \sum\limits_{p=1}^{k}{\dfrac{A_p}{(1-z)^p}},\quad p \geq 1$$
    因为$h(z) = \dfrac{1}{1-z} = \sum\limits_{n=0}^{\infty}{z^n}$,则$h(z)$的$p-1$阶导数
    
    \begin{align*}
        (1-z)^{(p-1)} & = \dfrac{(p-1)!}{(1-z)^p}\\
        & = (\sum\limits_{n=1}^{\infty}{z_n})^{(p-1)}\\
        & = \sum\limits_{n=0}^{\infty}{(n+1)(n+2)\cdots(n+p+1)z^n}
    \end{align*}

    设$g(z) = \sum\limits_{n=1}^{\infty}{b_nz^n}$,有
    $$b_n = \sum\limits_{p=1}^{k}{\dfrac{1}{(p-1)!}(n+1)(n+2)\cdots(n+p+1)}$$
    则$\lim\limits_{n\to\infty}{b_n} = $

\end{proof}

\fi

\begin{proposition}

    已知$D$是单位圆盘,假设$f:D\to\mathbb{C}$是解析函数,证明:\\
    函数$f$的直径$d =\sup\limits_{z,\ w\in D}{|f(z) - f(w)|}$满足
    $$2|f'(0)| \leq d$$

\end{proposition}

\begin{proof}

    由\textup{Cauthy}公式得,
    $$f'(0) = \dfrac{1}{2\mathrm{\pi i}} \oint_{|z| = r}{\dfrac{f(z)}{z^2}}\diff z, \quad 0 < r < 1 $$
    $$f'(0) = \dfrac{1}{2\mathrm{\pi i}} \oint_{|z| = r}{\dfrac{-f(-z)}{z^2}}\diff z, \quad 0 < r < 1 $$
    即
    $$2f'(0) = \dfrac{1}{2\mathrm{\pi i}} \oint_{|z| = r}{\dfrac{f(z)-f(-z)}{z^2}}\diff z, \quad 0 < r < 1$$

    \begin{align*}
        2|f'(0)| & = \dfrac{1}{2 \pi} \left| \oint_{|z| = r}{\dfrac{f(z)-f(-z)}{z^2}}\diff z \right| \\
        & \leq \dfrac{1}{2 \pi} \oint_{|z|=r}{\left|\dfrac{f(z)-f(-z)}{z^2}\right|}\diff z\\
        & \leq \dfrac{1}{2 \pi} \dfrac{d}{r^2} \cdot 2 \pi r\\
        & = \dfrac{d}{r}, \quad 0 < r < 1
    \end{align*}

    则显然有$2|f'(0)| \leq d$
    再证等号成立时的情况。充分性显然。
    % 必要性空缺
    % TODO

\end{proof}

\begin{proposition}
    
    设$f$在$0 < |z| < R$解析,$\exists M >0$,使得任意$r$,$0 < r < R$,有
    $$r \int_{0}^{2 \pi}{|2f(\mathrm{e^{i\theta}})|}\diff \theta < M$$
    证明:$z=0$是一个可去奇点,否则是简单极点。

\end{proposition}

\begin{proof}
    
    设$r_2 < R$,选取$z_0$,满足$0 < z_0 < \dfrac{r_2}{2}$,再选取$r_1$,使$0 < r_1 < \dfrac{|z_0|}{2}$\\
    由\textup{Cauthy}积分公式得

    \begin{align*}
        |f(z_0)| & = \left| \dfrac{1}{2\mathrm{\pi i}} \int_{C_2}{\dfrac{f(z)}{z-z_0}}\diff z - \int_{C_1}{\dfrac{f(z)}{z-z_0}}\diff z \right|\\
        & \leq \dfrac{1}{2 \pi} \left(\int_{C_2}{\left|\dfrac{f(z)}{z-z_0}\right|}|\diff z| + \int_{C_1}{\left|\dfrac{f(z)}{z-z_0}\right|}|\diff z|\right)
    \end{align*}

    其中$C_1 = r_1\mathrm{e^{i\theta}},\ C_2 = r_2\mathrm{e^{i\theta}}$
    由$|z-z_0| \geq \dfrac{r_2}{2},\ z \in C_2$有
    $$ \dfrac{1}{2 \pi} \int_{C_2}{\left| \dfrac{f(z)}{z-z_0}\right|}|\diff z| \leq \dfrac{1}{ \pi r_2}\int_{0}^{2 \pi}{|f(\euler^{\imag\theta})}\diff \theta \leq \dfrac{M}{ \pi r_2^2}$$
    由$|z-z_0| \geq \dfrac{|z|}{2},\ z \in C_1$有
    $$ \dfrac{1}{2 \pi} \int_{C_1}{\left| \dfrac{f(z)}{z-z_0}\right|}|\diff z| \leq \dfrac{1}{ \pi|z|}\int_{0}^{2 \pi}{|f(\euler^{\imag\theta})}\diff \theta \leq \dfrac{M}{ \pi|z|}$$
    因此,我们有
    % 似有不对
    $$|f(z)| \leq \dfrac{M}{ \pi r_2} + \dfrac{M}{ \pi|z|},\quad 0 < |z| < \dfrac{r_2}{2}$$
    所以$$|zf(z)| \leq \dfrac{M|z|}{ \pi r_2} + \dfrac{M}{\pi}$$
    则极限是有限的。\\
    $z=0$是可去极点,极限为$0$.\\
    $z=0$是简单极点,极限非$0$.

\end{proof}

\begin{proposition}

    设$f(z)$在$|z| < 1 $内解析,并且$z = 0$为$f(z)$的$n(n \geq 1)$级零点,当$|z|<1$时,$|f(z)| < 1$.\\
    证明:当$|z|<1$时,$|f(z)| \leq |z|^n$

\end{proposition}

\begin{proof}
    
    $f(z)$在$|z|<1$内解析,且$z = 0$为$n(n \geq 1)$级零点,则有
    $$f(z) = \sum\limits_{m = n}^{\infty}{a_mz^m}$$
    显然$\varphi(z) = \dfrac{f(z)}{z^{n-1}}$在$|z|<1$内解析,且$\varphi(0) = 0$,有
    $$ \max\varphi(z) \leq \max_{|z| = r}|\varphi(z)| = \dfrac{|f(z)|}{r^{n-1}} \leq \dfrac{1}{r^{n-1}} $$
    令$n \to 1$,则$|\varphi(z)| < 1$,于是$|\varphi(z)| \leq |z|$,即
    $$|f(z)| \leq z^n$$

\end{proof}

\section{共形映射}

\begin{proposition}

    设$f(z)$是单位圆盘$\Delta(0,1) = \{z \big| |z| < 1 \}$到其自身的解析映射,且有$f(z) \neq z$\\
    证明:$f(z)$至少有一个零点。

\end{proposition}

\begin{proof}
    
    反证法。若存在$a,\ b\in \Delta(0,1), \quad a \neq b$,且
    $$f(a) = a, \quad  f(b) = b$$
    则设$\varphi(z) = \dfrac{a-z}{1-\overline{a}z}$,$\varphi$将$\Delta(0,1)$映射成$\Delta(0,1)$,且$\varphi(0) = a $\\
    令$F(z) = \varphi^{-1} \circ f \circ \varphi(z)$,则有$F(0) = 0$,
    $$F[(\varphi^{-1}(b))] = \varphi^{-1}(b)$$
    由\textup{Schwarz}引理知$F(z) = \euler^{\mathrm{i \theta}}z$,\\
    因为
    $$F[(\varphi^{-1}(b))] = \varphi^{-1}(b) = \euler^{\mathrm{i \theta}}\varphi^{-1}(b)$$
    所以$ \euler^{\mathrm{i \theta}} = 1$
    则
    $$ \varphi^{-1} \circ f \circ \varphi(z) = z$$
    即$ f[\varphi(z)] = \varphi(z)$, $f$为恒等映射,与题设矛盾,得证。

\end{proof}

\begin{proposition}
    
    设$f(z)$是单位圆盘$\Delta(0,1) = \{z \big| |z| < 1 \}$到其自身的映射,且为解析函数,$f(0) = 0$,证明:
    
    \begin{enumerate}
        
        \item   $$|f(z) + f(-z)| \leq 2 |z|^2 \quad (|z| < 1)$$
        
        \item   如果$\exists z_0 \neq 0$,使得上面不等式中的等号在$z_0$处成立,则
                $$f(z) = \euler^{\imag\theta}z^2(\theta \in \mathbb{{R}})$$

    \end{enumerate}

\end{proposition}

\begin{proof}
    
    \begin{enumerate}
        
        \item
                $f(z)$在$|z|<1$内解析,则可以\textup{Taylor}展开,又因为$f(0) = 0$,所以
                $$ f(z) =\sum\limits_{n=1}^{\infty}{a_nz^n}$$
                令$\varphi(z) = \dfrac{1}{2} (f(z) + f(-z)) = \sum\limits_{n=1}^{\infty}{a_{2n}z^{2n}}$,
                显然$z=0$是$\varphi(z)$的$k(k\geq 2)$级零点,所以
                $$|\varphi(z)| = |z|^2$$
                即
                $$|f(z) + f(-z)| \leq 2|z|^2$$

        \item 
                $|\varphi(z) = \dfrac{1}{2} |f(z) + f(-z)| \leq \dfrac{1}{2} (|f(z)| + |f(-z)|) \leq 1$
                由\textup{(1)}知,$\left|\dfrac{\varphi(z)}{z^2}\right| \leq 1$,且$\left|\dfrac{\varphi(z_0)}{z_0^2}\right| = 1,\quad |z_0|<1$,则由\textup{Schwarz}引理得\\
                $\varphi(z) = \euler^{\imag\theta} z^2$,即$f(z) + f(-z) = 2\euler^{\imag\theta} z^2$
                $$f(z) + f(-z) = \sum\limits_{n=1}^{\infty}{a_{2n}z^{2n}}$$
                设$f(z) = \euler^{\imag\theta} z^2 + h(z)$,其中$h(z) = \sum\limits_{n=1}^{\infty}{a_{2n-1}z^{2n-1}}$\\
                显然有$h(z) = -h(-z)$,则由$|f(z)| \leq 1$得

                \begin{align*}
                    &|\euler^{\imag\theta} z^2 + h(z)| \leq 1\\
                    &|\euler^{\imag\theta} z^2 - h(z)| \leq 1
                \end{align*}

                即有

                \begin{align*}
                    &(\euler^{\imag\theta} z^2 + h(z)) \overline{(\euler^{\imag\theta} z^2 + h(z))} \leq 1 \\
                    &(\euler^{\imag\theta} z^2 - h(z)) \overline{(\euler^{\imag\theta} z^2 - h(z))} \leq 1
                \end{align*}

                则
                $$|z|^4 + |H(z)| \leq 1 $$
                由最大模定理可知$h(z)\equiv0$,即
                $$f(z) = \euler^{\imag\theta} z^2 $$

    \end{enumerate}

\end{proof}

\begin{proposition}
    
    函数$f(z)$在可求面积得区域$D$内单叶解析,并且满足$|f(z)| \leq 1$,证明:
    $$\iint_{D}{|f'(z)|^2}\diff x\diff y \leq  \pi$$

\end{proposition}

\begin{proof}
    
    $S = \iint_{D}{|f'(z)|^2}\diff x\diff y$为$f(z)$将$D$映射成的区域面积,又$|f(z)| \leq 1$,则显然有
    $$\iint_{D}{|f'(z)|^2}\diff x\diff y \leq  \pi$$

\end{proof}

\begin{proof}
    
    若$\lim\limits_{z\to z_0}{f(z)} = w_0$,其中$z_0,\ w_0 \in \mathbb{C}$,则
    $$\lim\limits_{z\to z_0}{|f(z)|} = |w_0|$$

\end{proof}

\begin{proposition}
    
    由极限定义,$\forall \varepsilon > 0$,$\exists \delta > 0$,$|z - z_0| < \delta$时
    $$|f(z) - w_0 | < \varepsilon$$
    又$\Big||a| - |b|\Big| \leq |a -b|$,则 
    $$\Big| |f(z) - |w| \Big| \leq | f(z) - w| < \varepsilon $$
    易知
    $$\lim\limits_{z\to z_0}{|f(z)} = |w_0|$$

\end{proposition}

\begin{proposition}
    
    设$f(z)$在$|z|\leq 1$内解析,在$|z| = 1$上有$|f(z)| > m$,并且$|f(0)| < m$,其中$m > 0$\\
    证明:$f(z)$在$|z|< 1$内至少有一个零点。

\end{proposition}

\begin{proof}
    
    反证法。假设$f(z)$在$|z| < 1$内无零点,又$|z|= 1$上$|f(0)| > m >0$\\
    则$f(z)$在$|z|\leq 1$内无零点,$\dfrac{1}{f(z)}$在$|z|\leq 1$解析,由最大模定理知
    $$\max_{|z| \leq 1}{\left|\dfrac{1}{f(z)}\right|} = \max_{|z|=1}{\left|\dfrac{1}{f(z)}\right|}  < \dfrac{1}{m}$$
    然而$\left|\dfrac{1}{f(z)}\right| > \dfrac{1}{m}$,与题设矛盾,所以$f(z)$在$|z|< 1$内至少有一个零点。

\end{proof}

\begin{proposition}
    
    设$f(z) = \sum\limits_{n=0}^{\infty}{a_nz^n}$为单位圆盘$D = \{z\big| |z| < 1\}$内的单叶解析函数,\\
    $G$为$f$将单位圆$D$映射成的区域,$A$为区域$G$的面积,证明:
    $$A =  \pi\sum\limits_{n=1}^{\infty}{n|a_n|^2}$$

\end{proposition}

\begin{proof}
    
    易证$A = \iint_{D}{|f'(z)|}\diff x\diff y$,又$f'(z) = \sum\limits_{n=1}^{\infty}{na_nz^{n-1}}$,设$z = \euler^{\imag\theta}$,则有

    \begin{align*}
        A & = \iint_{D}{|f'(z)|^2}\diff x\diff y \\ 
          & = \iint_{D}{f'(z)\overline{{f'(z)}}}\diff x\diff y \\
          & = \iint_{D}{\left(\sum\limits_{n=1}^{\infty}{na_nz^{n-1}}\right)\overline{\left(\sum\limits_{n=1}^{\infty}{na_nz^{n-1}}\right)}}\diff x\diff y \\
          & = \int_{0}^{1}{\int_{0}^{2 \pi}{\left(r\sum\limits_{n=1}^{\infty}{na_nr^{n-1}\euler^{(n-1)\imag\theta}}\right)\left(\sum\limits_{n=1}^{\infty}{n\overline{a_n}r^{n-1}\euler^{-(n-1)\imag\theta}}\right)}\diff r}\diff \theta \\
          & = \int_{0}^{1}{r\sum\limits_{n=1}^{\infty}{2n^2|a_n|^2 \pi r^{2n-2}}} \\
          & = \pi \sum\limits_{n=1}^{\infty}{n|a_n|^2}
    \end{align*}

    其中
    $$
    \int_{o}^{2\pi}{\euler^{\imag n\theta}}\diff \theta = 
    \left\{
        \begin{aligned}
            &2\pi, & n=0\\
            &0, & n \neq 0\\
        \end{aligned}
    \right.
    $$
    即
    $$A = \pi \sum\limits_{n=1}^{\infty}{n|a_n|^2}$$

\end{proof}

\begin{proposition}

    设$f:D\to\mathbb{C}$时解析函数,$|f(z)| \leq 1$,并且$f(\alpha) = \beta,\ \alpha,\ \beta \in D$,则
    $$|f'(\alpha)| \leq \dfrac{1-|\beta|^2}{1-|\alpha|^2}$$
    当且仅当等号成立时,有$f(z) = \varphi_{-\beta}(\varphi_{\alpha(z)}),\ c \in \mathbb{C}$,且$|c| = 1$,其中
    $$\varphi_{\alpha} = \dfrac{z -\alpha}{1 - \overline{\alpha}{z}},\ \varphi_{\beta} = \dfrac{z -\beta}{1 - \overline{\beta}{z}}$$

\end{proposition}

\begin{proof}
    
    令$g = \varphi_{\beta}\circ f \circ \varphi_{-\alpha}$,于是$g$在$D$内解析,$|g(z)| \leq 1$,并且$g(0) = 0$,由\textup{Schwarz}引理得
    $$|g'(0)| \leq 1$$
    且
    
    \begin{align*}
        g'(0) & = \varphi_{\beta}'(\beta) f'(\alpha) \varphi_{-\alpha}(0)\\
        & =  (1 -|\beta|^2)^{-1}f'(\alpha (1 - |\alpha|^2))
    \end{align*}

    则有
    $$|f'(\alpha)| \leq \dfrac{1-|\beta|^2}{1 - |\alpha|^2}$$

\end{proof}

\begin{proposition}
    
    设单叶解析函数$f(z)$将$z$平面上可求面积的区域$D$映射成$w$平面上的区域$G$,设区域$G$的面积为$A$,证明:
    $$ A = \iint_{D}{|f'(z)|^2}\diff x\diff y$$

\end{proposition}

\begin{proof}
    
    显然$A = \iint_{D}\diff x\diff y$,积分区域为$G$,则由积分换元公式知

    \begin{align*}
        A & = \iint_{G}\diff x\diff y \\
          & = \iint_{D}{\left|\dfrac{\partial(u,v)}{\partial(x,y)}\right|}\diff x\diff y \\
          & = \iint_{D}{|f'(z)|^2}\diff x\diff y
    \end{align*}

\end{proof}

\begin{proposition}
    
    设$f:D\to\mathbb{C}$是解析函数,$f(0) = 0$,并且任意$z \in D$,$\mathrm{Re}(f(z)) \leq A$,\\
    其中$D = \{z \big| |z| < 1\}$,$A$是正实数,那么任意$r \in (0,1)$
    $$M(r) \leq \dfrac{2Ar}{1-r}$$
    其中$M(r) = \max\limits_{|z|=r}\{|f(z)|\}$

\end{proposition}

\begin{proof}
    
    $\forall r \in (0,1)$,令
    $$A(r) = \max_{|z|=r}\{\mathrm{Re}f(z)\}$$
    因为
    $$\euler^{A(r)} = \max_{|z|=r}\{|\euler^{f(z)}|\}$$
    由最大模原理知$A(r)$是单调增加的飞赴函数,并且由假设知$A(r) \leq A$\\
    令
    $$ g(z) = \dfrac{f(z)}{2A - f(z)} = \dfrac{P + Q\imag}{(2A-P)-Q\imag} $$
    其中
    $$ P = \mathrm{Re}f(z),\quad Q = \mathrm{Im}f(z) $$
    那么$f(z)$在$D$内解析,$\forall x\in D$
    $$|g(z)|^2 = \dfrac{P^2 + Q^2}{(2A-P)^2 + Q^2} \leq \dfrac{R^2 + Q^2}{A^2 + Q^2} \leq 1$$
    从而$|g(z)| \leq 1,\ g(0) = 0$,于是由\textup{Schwarz}引理得
    $$|g(z) | \leq z$$
    由$g(z)$定义知
    $$f(z) = \dfrac{2Ag(z)}{1 + g(z)}$$
    则
    $$ |f(z)| \leq \dfrac{2A(z)}{1-|z|} \leq \dfrac{2Ar}{1-r}$$
    即
    $$M(r) \leq \dfrac{2Ar}{1-r}$$

\end{proof}

\begin{lemma}[Schwarz引理]
    
    设$f(z)$在单位圆盘$|z|<1$内解析,在闭圆盘$|z| \leq 1$上是连续的,并且满足$f(0) = 0$,又在$|z| < 1$内处处有$|f(z)| < 1$,则

    \begin{enumerate}
        
        \item 
            $|f(z)| \leq |z|$,且$|f'(0)| \leq 1$
        
        \item 
            若在开圆盘内有一复数$z_0 \neq 0$,是$|f(z_0)| = |z_0|$,或者 $|f'(0)| = 1$,那么$f(z) = e^{\imag\alpha}z$,$\alpha$为一实数。
        
    \end{enumerate}

\end{lemma}

\begin{proof}
    
        
    由于$f(0) = 0$,且$f(z)$在$|z| < 1$内解析,则$f(z)$在$|z|<1$内可以\textup{Taylor}展开,有
    $$f(z) = c_1 z + c_2 z^2 + \dots + $$
    令$\varphi(z) =  c_1 z + c_2 z^2 + \cdots $,则
    $$
    \varphi(z) = 
    \left\{
        \begin{aligned}
            &\dfrac{f(z)}{z}, & z \neq 0\\
            &f'(0), & z = 0\\
        \end{aligned}
    \right.
    $$
    显然$\varphi(z)$在$|z| < 1$内解析,对于圆内一点$z_0$,设$|z_0| < r < 1$,则由最大模定理
    $$|\varphi(z+0)| = \max_{|z| = r}|\varphi(z_0)| = \max_{|z| = r}{\left| \dfrac{f(z)}{z}\right|} \leq \dfrac{1}{r}$$
    令$r \to 1$,则有$|\varphi(z_0)| \leq 1$\\
    若$z_0 = 0$,有$|f'(0)| = |\varphi(0)| \leq 1$\\
    若$z_0 \neq 0$,有$|f'(0)| = \left| \dfrac{f(z_0)}{z_0} \right| \leq 1$,即
    $$|f(z_0)| \leq |z_0|$$
    由于$f(0) = 0$,上式当$z=0$时成立,则
    $$|f(z_0)| \leq |z_0|,\quad x\in \Delta(0,1)$$
    若有$z_0 \neq 0$,使$|f(z_0)| = |z_0| $,或者$|f'(0)| = 1$,则$\varphi(z)$在$\Delta(0,1)$内取得最大模,\\
    从而$\varphi(z) \equiv c$, 又$|c| = 1$,则$\varphi(z) = \euler^{\imag\alpha}$,所以
    $$f(z) = \euler^{\imag\alpha}z$$

\end{proof}

\begin{theorem}[Schwarz-Pick定理]

    设$f:D\to\mathbb{C}$在$D$内解析,$|f(z)| \leq 1$,且$f(z_0) = w_0,\ z\in D$,则
    $$ \left| \dfrac{f(z)-w_0}{1-\overline{w_0}f(z)}\right| \leq \left| \dfrac{z-z_0}{1 - z_0z}\right|$$
    等号当且仅当$f(z)$是分式线性映射时成立。

\end{theorem}

\begin{proof}
    
    设函数$w =f(z)$,考虑分式线性映射
    $$\zeta = T(z) = \dfrac{z - z_0}{1 - \overline{z_0}z},\quad \tau = S(w) \dfrac{w - w_0}{1 - \overline{w_0}w}$$
    分别$z$平面和$w$平面上的单位圆双方单值映射成$\zeta$和$\tau$平面上的单位元,并且把$z_0,\ w_0$映射成$\zeta = 0,\ \tau = 0$,则
    $$F(\zeta) = S(f(T^{-1}(\zeta)))$$
    在$|\zeta| < 1$内解析,$|F(\zeta)| \leq 1$,且$F(0) = 0$,于是在$|\zeta| < 1$内,由\textup{Schwarz}引理得
    $$|F(\zeta)| = |S((f(T^{-1}(\zeta))))| \leq |\zeta|$$
    从而在$|z|<1$内
    $$|S(f(z))| \leq |T(z)|$$
    即
    $$ \left| \dfrac{f(z)-w_0}{1-\overline{w_0}f(z)}\right| \leq \left| \dfrac{z-z_0}{1 - z_0z}\right|$$
    
\end{proof}